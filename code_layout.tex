The global layout can be understood as different
parts working together:
\begin{itemize}
\item the core part;
\item the thermodynamics part;
\item the chemistry part composed of;
        \begin{itemize}
        \item the kinetics model part;
        \item the chemical process part.
        \end{itemize}
\end{itemize}

%templ/.style={rounded corners,blue,-stealth},
%point/.style={rounded corners,violet,-stealth,double}
%The relationships between the objects are as follow:
\begin{itemize}
\item \tikz[baseline=(a.base)]\draw[red,-stealth,text=black] (0,0)node(a)[left]{A} -- (1,0) node[right]{B};: 
                        B contains a reference to A;
\item \tikz[baseline=(a.base)]\draw[stealth-,densely dotted,text=black] (0,0)node(a)[left]{A} -- (1,0) node[right]{B};: 
                        A derives from B;
\item \tikz[baseline=(a.base)]\draw[blue,-stealth,text=black] (0,0)node(a)[left]{A} -- (1,0) node[right]{B};: 
                        B is templated around objects of type A;
\item \tikz[baseline=(a.base)]\draw[violet,double,-stealth,text=black] (0,0)node(a)[left]{A} -- (1,0) node[right]{B};: 
                        B contains a/several pointers of type A;
\item \tikz[baseline=(a.base)]\draw[violet,double,-stealth,text=black] (0,0)node(a)[left]{A} -- (1,0) node[right]{B};: 
                        B contains a/several pointers of type A;
\item \tikz[baseline=(a.base)]\draw[)-,text=black] (0,0)node(a)[left]{A} -- (1,0) node[right]{B};: 
                        B contains an instanciation of A;
\end{itemize}
