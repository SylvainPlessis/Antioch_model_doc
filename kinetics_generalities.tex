A chemical reaction is ``a bunch of molecules
turning into another bunch of molecules'', kinetics\footnote{%
from the greek $\kappa\iota\nu\eta\sigma\iota\varsigma$, ``kinesis'', movement, to move}
is about answering the question ``how fast?''.
Thus chemical kinetics is the mathematical model
to calculate the rate at which the molecules disappear and
appear.

\antioch's kinetics is based on the elementary step
hypothesis. It means that, as far as the kinetics is
concerned, every reaction is an elementary step:
the reactants get together and produce the products
immediatly. Mathematically, it means
the partial orders are the absolute
value of the stoichiometric coefficients (see next).

Let's consider a chemical reaction:
\begin{chemicalEquation}
\ce{\scoefabs[A] A + \scoefabs[B] B ->[\rcons] \scoefabs[C] C + \scoefabs[D] D}
\label{genericX}
\end{chemicalEquation}
with \rcons\ the rate constant.
We want to model the evolution of the system, that is we want to
characterize 
$\doverdt{\conc[A]}$,
$\doverdt{\conc[B]}$,
$\doverdt{\conc[C]}$,
$\doverdt{\conc[D]}$.
Using the kinetics theory, we have:
\begin{equation}
\frac{1}{\scoef[A]}\doverdt{[A]} = 
\frac{1}{\scoef[B]}\doverdt{[B]} = 
\frac{1}{\scoef[C]}\doverdt{[C]} = 
\frac{1}{\scoef[D]}\doverdt{[D]} = 
\rcons\conc[A]^{\scoefabs[A]}\conc[B]^{\scoefabs[B]}
\end{equation}
with \scoef[A]\ being the stoichiometric coefficient, which is defined by:\\[5pt]
$\left\{\begin{array}{ll}
\scoef[S] = \scoefabs[S] & \text{if \ce{S} is a product} \\
\scoef[S] = -\scoefabs[S] & \text{if \ce{S} is a reactant} \\
\end{array}\right.$\\[5pt]
So the game is to define the rate constant. 

A rate constant is characterized by two things:
\begin{itemize}
\item the kinetics model,
\item the chemical process.
\end{itemize}
The kinetics model will mathematically describe the rate constant's dependence with
the temperature, it is noted \kinMod\ in this manual, the chemical process will
possibly add a pressure dependency, it is noted \chemProc, with \conc[M]\
denoting the pressure dependence.
\antioch\ propose six different kinetics models and five chemical processes.
The rate constant is characterized thus, for a choice of a chemical process and
a kinetics model:
\begin{equation}
\rateCons = \chemProc
\end{equation}
