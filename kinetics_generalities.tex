Let's consider a chemical reaction:
\begin{chemicalEquation}
\ce{\scoefabs[A] A + \scoefabs[b] B ->[\rcons] \scoefabs[C] C + \scoefabs[D] D}
\label{youpa}
\end{chemicalEquation}
with \rcons\ the rate constant.
We want to model the evolution of the system, that is we want to
characterize 
$\doverdt{\conc[A]}$,
$\doverdt{\conc[B]}$,
$\doverdt{\conc[C]}$,
$\doverdt{\conc[D]}$.
Using the kinetics theory, we have:
\begin{equation}
\frac{1}{\scoef[A]}\doverdt{[A]} = 
\frac{1}{\scoef[B]}\doverdt{[B]} = 
\frac{1}{\scoef[C]}\doverdt{[C]} = 
\frac{1}{\scoef[D]}\doverdt{[D]} = 
\rcons\conc[A]^{\scoefabs[A]}\conc[B]^{\scoefabs[B]}
\end{equation}
with \scoef[A]\ being the stoichiometric coefficient, which is defined by:\\
$\left\{\begin{array}{ll}
\scoef[S] = \scoefabs[S] & \text{if \ce{S} is a product} \\
\scoef[S] = -\scoefabs[S] & \text{if \ce{S} is a reactant} \\
\end{array}\right.$\\
So the game is to define the rate constant. 

A rate constant is characterized by two things:
\begin{itemize}
\item the kinetics model,
\item the chemical process.
\end{itemize}
The kinetics model will mathematically describe the rate constant's dependence with
the temperature, it is noted \kinMod\ in this manual, the chemical process will
possibly add a pressure dependency, it is noted \chemProc, with \conc[M]\
denoting the pressure dependence.
\antioch\ propose six different kinetics models and five chemical processes.
The rate constant is characterized thus, for a choice of a chemical process and
a kinetics model:
\begin{equation}
\rateCons = \chemProc
\end{equation}

\begin{table}
\centering\renewcommand{\arraystretch}{1.5}
\begin{tabular}{cl}\toprule
Kinetics model  & Expression \\\midrule
Hercourt-Hessen & $\kinMod = \PreExp \left(\frac{\Temp}{\Tref}\right)^\Power$ \\
Berthelot       & $\kinMod = \PreExp \exp\left(\BerthExp\Temp\right)$ \\
Arrhenius       & $\kinMod = \PreExp \exp\left(-\frac{\AcEn}{\Rg\Temp}\right)$ \\
Kooij           & $\kinMod = \PreExp \left(\frac{\Temp}{\Tref}\right)^\Power\exp\left(-\frac{\AcEn}{\Rg\Temp}\right)$ \\
Berthelot Hercourt-Essen
                & $\kinMod = \PreExp \left(\frac{\Temp}{\Tref}\right)^\Power\exp\left(\BerthExp\Temp\right)$ \\
Van't Hoff      & $\kinMod = \PreExp \left(\frac{\Temp}{\Tref}\right)^\Power\exp\left(\BerthExp\Temp-\frac{\AcEn}{\Rg\Temp}\right)$ \\
\bottomrule
\end{tabular}
\caption{\label{antioch::kinMod}Kinetics models available in \antioch.}
\end{table}

\begin{table}
\centering\renewcommand{\arraystretch}{2}
\begin{tabular}{cl}\toprule
Chemical process  & Expression \\\midrule
Elementary        & $\chemProc = \kinMod$ \\
Duplicate \dag    & $\chemProc = \displaystyle\sum_i^{\mathrm{N_{proc}}}\kinMod_i$\\
Three-Body        & $\chemProc = \kinMod_i \threeBody$\\
Lindemann falloff & $\chemProc = \frac{\conc[M]\kinMod_0}{1 + \conc[M]\frac{\kinMod_0}{\kinMod\infty}}F_{\text{Lind}}$\\
Troe falloff      & $\chemProc = \frac{\conc[M]\kinMod_0}{1 + \conc[M]\frac{\kinMod_0}{\kinMod\infty}}F_{\text{Troe}}$\\
\bottomrule
\end{tabular}
\caption{\label{antioch::chemProd}Chemical processes available in \antioch.
\dag: the duplicate chemical process do not permit several kinetics
models to be mixed. The functions $F$ for the falloff are described in section~\ref{falloff}}
\end{table}
