\subsubsection{Kinetics models}
The kinetics models are all given in Tab.~\ref{antioch::kinMod}.
The kinetics model will render the temperature evolution of
the rate constant, thus the only differential that is needed
at that level is the temperature differentiation.
\begin{table}
\centering\renewcommand{\arraystretch}{1.5}
\begin{tabular}{clr}\toprule
\multirow{2}{*}{Kinetics model}
                & Expression                     & Parameters \\
                & $\frac{\dd}{\dd T}$ Expression &\\\midrule
\multirow{2}{*}{Hercourt-Hessen} 
                & $\kinMod = \PreExp \left(\frac{\Temp}{\Tref}\right)^\Power$ 
                        & \multirow{2}{*}{\PreExp, \Power}\\
                & $\ddoverdT{\kinMod} = \kinMod \frac{\Power}{\Temp}$ \\[10pt]
\multirow{2}{*}{Berthelot}
                & $\kinMod = \PreExp \exp\left(\BerthExp\Temp\right)$ 
                        & \multirow{2}{*}{\PreExp, \BerthExp}\\
                & $\ddoverdT{\kinMod} = \kinMod \BerthExp$ \\[10pt]
\multirow{2}{*}{Arrhenius}
                & $\kinMod = \PreExp \exp\left(-\frac{\AcEn}{\Rg\Temp}\right)$ 
                        & \multirow{2}{*}{\PreExp, \AcEn}\\
                & $\ddoverdT{\kinMod} = \kinMod \frac{\AcEn}{\Rg\Temp^2}$ \\[10pt]
\multirow{2}{*}{Kooij}
                & $\kinMod = \PreExp \left(\frac{\Temp}{\Tref}\right)^\Power\exp\left(-\frac{\AcEn}{\Rg\Temp}\right)$
                        & \multirow{2}{*}{\PreExp, \Power, \AcEn}\\
                & $\ddoverdT{\kinMod} = \kinMod \left(\frac{\Power}{\Temp} + \frac{\AcEn}{\Rg\Temp^2}\right)$ \\[10pt]
\multirow{2}{*}{Berthelot Hercourt-Essen}
                & $\kinMod = \PreExp \left(\frac{\Temp}{\Tref}\right)^\Power\exp\left(\BerthExp\Temp\right)$
                        & \multirow{2}{*}{\PreExp, \Power, \BerthExp}\\
                & $\ddoverdT{\kinMod} = \kinMod \left(\frac{\Power}{\Temp} + \BerthExp\right)$ \\[10pt]
\multirow{2}{*}{Van't Hoff}
                & $\kinMod = \PreExp \left(\frac{\Temp}{\Tref}\right)^\Power\exp\left(\BerthExp\Temp-\frac{\AcEn}{\Rg\Temp}\right)$
                        & \multirow{2}{*}{\PreExp, \Power, \BerthExp, \AcEn}\\
                & $\ddoverdT{\kinMod} = \kinMod \left(\frac{\Power}{\Temp} + \BerthExp +  \frac{\AcEn}{\Rg\Temp^2}\right)$ \\
\bottomrule
\end{tabular}
\caption{\label{antioch::kinMod}Kinetics models available in \antioch.}
\end{table}

\subsubsection{Chemical processes}
\label{subsec:chem_proc}
The chemical processes is how this kinetics model behave with respect to
the densities of the molecules in the medium. Either they're all equivalent,
and only the total density matters (falloff type), or each species can 
influence the rate constant in a specific manner (three-body type).
The chemical processes are given in Tab.~\ref{antioch::chemProc}. This is where
chemical modeling begins to be stretched a little: as seen on section~\ref{kinetics_gen},
we assume that every reaction modeled is an elementary step, even in the case of
non-elementary reactions. Therefore some corrections need to be made. This is where
the chemical process adds, if needed, more suppleness to the rate constant
function, from adding degrees of freedom to the temperature dependency (duplicate
chemical process) to adding a pressure dependance over the temperature dependance
(falloff).
\begin{table}
\centering\renewcommand{\arraystretch}{2}
\begin{tabular}{ccc}\toprule
\multirow{2}{*}{Chemical process}
                  & \multirow{2}{*}{Expression}
                                           & $\doverdT{\text{Expression}}$ \\
                  &                        & $\doverdc{\text{Expression}}$ \\\midrule
\multirow{2}{*}{Elementary}
                  & \multirow{2}{*}{$\chemProc = \kinMod$}  
                                           & $\doverdT{\chemProc} = \ddoverdT{\kinMod}$ \\
                  &                        & $\doverdc{\chemProc} = 0$ \\[10pt]
\multirow{2}{*}{Duplicate \dag}
                  & \multirow{2}{*}{$\chemProc = \displaystyle\sum_i^{\mathrm{N_{proc}}}\kinMod_i$}
                                           & $\doverdT{\chemProc} = \displaystyle\sum_i^\mathrm{N_{proc}}\ddoverdT{\kinMod_i}$ \\
                  &                        & $\doverdc{\chemProc} = 0$ \\[10pt]
\multirow{2}{*}{Three-Body}
                  & \multirow{2}{*}{$\chemProc = \kinMod \threeBody$}
                                           & $\doverdT{\chemProc} = \ddoverdT{\kinMod}\threeBody$ \\
                  &                        & $\doverdc[I]{\chemProc} = \kinMod\epsilon_i\sum_{s\neq i}\epsilon_s\conc[S] $ \\[10pt]
\multirow{2}{*}{Lindemann falloff}
                  & \multirow{2}{*}{$\chemProc = \frac{\conc[M]\kinModZ}{1 + \conc[M]\frac{\kinModZ}{\kinModI}}\FLind$}
                                          & \ref{Falloff:doverdT} with $F = \FLind$\\
                  &                       & \ref{Falloff:doverdc} with $F = \FLind$\\[10pt]
\multirow{2}{*}{Troe falloff}
                  & \multirow{2}{*}{$\chemProc = \frac{\conc[M]\kinModZ}{1 + \conc[M]\frac{\kinModZ}{\kinModI}}\FTroe$}
                                          & \ref{Falloff:doverdT} with $F = \FTroe$\\
                  &                       & \ref{Falloff:doverdc} with $F = \FTroe$\\[10pt]
\bottomrule
\end{tabular}
\caption{\label{antioch::chemProc}Chemical processes available in \antioch.
\dag: the duplicate chemical process do not permit several kinetics
models to be mixed. The functions $F$ for the falloff are described in section~\ref{subsub:falloff}}
\end{table}

\subsubsection{Falloffs}
\label{subsub:falloff}
The falloffs are reaction having two different regimes: a low-pressure (\kinModZ) and
a high-pressure (\kinModI) regime. Any pressure's rate constant can be expressed in
terms if those two regimes:
\begin{equation}
\chemProc = \frac{\conc[M]\kinModZ}{1 + \conc[M]\frac{\kinModZ}{\kinModI}} F
\label{falloff:k}
\end{equation}
with \conc[M] being the total molar density and $F$ the falloff model.

The differentiations are:
\begin{equation}
\begin{split}
\doverdT{\chemProc} & = \chemProc 
                        \Bigg[\frac{1}{F}\doverdT{F} \\
                    &      + \left(\frac{1}{\kinModZ} - \frac{\conc[M]}{\kinModI}\frac{1}{1 + \conc[M]\frac{\kinModZ}{\kinModI}}\right)\doverdT{\kinModZ} \\
                    &      - \conc[M]\frac{\kinModZ}{\kinModI^2}\frac{1}{1 + \conc[M]\frac{\kinModZ}{\kinModI}}\doverdT{\kinModI}
                        \Bigg]
\end{split}
\label{Falloff:doverdT}
\end{equation}
and
\begin{equation}
\doverdc[I]{\chemProc} = \chemProc
                          \left(\frac{1}{\conc[M]} 
                               + \frac{\kinModZ}{\kinModI}\frac{1}{1 + \conc[M]\frac{\kinModZ}{\kinModI}} 
                               + \doverdT{F}\frac{1}{F}
                          \right)
\label{Falloff:doverdc}
\end{equation}
\paragraph{Lindemann}
The Lindemann falloff is the simplest falloff, it is given by:
\begin{equation}
\FLind = 1,
\label{Falloff:Lindeman}
\end{equation}
which gives:
\begin{equation}
\begin{split}
\doverdT{\FLind} & = 0 \\
\doverdc{\FLind} & = 0 \\
\end{split}
\label{Falloff:Lindemann:diff}
\end{equation}

\paragraph{Troe}
The Troe model is more elaborate and rely on the definition of three or
four parameters:
\begin{itemize}
\item \Troealpha, with \nounit,
\item \TroeTone\ in \unit{K},
\item \TroeTtwo\ in \unit{K},
\item \TroeTthree\ in \unit{K}.
\end{itemize}
Sometimes, the parameter \TroeTtwo\ is not provided, as its contribution
may be negligible.
The falloff calculations need the definition of several parameters:
\begin{equation}
\begin{split}
\TroeFcent & = \left(1 - \Troealpha\right)\exp\left(-\frac{\Temp}{\TroeTthree}\right)
              +          \Troealpha       \exp\left(-\frac{\Temp}{\TroeTone}\right)
              +                           \exp\left(-\frac{\TroeTtwo}{\Temp}\right) \\[5pt]
\Troen     & =   \numprint{0.75} - \numprint{1.27}\log_{10}\left(\TroeFcent\right)  \\[5pt]
\Troec     & = - \numprint{0.4}  - \numprint{0.64}\log_{10}\left(\TroeFcent\right)  \\[5pt]
\Troed     & =   \numprint{0.14}
\end{split}
\label{Troe:parameter}
\end{equation}
to finally have
\newcommand{\pr}{\ensuremath{\log_{10}\left(\frac{\conc[M]\kinMod_0}{\kinMod_\infty}\right)}}
\begin{equation}
\FTroe = \frac{\log_{10}\left(\TroeFcent\right)}
              {1 + \left[
                        \log_{10}\left(
                                        \frac{\pr + \Troec}
                                             {\Troen - \Troed\left(
                                                                \pr + \Troec
                                                            \right)}
                                 \right) 
                   \right]^2}
\label{Troe:F}
\end{equation}
Thus the differentials will be:
\begin{equation}
\begin{split}
\doverdT{\TroeFcent} & = - \frac{1 - \Troealpha}{\TroeTthree}\exp\left(-\frac{\Temp}{\TroeTthree}\right)
                         - \frac{\Troealpha}{\TroeTone}\exp\left(-\frac{\Temp}{\TroeTone}\right)
                         + \frac{\TroeTtwo}{\Temp^2}\exp\left(-\frac{\TroeTtwo}{\Temp}\right)\\[5pt]
\doverdT{\Troen}     & = - \frac{\numprint{1.27}}{\ln(10)\TroeFcent}\doverdT{\TroeFcent} \\[5pt]
\doverdT{\Troec}     & = - \frac{\numprint{0.64}}{\ln(10)\TroeFcent}\doverdT{\TroeFcent} \\[5pt]
\doverdT{\Troed}     & = 0
\end{split}
\label{Troe:dparameterdT}
\end{equation}
and of course
\begin{equation}
\doverdc{\TroeFcent}  = 0,\qquad
\doverdc{\Troen}      = 0,\qquad
\doverdc{\Troec}      = 0,\qquad
\doverdc{\Troed}      = 0
\label{Troe:dparameterdc}
\end{equation}
For simplicity of writing, we define:
\begin{equation}
\begin{split}
g(\kinMod_0,\kinMod_\infty) = & \conc[M] \frac{\kinMod_0}{\kinMod_\infty} \\[5pt]
h(\kinMod_0,\kinMod_\infty) = & \log_{10}(g) + \Troec \\[5pt]
l(\kinMod_0,\kinMod_\infty) = & \log_{10}\left(\frac{h}{\Troen - \Troed h}\right) \\
\end{split}
\end{equation}
Therefore:
\begin{equation}
\FTroe = \frac{\log_{10}\left(\TroeFcent\right)}{1 + l^2}
\end{equation}
When we derive, we obtain:
\begin{equation}
\begin{split}
\doverdT{g} & = g \left(\frac{1}{\kinMod_0}\doverdT{\kinMod_0} - \frac{1}{\kinMod_\infty}\doverdT{\kinMod_\infty} \right) \\[5pt]
\doverdc{g} & = g \frac{1}{\conc[M]}\doverdc{\conc[M]}
\end{split}
\end{equation}
\begin{equation}
\begin{split}
\doverdT{h} & = \frac{1}{g \ln(10)}\doverdT{g} + \doverdT{\Troec} \\[5pt]
\doverdc{h} & = \frac{1}{g\ln(10)}\doverdc{g}
\end{split}
\end{equation}
\begin{equation}
\begin{split}
\doverdT{l} & = \frac{1}{h \ln(10)}\left[\left(1 + \frac{\Troed h}{\Troen - \Troed h}\right) \doverdT{h} - \frac{h}{\Troen - \Troed h}\doverdT{\Troen} \right]  \\[5pt]
\doverdc{l} & = \frac{1}{h \ln(10)}\left(1 + \frac{h\Troed}{\Troen - \Troed h} \right)\doverdT{h}
\end{split}
\end{equation}
Thus, we have:
\begin{equation}
\doverdT{\FTroe} = \FTroe \left[
                                  \frac{1}{\ln(10)\log_{10}(\TroeFcent)}\doverdT{\TroeFcent}
                                - \frac{2 l}{1 + l^2} \doverdT{l}
                          \right]
\end{equation}
and
\begin{equation}
\doverdc{\FTroe} = \FTroe \left[
                                  \frac{1}{\ln(10)\log_{10}(\TroeFcent)}\doverdc{\TroeFcent}
                                - \frac{2 l}{1 + l^2} \doverdc{l}
                          \right]
\end{equation}
\begin{figure}
\centering
\includegraphics{falloffs}
\caption{\label{kinetics::falloffs}%
Example of falloff for the reaction \ce{CH3 + CH3 (+M) -> C2H6 (+M)}.
The rate constants are calculated for a temperature of $T = 1000~\unit{K}$. The kinetics model
used is a Kooij model. Parameters are:
$\PreExp_0      = \numprint{1.135}\,10^{36}~\unit{mol^{-2}\,cm^6\,s^{-1}}$,
$\Power_0       = \numprint{-5.245}$,
$\AcEn_0        = \numprint{1704.8}~\unit{cal\,mol^{-1}}$,
$\PreExp_\infty = \numprint{6.22}\,10^{16}~\unit{mol^-2\,cm^6\,s^{-1}}$,
$\Power_\infty  = \numprint{-1.174}$,
$\AcEn_\infty   = \numprint{653.8}~\unit{cal\,mol^{-1}}$,
$\Troealpha     = \numprint{0.405}$,
$\TroeTone      = \numprint{69.6}~\unit{K}$ and
$\TroeTthree    = \numprint{1120.0}~\unit{K}$.}
\end{figure}
