%first thing first
\graphicspath{{./figs/}}
%%%% meta-meta-function
\newcommand{\make@}[3][upshape]{%
\expandafter\gdef\csname font@#2\endcsname##1{% create font@something macro
{\csname#1\endcsname\csname#3\endcsname{##1}}%
}%
\expandafter\gdef\csname make@#2\endcsname##1{% create make@something meta macro
\expandafter\gdef\csname ##1\endcsname{\csname font@#2\endcsname{##1}}%
}
\expandafter\gdef\csname#2\endcsname##1{\csname font@#2\endcsname{##1}}%create the \something macro
}
%% meta font
\newcommand{\prog@type}[1]{\textcolor{green!60!black}{\object{#1}}}
%% meta function
\make@{program}{sf}
\make@{library}{tt}
\make@[bf]{object}{tt}
\make@{file}{sf}
\make@{prog}{prog@type}
%% macros
\make@program{Doxygen}
\make@library{Antioch}
\make@library{antioch}
\make@library{Boost}
\make@library{VexCL}
\make@library{ViennaCL}
\make@library{MetaPhysicL}
\make@library{GRVY}
\make@library{EIGEN}
%% for tutorial
\make@object{ReactionSet}
\make@object{KineticsEvaluator}
\make@object{ChemicalMixture}
\make@object{CEAThermodynamics}
\make@object{Units}
\make@prog{double}
\newcommand{\stdvector}{\prog{std::vector< >}}
\newcommand{\stdstring}{\prog{std::string}}
%%physics
\newcommand{\phenomenom}{\textcolor{red}{phenomenom}}
\newcommand{\quantity}{\textcolor{blue!60!black}{quantity}}
\newcommand{\Quantity}{\textcolor{blue!60!black}{Quantity}}
\newcommand{\model}{\textcolor{green!60!black}{model}}
%

%%% versions
\newcommand{\version}[3]{%
\setcounter{vmajor}{#1}
\setcounter{vmedium}{#2}
\setcounter{vminor}{#3}
}
\newcounter{vmajor}
\newcounter{vmedium}
\newcounter{vminor}
\newcommand{\theversion}{\thevmajor.\thevmedium.\thevminor}

%%% constant equation management
\newcommand{\make@equation}[6][0] %% name, symbol, value, dvalue, unit | symbol = value (dvalue) \unit{unit}
{%
\expandafter\gdef\csname#2\endcsname{\ensuremath{#3}}%                                                          \"name"     produces "symbol"
\expandafter\gdef\csname#2val\endcsname{\numprint{#4}\ifnum#1=0\else\,10^{#1}\fi}%                              \"name"val  produces "value"
\expandafter\gdef\csname#2dval\endcsname{\numprint{#5}}%                                                        \"name"dval produces "dvalue"
\expandafter\gdef\csname#2num\endcsname{\numprint{#4}\,(\csname#2dval\endcsname)\ifnum#1=0\else\,10^{#1}\fi}%   \"name"num  produces "value (dvalue)"
\expandafter\gdef\csname#2unit\endcsname{\unit{#6}}%                                                            \"name"unit produces "unit"
\expandafter\gdef\csname#2Equation\endcsname{% %% equation
 \ensuremath{\csname#2\endcsname = \csname#2num\endcsname~\csname#2unit\endcsname}}%                            \"name"Equation produces "symbol = value (dvalue)~unit"
}

%% fun
\newcommand{\ANTIOCH}{\font@library{A}
                      \font@library{N}ew
                      \font@library{T}emplated
                      \font@library{I}mplementation
                      \font@library{O}f
                      \font@library{C}hemistry
                      \font@library{H}ydrodynamics}
\newcommand{\ANTIOCHPhys}{\Antioch's
                         \font@library{N}oble
                         \font@library{T}ry to
                         \font@library{I}mitate the
                         \font@library{O}rdered
                         \font@library{C}osmos
                         \font@library{H}umbly}
\newcommand{\ANTIOCHTech}{\Antioch's
                         \font@library{N}ot
                         \font@library{T}emplated
                         \font@library{I}rresponsibly, this
                         \font@library{O}utstanding
                         \font@library{C}ode is
                         \font@library{H}appiness}
\newcommand{\ANTIOCHPrac}{\Antioch\
                         \font@library{N}ewbies
                         \font@library{T}empting
                         \font@library{I}nvitation to
                         \font@library{O}pen this
                         \font@library{C}hapter with
                         \font@library{H}aste}
%% I like my equations this way
\renewcommand{\theequation}{Eq.~\thesection-\arabic{equation}}
\@addtoreset{equation}{section}
%%% beauty here
\pagestyle{fancy}
\renewcommand{\chaptermark}[1]{\markboth{#1}{}}
\renewcommand{\sectionmark}[1]{\markright{\thesection~\sl #1}}
\geometry{headheight=2cm}
\fancyhead{} %empty
\fancyhead[LO]{\parbox[b]{0.45\textwidth}{\raggedright\leftmark}}
\fancyhead[RE]{\parbox[b]{0.45\textwidth}{\raggedleft\leftmark}}
\fancyhead[LE]{\parbox[b]{0.45\textwidth}{\raggedright\rightmark}}
\fancyhead[RO]{\parbox[b]{0.45\textwidth}{\raggedleft\rightmark}}
%% a little hack
\newcommand{\spacehack}{\ensuremath{\textcolor{white}{=}}}

%%%%%%%%%%%%%%%%%%%%%%%%%%%%%%%%%%%%
%%%%
%%%%  stuffs start here
%%%%
%%%%%%%%%%%%%%%%%%%%%%%%%%%%%%%%%%%%

%layout stuff
\newlength{\tocenter}
\newcommand{\graphAtPaperCenter}[2][width=\paperwidth]
{
%%%% from geometry package
%% inner/outer margin ratio 2:3
%% => \parperwidth - \textwidth = 5
%% center = (inner +) 0.5\textwidth + (\parperwidth - \textwidth)/10 %% offset from inner (odd page)
%% center = (outer +) 0.5\textwidth - (\parperwidth - \textwidth)/10 %% offset from outer (even page)
%%%%%%%%%%%
\setlength{\tocenter}{\paperwidth}
\addtolength{\tocenter}{-\textwidth}
\ifodd\number\c@page% TeX register of LaTeX counter, late one page
  \setlength{\tocenter}{0.1\tocenter}
\else% even
  \setlength{\tocenter}{-0.1\tocenter}
\fi%
\addtolength{\tocenter}{0.5\textwidth}
\noindent\hspace{\tocenter}\makebox[0pt]{\includegraphics[#1]{#2}}\par
}

%% chapter
\@addtoreset{chapter}{part}

% unit defs
\newcommand{\unitbase}{[\unit{m},\unit{kg},\unit{s},\unit{A},\unit{K},\unit{mol},\unit{cd},\unit{rad}]}
%%%%%%%
%%
%% 'cause I'm lazy 
%%
%%%%
\newcommand{\example}[1]{\textit{i.e.} #1}


%internal stuff
\newcommand{\optional@sub} [2][\relax]{\ifx#1\relax\ensuremath{#2}\else\ensuremath{{{#2}_{#1}}}\fi}
\newcommand{\optional@@sub}[2][\relax]{\ifx#1\relax\ensuremath{#2}\else\ensuremath{{#2}_{#1}}\fi}
\newcommand{\d@d} [2]{\ensuremath{\frac{\partial#2}{\partial #1}}}
\newcommand{\dd@d}[2]{\ensuremath{\frac{\dd #2}{\dd #1}}}

%%%%%%%%%%%%%%%%%%%%%%%%%%%%%
%%
%% universal constant, using
%% meta macro make@equation
%%
%%%%%%%%%%%%%%%%%%%%%%%%%%%%%%

\make@equation     {Rg}       {\mathrm{R}}            {8.3144621} {75}{J\,mol^{-1}\,K^{-1}}
\make@equation[-23]{Boltzmann}{\mathrm{k}_\text{B}}   {1.3806488} {13}{J\,K^{-1}}
\make@equation[23] {Navo}     {\mathcal{N}_\text{Avo}}{6.02214129}{27}{mol^{-1}}

%%%%%%%%%%%%%%%%%%%%%%%%%%%%%%%%%%%%
%%
%% mathematical stuff
%%
%%%%%%%%%%%%%%%%%%%%%%%%%%%%%%%%%%%%
\newcommand{\doverdt}    [1]{\d@d{t}{#1}}
\newcommand{\doverdT}    [1]{\d@d{\Temp}{#1}}
\newcommand{\ddoverdT}   [1]{\dd@d{\Temp}{#1}}
\newcommand{\doverdm}    [2][\relax]{\d@d{\mass[#1]}{#2}}
\newcommand{\doverdc}    [2][\relax]{\d@d{\conc[#1]}{#2}}
\newcommand{\doverdn}    [2][\relax]{\d@d{\Mol[#1]}{#2}}
\newcommand{\doverdext}  [2][\relax]{\d@d{\ext[#1]}{#2}}
\newcommand{\ddoverddext}[3]{\ensuremath{\frac{\partial^2#1}{\partial\ext[#2]\partial\ext[#3]}}}
%%%%%%%%%%%%%%%%%%%%%%%%%%%%%%
%%
%%  chemical definitions
%%
%%%%%%%%%%%%%%%%%%%%%%%%%%%%%
\newcommand{\pz}            {\ensuremath{\mathrm{p^0}}}
\newcommand{\Tz}            {\ensuremath{\mathrm{T^0}}}
\newcommand{\eq}            {\ensuremath{\text{eq}}}
\newcommand{\Rmix}          {\ensuremath{\mathrm{R}_\text{mix}}}
\newcommand{\reac}          {\ce{R}}
\newcommand{\product}       {\ce{P}}
\newcommand{\tc}         [1]{\ensuremath{a_{#1}}}
%%stuff for kinetics
\newcommand{\rcons}   [1][\relax]{\optional@@sub[#1]{k}}
\newcommand{\kinMod}  [1][\relax]{\optional@@sub[#1]{\alpha}\ensuremath{(\Temp)}}
\newcommand{\kinModZ}            {\ensuremath{\alpha_0(\Temp)}}
\newcommand{\kinModI}            {\ensuremath{\alpha_\infty(\Temp)}}
\newcommand{\chemProc}[1][\relax]{\optional@@sub[#1]{\chi}\ensuremath{(\kinMod,\conc[M])}}
\newcommand{\rateCons}           {\ensuremath{\rcons(T,\conc[M])}}
\newcommand{\rate}    [1][\relax]{\optional@@sub[#1]{r}}
\newcommand{\threeBody}          {\ensuremath{\sum_s \epsilon_s\conc[S]}}
\newcommand{\FLind}              {\ensuremath{F_\text{Lind}}}
\newcommand{\FTroe}              {\ensuremath{F_\text{Troe}}}
%%%% kinetics model parameter
\newcommand{\Tref}        {\ensuremath{\mathrm{T_{ref}}}}
\newcommand{\PreExp}      {\ensuremath{A}}
\newcommand{\Power}       {\ensuremath{\beta}}
\newcommand{\BerthExp}    {\ensuremath{D}}
\newcommand{\AcEn}        {\ensuremath{{E_a}}}
\newcommand{\wavelength}  {\ensuremath{{\lambda}}}
\newcommand{\crosssection}{\ensuremath{{\sigma(\wavelength)}}}
%%%% Troe falloff
\newcommand{\Troealpha} {\ensuremath{\alpha}}
\newcommand{\TroeTone}  {\ensuremath{T^{*}}}
\newcommand{\TroeTtwo}  {\ensuremath{T^{**}}}
\newcommand{\TroeTthree}{\ensuremath{T^{***}}}
\newcommand{\TroeFcent} {\ensuremath{F_\text{cent}}}
\newcommand{\Troen}     {\ensuremath{n_\text{T}}}
\newcommand{\Troed}     {\ensuremath{d_\text{T}}}
\newcommand{\Troec}     {\ensuremath{c_\text{T}}}
%
\newcommand{\conc}        [1][\relax]{\ifx#1\relax\molar\else\ce{[#1]}\fi}
\newcommand{\Mm}          [1][\relax]{\optional@sub[#1]{\mathrm{M}}}
\newcommand{\scoefabs}    [1][\relax]{\optional@sub[#1]{\mathrm{n}}}
\newcommand{\scoef}       [1][\relax]{\optional@sub[#1]{\nu}}
\newcommand{\sumscoef}    [1][\relax]{\optional@sub[#1]{\gamma}}
\newcommand{\partialOrder}[1][\relax]{\optional@sub[#1]{\mathrm{m}}}
\newcommand{\orderReac}              {\ensuremath{{\mathrm{m}}}}
\newcommand{\massfrac}    [1][\relax]{\optional@sub[#1]{y}}
\newcommand{\mass}        [1][\relax]{\optional@sub[#1]{\rho}}
\newcommand{\mdot}        [1][\relax]{\optional@sub[#1]{\dot{\omega}}}
\newcommand{\molarfrac}   [1][\relax]{\optional@sub[#1]{x}}
\newcommand{\molar}       [1][\relax]{\optional@sub[#1]{c}}
\newcommand{\Eqconst}     [1][\relax]{\optional@sub[#1]{K}}
\newcommand{\fwdratecons} [1][\relax]{\ensuremath{\rcons[#1]^{(f)}}}
\newcommand{\bkwdratecons}[1][\relax]{\ensuremath{\rcons[#1]^{(b)}}}
\newcommand{\fwdrate}     [1][\relax]{\ensuremath{\rate[#1]^{(f)}}}
\newcommand{\bkwdrate}    [1][\relax]{\ensuremath{\rate[#1]^{(b)}}}
\newcommand{\ext}         [1][\relax]{\optional@sub[#1]{\xi}}
\newcommand{\nspecies}               {\ensuremath{\mathrm{n_{species}}}}
%%thermo
%%%% phase
\newcommand{\Vol}   [1][\relax]{\optional@sub[#1]{V}}
\newcommand{\Press} [1][\relax]{\optional@sub[#1]{P}}
\newcommand{\Temp}  [1][\relax]{\optional@sub[#1]{T}}
\newcommand{\RedTemp}          {\ensuremath{T_r}}
\newcommand{\Mol}   [1][\relax]{\optional@sub[#1]{N}}
\newcommand{\Gibbs} [1][\relax]{\optional@sub[#1]{G}}
\newcommand{\GibbsZ}[1][\relax]{\ensuremath{{\Gibbs[#1]}^{0}}}
\newcommand{\Enth}  [1][\relax]{\optional@sub[#1]{H}}
\newcommand{\Entr}  [1][\relax]{\optional@sub[#1]{S}}
\newcommand{\IntEn} [1][\relax]{\optional@sub[#1]{U}}
\newcommand{\Mass}  [1][\relax]{\optional@sub[#1]{m}}
%%%% species and Delta
\newcommand{\press}   [1][\relax]{\optional@sub[#1]{p}}
\newcommand{\chempot} [1][\relax]{\optional@sub[#1]{\mu}}
\newcommand{\chempotZ}[1][\relax]{\ensuremath{\chempot[#1]^{0}}}
\newcommand{\gibbs}   [1][\relax]{\optional@sub[#1]{g}}
\newcommand{\enth}    [1][\relax]{\optional@sub[#1]{h}}
\newcommand{\entr}    [1][\relax]{\optional@sub[#1]{s}}
\newcommand{\gibbsZ}  [1][\relax]{\ensuremath{{\gibbs[#1]}^{0}}}
\newcommand{\enthZ}   [1][\relax]{\ensuremath{{\enth[#1]}^{0}}}
\newcommand{\entrZ}   [1][\relax]{\ensuremath{{\entr[#1]}^{0}}}
\newcommand{\DGibbs}  [1][\relax]{\ensuremath{\optional@sub[#1]{\Delta}G}}
\newcommand{\DGibbsZ} [1][\relax]{\ensuremath{{\DGibbs[#1]}^{0}}}
\newcommand{\Denth}   [1][\relax]{\optional@sub[#1]{\Delta}\ensuremath{H}}
\newcommand{\DenthZ}  [1][\relax]{\ensuremath{{\Denth[#1]}^{0}}}
\newcommand{\Dentr}   [1][\relax]{\optional@sub[#1]{\Delta}\ensuremath{S}}
\newcommand{\DentrZ}  [1][\relax]{\ensuremath{{\Dentr[#1]}^{0}}}

%%%%%%%%%%%%%%%%%%%%%%%%%%%%%%%%%%%%%%%
%%
%% transport definitions
%%
%%%%%%%%%%%%%%%%%%%%%%%%%%%%%%%%%%%%%%%

\newcommand{\vis}           [1][\relax]{\optional@sub[#1]{\eta}}
\newcommand{\diff}          [1][\relax]{\optional@sub[#1]{D}}
\newcommand{\thermcond}     [1][\relax]{\optional@sub[#1]{\lambda}}
\newcommand{\LJdepth}       [1][\relax]{\optional@sub[#1]{\epsilon}}
\newcommand{\LJdia}         [1][\relax]{\optional@sub[#1]{\sigma}}
\newcommand{\Stockmayer}    [1]        {\ensuremath{\left<\Omega^{(#1,#1)*}\right>}}
\newcommand{\dipole}        [1][\relax]{\optional@sub[#1]{\mu}}
\newcommand{\polarizability}[1][\relax]{\optional@sub[#1]{\alpha}}
\newcommand{\specificHeat}  [1][\relax]{\optional@sub[#1]{C}}
\newcommand{\rotRelax}      [1][\relax]{\optional@sub[\text{rot}#1]{Z}}

%misc
\newcommand{\prodReac}    {\ensuremath{\prod_{\text{reactants}} \conc[\reac]^{\scoefabs[\reac]}}}
\newcommand{\prodReacMass}{\ensuremath{\prod_{\text{reactants}} \left(\frac{\mass[\reac]}{\Mm[\reac]}\right)^{\scoefabs[\reac]}}}
\newcommand{\prodProd}    {\ensuremath{\prod_\text{products}\conc[\product]^{\scoefabs[\product]}}}
\newcommand{\Kooij} [4][1]{\ensuremath{{#2} \left(\frac{T}{#1}\right)^{#3}\exp\left(-\frac{#4}{\Rg T}\right)}}
\newcommand{\KooijEq}     {\Kooij[\mathrm{T_{ref}}]{A}{\beta}{E_a}}
\newcommand{\dd}          {\ensuremath{\mathrm{d}}}
\newcommand{\unit}     [1]{\ensuremath{\mathtt{#1}}}
\newcommand{\nounit}      {\unit{no~unit}}
%%%%%
\newcommand{\reactionEq}[1]{%
\ce{#1}%
}
\newcommand{\ArrjPar}[4]{%
\texttt{Arrhenius} model\\*[2pt]
\null\hspace{12pt}$\left\{\begin{array}{l@{~=~}l}
\PreExp & #1~\unit{#2} \\
\AcEn   & #3~\unit{#4} \\
\end{array}\right.$%
}
\newcommand{\KooijPar}[5]{%
\texttt{Kooij} model\\*[2pt]
\null\hspace{12pt}$\left\{\begin{array}{l@{~=~}l}
\PreExp & #1~\unit{#2} \\
\Power  & #3           \\
\AcEn   & #4~\unit{#5} \\
\end{array}\right.$%
}
\newcommand{\TBcoeff}[2]{%
\ensuremath{\epsilon_{\ce{#1}} = #2}%
}
\newcommand{\reactionParameter}[3][\null]{%
\vspace{5pt minus 3pt}
\begin{minipage}{5.5cm}
\null\hfill#2\hfill\null\\[3pt]
#3\\[5pt]
#1%
\end{minipage}}
%% chemistry equation 
\newcounter{chemeq}
\setcounter{chemeq}{0}
\renewcommand{\thechemeq}{$\chi$~\thesection-\arabic{chemeq}}
\newenvironment{chemicalEquation}
{\begin{displaymath}}
{\refstepcounter{chemeq}\tag{\thechemeq}%
\end{displaymath}}
\@addtoreset{chemeq}{section}

%%%%%%%%%%%%%%%%%%%%%%%%%%%%%%%%
%%
%% code
%% code + output two column
%%
%%%%%%%%%%%%%%%%%%%%%%%%%%%%%%%%
%% language loading
\lstloadlanguages{C++,XML}
\lstset{basicstyle=\footnotesize,commentstyle=\color{blue}\bf,stringstyle=\color{magenta}\bf,
        numbers=left,numbersep=10pt,numberstyle=\scriptsize}

%% c++ code one column
\lstnewenvironment{cpp}[1][\footnotesize]
{\lstset{language=C++,
         morekeywords={Scalar,CoeffType,StateType},keywordstyle=\bf,
         morestring=[b]",
         morecomment=[s]{/*}{*/},morecomment=[l]{//},
         basicstyle=#1}
}
{}

%% c++ code two column
\lstnewenvironment{cpp|}
{\lstset{linewidth=0.9\linewidth,frame=r,framesep=-5pt,
         morekeywords={Scalar,CoeffType,StateType},keywordstyle=\bf,
         morestring=[b]",
         language=C++,morecomment=[s]{/*}{*/},morecomment=[l]{//},
         numbers=none}
}
{}

%% xml code one column
\lstnewenvironment{xml}[1][\footnotesize]
{\lstset{language=XML,
         morestring=[b]",
         basicstyle=#1}
}
{}

% output in terminal
\newlength{\outdepth}
\newcommand{\terminal}[1]{%
\settototalheight{\outdepth}{#1}
\ifdim\outdepth < 4em%
  \setlength{\outdepth}{5em}%
\fi%
\fcolorbox{white}{black}{%
\begin{minipage}[c][2\outdepth][c]{0.42\linewidth}
\color{white}
\textsf{[user@here \$]./program.x}\\
\textsf{#1}
\end{minipage}}
}
