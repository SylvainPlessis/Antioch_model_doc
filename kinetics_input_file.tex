At this point, \antioch-\theversion, only the \verb!xml! format is
supported, but soon some \verb!hdf5! joy will be added.

\subsubsection{The file}

Thus you want to have a proper input file, therefore providing
\Antioch\ with the proper data, as seen in subsection~\ref{kin:data}.
This is pretty straightforward, here what it should look like:
\begin{xml}
<?xml version="1.0"?>
<ctml>
  <reactionData id="some name, does not matter">
    <!-- reaction 0001    -->
    <reaction reversible="yes" type="ThreeBody" id="0001">
      <equation>N2 + M [=] 2 N + M</equation>
      <rateCoeff>
        <Kooij>
           <A units="m3/kmol/s">7.e+18</A>
           <b>-1.6</b>
           <E units="cal/mol">224801.3</E>
        </Kooij>
        <efficiencies default="1.0">N:4.2857 O:4.2857 </efficiencies>
      </rateCoeff>
      <reactants>N2:1.0</reactants>
      <products>N:2.0</products>
    </reaction>

    <!-- reaction 0002    -->
    <reaction reversible="yes" type="ThreeBody" id="0002">
      <equation>O2 + M [=] 2 O + M</equation>
      <rateCoeff>
        <Kooij>
           <A units="m3/kmol/s">2.e+18</A>
           <b>-1.5</b>
           <E units="cal/mol">117881.7</E>
        </Kooij>
        <efficiencies default="1.0">N:5.0 O:5.0</efficiencies>
      </rateCoeff>
      <reactants>O2:1.0</reactants>
      <products>O:2.0</products>
    </reaction>

    <!-- reaction 0003    -->
    <reaction reversible="yes" type="ThreeBody" id="0003">
      <equation>NO + M [=] N + O + M</equation>
      <rateCoeff>
        <Arrhenius>
           <A units="m3/kmol/s">5.e+12</A>
           <E units="cal/mol">149943.0</E>
        </Arrhenius>
        <efficiencies default="1.0">NO:22 N:22 O:22</efficiencies>
      </rateCoeff>
      <reactants>NO:1</reactants>
      <products>N:1 O:1</products>
    </reaction>

    <!-- reaction 0004    -->
    <reaction reversible="yes" type="Elementary" id="0004">
      <equation>N2 + O [=] NO + N</equation>
      <rateCoeff>
        <Kooij>
           <A units="m3/kmol/s">5.7e+9</A>
           <b>0.42</b>
           <E units="cal/mol">85269.6</E>
        </Kooij>
      </rateCoeff>
      <reactants>N2:1 O:1</reactants>
      <products>NO:1 N:1</products>
    </reaction>

    <!-- reaction 0005    -->
    <reaction reversible="yes" type="Elementary" id="0005">
      <equation>NO + O [=] O2 + N</equation>
      <rateCoeff>
        <Arrhenius>
           <A units="m3/kmol/s">8.4e+09</A>
           <E units="cal/mol">38526.0</E>
        </Arrhenius>
      </rateCoeff>
      <reactants>NO:1 O:1</reactants>
      <products>O2:1 N:1</products>
    </reaction>
  </reactionData>
<\ctml>
\end{xml}

The things you want to know about this file are simple:
\begin{itemize}
\item \verb!reactionData! needs to be called \verb!reactionData!, 
        this is hard-coded and imposed;
\item if you place several \verb!reactionData! groups, only the first
        one is considered;
\item every keyword is actually hard-coded and imposed, check
        throughfully your input file before launching it;
\item \Antioch's a little touchy with the formatting of the inputs.
        If you don't provide the units for example, she\footnote{Yes, \Antioch's feminine,
        if you don't like it, quit whining and get over it.} will use
        default units (see Tab.~\ref{unit:default}), but she'll complain.
\item Historically, a \verb!GRI-Mech! file format was used, thus defining
        the defaults. Using \verb!GRI-Mech! files will give you the expected
        behavior, but she'll complain.
\end{itemize}

\subsubsection{Formatting your inputs}

Letting alone the \verb!xml! imposed rules, once you've put
the lines
\begin{xml}
<?xml version="1.0"?>
<ctml>
</ctml>
\end{xml}
you enter into \Antioch's realm, her rules apply from there, don't
resist, there's no hero, you will eventually abide by her rules.
Again, let's insist on the \verb!reactionData! environment, only
a \verb!reactionData! is searched, and only the first is read.
Any meta data given in the \verb!reactionData! tag is ignored.

Before starting, a little vocabulary for this section:
\begin{itemize}
\item an \emph{environment} requires an opening tag
        and a closing tag. Anything in between belongs to it.
\item an \emph{attribute} is within the opening tag of an
        environment. 
\item a \emph{value} is the value associated with either
        an environment or an attribute.
\end{itemize}
An attribute's value is in between quotation marks, an environment's
value is between its tags, with nothing else.
%%%%%% horrible, make it better some time
\begin{center}
\begin{minipage}{9cm}
\tt
\tikz[overlay,baseline={(0,-3pt)}]\draw[red,stealth-] (0,0) -- (-2,0)node[left]{Opening tag};<anEnvironment %
   anAttr\tikz[overlay,baseline={(0,-6pt)}]\draw[red,stealth-](0,0) |- (2,0.5)node[right]{Attribute} ;ibute="attribu%
               \tikz[overlay,baseline={(0,2pt)}]\draw[red,stealth-] (0,0) |- (2,-0.5)node[right]{Attribute value};te value">\\
\tikz[overlay,baseline={(0,-3pt)}]\draw[red,stealth-,shorten <=-2em](0,0) -- (-2,0)node[left]{Environment value};%
\null\hspace{2em} environment value\\
\tikz[overlay,baseline={(0,-3pt)}]\draw[red,stealth-](0,0) -- (-2,0)node[left]{Closing tag};</anEnvironment>
\end{minipage}
\end{center}

Let's consider a reaction closely:
\begin{xml}
    <!-- reaction 0005    -->
    <reaction reversible="yes" type="Elementary" id="0005">
      <equation>NO + O [=] O2 + N</equation>
      <rateCoeff>
        <Arrhenius>
           <A units="m3/kmol/s">8.4e+09</A>
           <E units="cal/mol">38526.0</E>
        </Arrhenius>
      </rateCoeff>
      <reactants>NO:1 O:1</reactants>
      <products>O2:1 N:1</products>
    </reaction>
\end{xml}
%
\begin{table}
\centering
\begin{tabular}{cccc}\toprule
Parameter         & keyword(s)              & Default value  & Type\\\midrule
Chemical reaction & \verb!reaction!         &                & environment \\
Chemical process  & \verb!type!             & Elementary     & attribute   \\
Elementary        & \verb!Elementary!       &                & value       \\
Duplicate         & \verb!Duplicate!        &                & value       \\
Three-body        & \verb!ThreeBody! or
                    \verb!threeBody!        &                & value       \\
Lindemann falloff & \verb!LindemannFalloff! &                & value       \\
Troe Falloff      & \verb!TroeFalloff!      &                & value       \\[5pt]
id                & \verb!id!               &                & attribute   \\
Reversible?       & \verb!reversible!       & yes            & attribute   \\[5pt]
Chemical equation & \verb!equation!         &                & environment \\
Reactants         & \verb!reactants!        &                & environment \\
Products          & \verb!products!         &                & environment \\[5pt]
Rate constant     & \verb!rateCoeff!        &                & environment \\
Rate constant's name
                  & \verb!name!             &                & attribute   \\
\kinModZ          & \verb!k0!               &                & value       \\
Efficiencies      & \verb!efficiency!       & 1.0            & environment \\[5pt]
Hercourt-Essen    & \verb!HercourtEssen!    &                & environment \\
Arrhenius         & \verb!Arrhenius!        &                & environment \\
Berthelot         & \verb!Berthelot!        &                & environment \\
Berthelot Hercourt-Essen
                  & \verb!BerthelotHercourtEssen!  &         & environment \\
Kooij             & \verb!Kooij!            &                & environment \\
Van't Hoff        & \verb!VantHoff!         &                & environment \\
Photochemistry    & \verb!photochemistry!   &                & environment \\[5pt]
\Tref             & \verb!Tref!             & 1~\unit{K}     & environment \\
\PreExp           & \verb!A!                &                & environment \\
\Power            & \verb!b!                &                & environment \\
\AcEn             & \verb!E!                &                & environment \\
\BerthExp         & \verb!D!                &                & environment \\
\wavelength       & \verb!lambda!           &                & environment \\
\crosssection     & \verb!cross_section!    &                & environment \\[5pt]
Unit              & \verb!units!            & see Tab. \ref{unit:default} 
                                                             & attribute \\
\bottomrule
\end{tabular}
\caption{\label{antioch:keyword_reading}Keyword for the \texttt{xml} input file.}
\end{table}

\begin{table}
\centering
\begin{tabular}{lcccccc}\toprule
                   & \verb!A!                                        & \verb!b! & \verb!E!             & \verb!D!       & \verb!lambda! & \verb!cross_section! \\\midrule
Elementary         & \unit{(m^3\,mol^{-1})^{\orderReac - 1}\,s^{-1}} & \nounit  & \unit{cal\,mol^{-1}} & \unit{K^{-1}}  &  \unit{nm}    &  \unit{cm^2\,nm^{-1}} \\
Duplicate          & \unit{(m^3\,mol^{-1})^{\orderReac - 1}\,s^{-1}} & \nounit  & \unit{cal\,mol^{-1}} & \unit{K^{-1}}  &   --          & -- \\
ThreeBody          & \unit{(m^3\,mol^{-1})^{\orderReac}\,s^{-1}}     & \nounit  & \unit{cal\,mol^{-1}} & \unit{K^{-1}}  &   --          & -- \\
Falloff (\kinModZ) & \unit{(m^3\,mol^{-1})^{\orderReac}\,s^{-1}}     & \nounit  & \unit{cal\,mol^{-1}} & \unit{K^{-1}}  &   --          & -- \\
Falloff (\kinModI) & \unit{(m^3\,mol^{-1})^{\orderReac - 1}\,s^{-1}} & \nounit  & \unit{cal\,mol^{-1}} & \unit{K^{-1}}  &   --          & -- \\
\bottomrule
\end{tabular}
\caption{\label{unit:default}Default units given to the parameters by \Antioch\ 
if not provided in the input file.}
\end{table}

Let's build it together, asking and answering every question that
arises along the way.
So we start from an \verb!xml! empty reaction database:
%
\begin{xml}
<?xml version="1.0"?>
<ctml>
  <reactionData>
  </reactionData>
</ctml>
\end{xml}
%
We want to add a reaction, with a commentary for the
reader:
\begin{xml}
<?xml version="1.0"?>
<ctml>
  <reactionData>
    <!-- my very first xml reaction, how cute! -->
    <reaction>
    </reaction>
  </reactionData>
</ctml>
\end{xml}
This reaction, as can be seen in chapter~\ref{Antioch:physics},
is characterized by its chemical process, its kinetics model and
if it is reversible or not. So, as attribute, we will provide
the chemical process and the reversibility of the reaction. An id
can be provided too, \Antioch\ will use it for warning or error
message purpose at the reading step.
\begin{xml}
<?xml version="1.0"?>
<ctml>
  <reactionData>
    <!-- my very first xml reaction, how cute! -->
    <reaction reversible="yes" type="Elementary" id="Reaction1">
    </reaction>
  </reactionData>
</ctml>
\end{xml}
Next thing we want, is the equation, we simply put it in the
appropriate environment:
\begin{xml}
<?xml version="1.0"?>
<ctml>
  <reactionData>
    <!-- my very first xml reaction, how cute! -->
    <reaction reversible="yes" type="Elementary" id="Reaction1">
      <equation>NO + O [=] O2 + N </equation>
    </reaction>
  </reactionData>
</ctml>
\end{xml}
Along with the equation, comes the description of the reaction
in terms of reactants and products. There we need to define
each reactant (resp. product) with its stoichiometric
coefficient:
\begin{xml}
<?xml version="1.0"?>
<ctml>
  <reactionData>
    <!-- my very first xml reaction, how cute! -->
    <reaction reversible="yes" type="Elementary" id="Reaction1">
      <equation>NO + O [=] O2 + N </equation>
      <reactants>NO:1 O:1 </reactants>
      <products>O2:1 N:1 </products>
    </reaction>
  </reactionData>
</ctml>
\end{xml}
This step is needed because \Antioch\ does not parse the
chemical equation (so basically you can get creative there),
\Antioch\ parses these environments to internally define the
reaction.

Next step is to define the rate constant. All these information
are going into a \verb!rateCoeff! environment. Several cases arise
there. In our case, an elementary reaction, we but need to define
the kinetics models and its parameters, each with its unit. But
if we had a three-body reaction, we then should define the efficiencies,
if we had a falloff, we need to define two rate constants.
The kinetics model is an environment, each parameter is an environment
too, having a unit attribute (except \Power).
Thus the general layout is:
\begin{xml}
      <rateCoeff>
        <kinModel name="k0">
          <Parameter units="appropriate unit"> value of parameter</A>
        </kinModel>
        <efficiencies> molecule:efficiency ... </efficiencies>
      </rateCoeff>
\end{xml}
with the \verb!name! attribute having a sense only in a falloff
reaction, and the \verb!efficiencies! for three-body reactions.
The efficiencies are formatted as shown, the molecule, a colon, the
stoichiometric coefficient. If a molecule has no efficiency
given, the default is 1.0.
Thus in our case, we have an Arrhenius kinetics model with the
following parameters:
\begin{itemize}
\item $\PreExp = 8.4\,10^9~\unit{m^3\,kmol^{-1}\,s^{-1}}$,
\item $\AcEn = \numprint{38526.0}~\unit{cal\,mol^{-1}}$.
\end{itemize}
Thus, our final reaction is therefore:
\begin{xml}
<?xml version="1.0"?>
<ctml>
  <reactionData>
    <!-- my very first xml reaction, how cute! -->
    <reaction reversible="yes" type="Elementary" id="Reaction1">
      <equation>NO + O [=] O2 + N </equation>
      <reactants>NO:1 O:1 </reactants>
      <products>O2:1 N:1 </products>
      <rateCoeff>
        <Arrhenius>
           <A units="m3/kmol/s">8.4e+09</A>
           <E units="cal/mol">38526.0</E>
        </Arrhenius>
      </rateCoeff>
    </reaction>
  </reactionData>
</ctml>
\end{xml}
Note that the order of environments inside environment have little
interest, at the slight exception of falloff. If ever you are
as bold as to provide a falloff without precision as to what rate
constant is the low pressure limit, the so-called \kinModZ, then
\Antioch\ will consider that the first one encountered is the
low-pressure limit.

Congratulations, you have a nice input file.
