\section{The particle flux}

A particle flux is the quantity of particles through
a surface per second (\unit{particle\,cm^{-2}\,s^{-1}}). This
value depends on some characteristics of the considered
particles, usually their energy. For instance, a photochemical
reaction will be expressed with respect to a certain photons
flux, \textit{i.e.} a photon spectrum. Such a spectrum is
expressed in terms of particles flux per energy. Thus the
``classic'' units you'll find for this will be
  \unit{photon\,cm^{-2}\,s^{-1}} or \unit{W\,m^{-2}} 
for the photon flux, and
  \unit{eV} or \unit{nm}
for the energy. 

Now typically, you have instrument resolution here, and you
measure those thing per windows of resolution, per bin. Thus,
your measured flux is per unit of energy (or equivalent). Thus,
informations on these quantities are typically of the form
(\unit{photon\,cm^{-2}\,s^{-1}\,nm^{-1}}~,~\unit{nm}).

The rate constant is calculated by integration of the
cross-section with the incoming particle flux 
(Tab.~\ref{antioch::kinMod}), thus a rescaling
is due. \Antioch\ rescale cross-sections to given
flux.

The principle is simple, let's put some units first. We
consider a photon flux measured in \unit{photon\,cm^{-2}\,s^{-1}\,nm}
on a grid of \unit{nm}. We have a cross-section on \unit{cm^2\,\AA}
on a grid of \unit{\AA}. The principle will be to rescale
the cross-section to obtain the wanted units, then to calculate
the integrated value on each photon flux bin, and then deduced
the corresponding value of the cross-section on each photon
flux bin. Fig.~\ref{dev:particle_flux_rescaling} pictures
this protocol.

\begin{figure}
\centering
\includegraphics{particle_flux}
\caption{\label{dev:particle_flux_rescaling}The rescaling of the cross section (blue) on
a new grid (black) requires to calculate the surface (dotted/stripped areas) to rescale
on this new grid (red).}
\end{figure}
