\Antioch\ has three viscosity models available:
\begin{itemize}
\item Blottner viscosity model;
\item Sutherland viscosity model;
\item viscosity as derived from kinetics theory under the ideal gas assumption.
\end{itemize}

\subsubsection{Blottner}
\label{transport:viscosity:blottner}

This viscosity model is a fit of the form:
\begin{equation}
\vis = \numprint{0.1} \exp\left((a \ln(\Temp) + b) \ln(\Temp) + c\right)
\label{blottner:eq}
\end{equation}
$a$, $b$, $c$ being the Blottner parameter for the considered
species.

\subsubsection{Sutherland}
\label{transport:viscosity:sutherland}

This viscosity model is of the form:
\begin{equation}
\vis = \vis_\text{ref} \frac{\Temp^{\numprint{1.5}}}{\Temp + \text{T}_\text{ref}}
\label{sutherland:eq}
\end{equation}

\subsubsection{Kinetics Theory}
\label{transport:viscosity:kin_theo}

The kinetics theory uses the assumption of ideality in a 
gaseous medium. This assumption means basically that the
state law is
\begin{equation}
\Press = \conc\Rg \Temp
\end{equation}
with \Press\ the pressure (\unit{Pa}), \conc\ the molar concentration
(\unit{mol\,m^{-3}}), \Rg\ the universal gas constant (\RgEquation)
and \Temp\ the temperature (\unit{K}).

We use the following formul\ae\ for the species $s$:
\begin{equation}
\vis[s] = \frac{5}{16} \frac{\sqrt{\pi\Mass[s]\Boltzmann\Temp}}{\Press\LJdia[s]^2\Stockmayer{2}}
\label{visc_kin_theory:eq}
\end{equation}
with \Mass\ the molecular mass of the species (\unit{kg}), \LJdia\ the Lennard-Jones
potential diameter associated to the species (\unit{m}), \Boltzmann\ the Boltzmann
constant (\BoltzmannEquation), \Press\ the pressure (\unit{Pa}), \Stockmayer{2}\ is the
value of the integrated integral as discussed in \citet{Monchick1961}

