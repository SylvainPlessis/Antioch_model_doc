\section{Different parsers}

\Antioch's supports three different parsing format,
\ChemKin, \XML\ and \ascii\ free format.

\subsection{\XML\ inputs}

The \XML\ format consists of tags and labels. It is an interesting
format in the sense that it is explicit, therefore nothing is
left to the imagination, the defaults are defined (and should be
regarded as) a backup for user's laziness.

\subsection{\ChemKin\ inputs}

\Antioch's follow the rules as given in \citet{ChemKinReacDesign2000}.
This format is fully constrained, only the units are, within
certain limits, customizable.

\subsection{\ascii\ free format inputs}

The \ascii\ free format is a column-based format. The parser can
ignore arbitrary columns, designed by their index, but the order
is imposed.

The first column is \emph{always} the name of the molecule. It is
not counted when specifying the ignored columns index. The index
are \prog{C} indexes, they start at zero. Thus the second column,
being the first characteristic read, has index zero.

\section{Different quantities to parse}

\subsection{Defining the species, mandatory description.}

A \object{ChemicalSpecies} is defined by:\\
\begin{minipage}{0.48\linewidth}
\begin{itemize}
\item a name;
\item a molar mass (\unit{kg\,mol^{-1}});
\item a formation enthalpie at 0$^\circ$\unit{C};
\item a translational/rotational number of degrees of freedom;
\item a charge number.
\end{itemize}
\end{minipage}
\hfill
\begin{minipage}{0.48\linewidth}
\centering
\includegraphics[width=0.8\linewidth]{parser_chem_spec}
\end{minipage}
\smallskip

This is mandatory, and currently only an \ascii\ file
will be used for this. Thus the file requires at least
five columns in the order:
\begin{enumerate}
\item molecular weight in \unit{g\,mol^{-1}};
\item heat of formation in \unit{J\,kg};
\item the number of translational/rotational degrees of freedom, \nounit;
\item the charge, \nounit.
\end{enumerate}

\subsection{The kinetics description}

\subsection{The transport description}

\subsection{The thermodynamics description}
