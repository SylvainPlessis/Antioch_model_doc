The units in \Antioch\ are managed at the reading
steps. The SI system is the internal system, thus
for kinetics parameters input file.


The parameters we need to be aware of:
\begin{itemize}
\item in the file \file{species\_ascii\_parsing.h}
        \begin{itemize}
        \item molecular weight: mass per quantity of matter (SI is \unit{kg\,mol^{-1}}),
        \item heat of formation at 0~\unit{K}: energy per mass (SI is \unit{J\,kg^{-1}})
        \end{itemize}
\item in the file \file{physical\_constants.h}
        \begin{itemize}
        \item \Rg: energy per quantity of matter per unit of temperature (SI is \unit{J\,mol^{-1}\,K^{-1}}),
        \item Avogadro number: one over quantity of matter (SI is \unit{mol^{-1}})
        \end{itemize}
\end{itemize}

Exemple with \Rg. If you go to the
\href{http://physics.nist.gov/cgi-bin/cuu/Value?r}{NIST definition}
of \Rg, you obtain the advised value of \Rg:
$\Rg = \numprint{8.3144621} \pm \numprint{0.0000075}~\unit{J\,mol^{-1}\,K^{-1}}$.
Thanks to \href{https://en.wikipedia.org/wiki/Gas_constant}{wikipedia}
(right-side table), we have also the value in \unit{calorie}:
$\Rg = \numprint{1.9858775} \pm \numprint{0.0000034}~\unit{cal\,mol^{-1}\,K^{-1}}$.
We look now at the definition of the \unit{calorie} unit, again,
we go to the \href{http://physics.nist.gov/Pubs/SP811/appenB9.html#ENERGY}{NIST website}.
We have then several definitions:
\begin{enumerate}
\item International Table, \unit{calorie_\text{IT}},   defined as \numprint{4.1868}  \unit{Joule};
\item thermodynamic, \unit{calorie_\text{th}},   defined as \numprint{4.184}   \unit{Joule};
\item mean (?), \unit{calorie_\text{mean}}, defined as \numprint{4.19002} \unit{Joule}.
\end{enumerate}
Comparing everyone leads to table~\ref{Rwtf}.
\begin{table}
\centering
\begin{tabular}{lcc}\toprule
\null\hfill Unit \hfill\null                  & \Rg\ value                          & factor to SI \\\midrule
\unit{J\,mol^{-1}\,K^{-1}}                    & \numprint{8.3144621}(75)            & \numprint{1.00000} \\
\unit{cal_\text{IT}\,mol^{-1}\,K^{-1}}        & \color{red}\numprint{1.9858752}(18) & \numprint{4.18680} \\
\unit{cal_\text{th}\,mol^{-1}\,K^{-1}}        & \color{red}\numprint{1.9872041}(18) & \numprint{4.18400} \\
\unit{cal_\text{mean}\,mol^{-1}\,K^{-1}}      & \color{red}\numprint{1.9843490}(18) & \numprint{4.19002} \\
\unit{cal_\text{wikipedia}\,mol^{-1}\,K^{-1}} & \numprint{1.9858775}(34)            & \color{red}\numprint{4.18680}\\
\bottomrule
\end{tabular}
\caption{\label{Rwtf}Discrepancies in \Rg. Red are calculated values from data.}
\end{table}
It seems obvious then that wikipedia gives the International Table (IT) calorie, with a
relative difference in the factor with the given one of $\numprint{1.1806}\,10^{-6}$. 
This relative error is bigger than machine tolerance (from the \textcolor{green!60!black}{\bf double}
to more precise).

We use thus the advised value in \unit{J\,mol^{-1}\,K^{-1}} and the factor
\numprint{4.1868} given by the NIST to make the conversion in \unit{calorie}.
Be aware than in this case, you have a relative discrepancy of $\numprint{1.2}\,10^{-6}$
with the wikipedia value.
