\Antioch\ takes care of the units for every
parameter thanks to the \Units\ object. This object
enables powerful unit evaluation and conversion,
of a physical quantity as well as an equation.

A description of this object and its dependencies is
given in the \Doxygen\ documentation, thus only a
brief description and some examples will be given here.

The \Units\ objects projects an unit on a unit basis,
defined by the SI system plus radian for angles. There are
three different parts:
\begin{itemize}
\item the symbol of the unit,
\item the power array (basis projection),
\item the coefficient to convert the unit to the basis.
\end{itemize}

All is defined so as to provide an easy use should the user
want to play (or test/analyze) the dimension of some physical
quantity.

The basis is defined as: \unitbase.

The rules for a unit are simple:
\begin{itemize}
\item ``.'' and ``/'' are the separators 
        (thus \verb!m.s-1! and \verb!m/s! are equivalent),
\item it supports grouping with parenthesises.
\end{itemize}

\subsection{Simple example}

Let's consider an energy for instance, say parameter
$E$ is given in \unit{cal}. You would initialize a \Units\
object by 
\begin{cpp}
Units<Scalar> E_unit("cal")
\end{cpp}
\prog{Scalar} being the precision you want, typically \prog{float},
\prog{double} or \prog{long double}. Then you can obtain factors
to any unit that is homogeneous to an energy:
\begin{cpp}
Scalar j = E_unit.factor_to_some_unit("J"); // j = 4.1868
Scalar strange = E_unit.factor_to_some_unit("g.m2/s2"); // strange = 4.1868e3
Scalar weird = E_unit.factor_to_some_unit("Da.ang2.min-2"); // weird = 9.07686e50
Scalar error = E_unit.factor_to_some_unit("Da.ang.min-2"); 
/*
terminate called after throwing an instance of 'Antioch::UnitError'
  what():  Units are not homogeneous:
"cal" and "Da.ang.min-2".
*/ 
\end{cpp}

\subsection{Combining units}

The \Units\ object enables to combine unit
as they would be combined by an equation.
\begin{itemize}
\item A multiplication is a unit addition, 
        $A \cdot B \Rightarrow \unit{U_{A\cdot B}} = \unit{U_A} + \unit{U_B}$;
\item a division       is a unit substraction, 
        $\frac{A}{B} \Rightarrow \unit{U_{\frac{A}{B}}} = \unit{U_A} - \unit{U_B}$;
\item an addition      is a unit homogeneity test, 
        $A + B \Rightarrow \unit{U_A} \text{ is homogeneous to } \unit{U_B}$;
\item a substraction   is a unit homogeneity test, 
        $A - B \Rightarrow \unit{U_A} \text{ is homogeneous to } \unit{U_B}$.
\end{itemize}
Mathematical functions have specifics requirements:
\begin{itemize}
\item $\log$, $\exp$, $\mathrm{pow}$, $\arccos$, $\arcsin$, $\arctan$ require unitless parameters;
\item $\cos$, $\sin$, $\tan$ require angled-united parameters;
\item \mbox{\vdots}
\end{itemize}

The \Units\ object supports all these operations.
It is therefore easy to define an object to automatically get the unit
from an equation. One such object is given in appendix~\ref{unit_goodie},
some capabilities are given here.


\begin{minipage}{0.55\linewidth}
\begin{cpp|} 
// 1 gram 
// light celerity
// Energy contained in 1 gram
Antioch::PhysicalQuantity<double> 
        m(1.,"g"),
        c(299792458.,"m.s-1"),
        E = m * c * c; 

std::cout << E.value() << " " 
          << E.unit() << std::endl;

E.change_unit_to("J");
std::cout << E.value() << " " 
          << E.unit() << std::endl;

E.change_unit_to("cal");
std::cout << E.value() << " " 
          << E.unit() << std::endl;

E.change_unit_to_SI();
std::cout << E.value() << " " 
          << E.unit() << std::endl;
\end{cpp|}
\end{minipage}
\terminal{%
8.987552e+16 g.m.s-1.m.s-1\\
8.987552e+13 J\\
2.146640e+13 cal \\
8.987552e+13 kg.m2.s-2%
}

\subsection{Restrictions}
No partial power! The physical senses of it is questionnable
anyway, \Antioch\ will not let you define in any way
a partial power.
