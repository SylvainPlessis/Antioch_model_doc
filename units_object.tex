\Antioch\ takes care of the units for every
parameter thanks to the \Units\ object. This object
enables powerful unit evaluation and conversion,
of a physical quantity as well as an equation.

A description of this object and its dependencies is
given in the \Doxygen\ documentation, thus only a
brief description and some examples will be given here.

The \Units\ objects projects an unit on a unit basis,
defined by the SI system plus radian for angles. There are
three different parts:
\begin{itemize}
\item the symbol of the unit,
\item the power array (basis projection),
\item the coefficient to convert the unit to the basis.
\end{itemize}

All is defined so as to provide an easy use should the user
want to play (or test/analyze) the dimension of some physical
quantity.

The basis is defined as: \unitbase.

The rules for a unit are simple:
\begin{itemize}
\item ``.'' and ``/'' are the separators 
        (thus \verb!m.s-1! and \verb!m/s! are equivalent),
\item it supports grouping with parenthesises.
\end{itemize}

\subsection{Simple example}

Let's consider an energy for instance, say parameter
$E$ is given in \unit{cal}. You would initialize a \Units\
object by 
\begin{verbatim}
Units<Scalar> E_unit("cal")
\end{verbatim}
\prog{Scalar} being the precision you want, typically \prog{float},
\prog{double} or \prog{long double}. Then you can obtain factors
to any unit that is homogeneous to an energy:
\begin{verbatim}
Scalar j = E_unit.factor_to_some_unit("J"); // j = 4.1868
Scalar strange = E_unit.factor_to_some_unit("g.m2/s2"); // strange = 4.1868e3
Scalar weird = E_unit.factor_to_some_unit("Da.ang2.min-2"); // weird = 9.07686e50
Scalar error = E_unit.factor_to_some_unit("Da.ang.min-2"); 
/*
terminate called after throwing an instance of 'Antioch::UnitError'
  what():  Units are not homogeneous:
"cal" and "Da.ang.min-2".
*/ 
\end{verbatim}

\subsection{Combining units}

The \Units\ object enables to combine unit
as they would be combined by an equation.


\subsection{Restrictions}
No partial power! The physical senses of it is questionnable
anyway, \Antioch\ will not let you define in any way
a partial power.
\subsection{Goodies}
An object to automatically get the unit
from an equation:
{\footnotesize
\begin{verbatim}

namespace Antioch{

  template <typename CoeffType>
  class PhysicalQuantity{

    public:
        PhysicalQuantity();
        PhysicalQuantity(const CoeffType & value, const std::string & unit = std::string() );

        ~PhysicalQuantity();

        /* getter/setter */

        void set_value(const CoeffType & value);

        void set_unit(const std::string & unit);

        const CoeffType value() const;

        const Units<CoeffType> & unit() const;

        /* operators */

        PhysicalQuantity<CoeffType> & operator+=(const PhysicalQuantity<CoeffType> &rhs);

        PhysicalQuantity<CoeffType> & operator-=(const PhysicalQuantity<CoeffType> &rhs);

        PhysicalQuantity<CoeffType> & operator*=(const PhysicalQuantity<CoeffType> &rhs);

        PhysicalQuantity<CoeffType> & operator/=(const PhysicalQuantity<CoeffType> &rhs);

        PhysicalQuantity<CoeffType>   operator* (const PhysicalQuantity<CoeffType> &rhs) const;

        PhysicalQuantity<CoeffType>   operator/ (const PhysicalQuantity<CoeffType> &rhs) const;

        PhysicalQuantity<CoeffType>   operator+ (const PhysicalQuantity<CoeffType> &rhs) const;

        PhysicalQuantity<CoeffType>   operator- (const PhysicalQuantity<CoeffType> &rhs) const;

    private:
        Units<CoeffType> _unit;
        CoeffType        _value;

  }

  template <typename CoeffType>
  PhysicalQuantity<CoeffType>::PhysicalQuantity():
        _unit(),
        _value(0.)
  {
     return;
  }

  template <typename CoeffType>
  PhysicalQuantity<CoeffType>::PhysicalQuantity(const std::string & unit, const CoeffType & value):
        _unit(unit),
        _value(value)
  {
     return;
  }

  template <typename CoeffType>
  void PhysicalQuantity<CoeffType>::set_value(const CoeffType & value)
  {
     _value = value;
  }

  template <typename CoeffType>
  void PhysicalQuantity<CoeffType>::set_unit(const std::string & unit)
  {
     _unit = unit;
  }

  template <typename CoeffType>
  const CoeffType PhysicalQuantity<CoeffType>::value() const
  {
     return _value;
  }

  template <typename CoeffType>
  const Units<CoeffType> & PhysicalQuantity<CoeffType>::unit() const
  {
     return _unit;
  }

  template <typename CoeffType>
  PhysicalQuantity<CoeffType> & 
        PhysicalQuantity<CoeffType>::operator+=(const PhysicalQuantity<CoeffType> &rhs)
  {
     if(!rhs.unit().is_homogeneous(this->unit()))antioch_error();
     this->set_value(this->value() + rhs.value());
     return *this;
  }

  template <typename CoeffType>
  PhysicalQuantity<CoeffType> & 
        PhysicalQuantity<CoeffType>::operator-=(const PhysicalQuantity<CoeffType> &rhs)
  {
     if(!rhs.unit().is_homogeneous(this->unit()))antioch_error();
     this->set_value(this->value() - rhs.value());
     return *this;
  }

  template <typename CoeffType>
  PhysicalQuantity<CoeffType> & 
        PhysicalQuantity<CoeffType>::operator*=(const PhysicalQuantity<CoeffType> &rhs)
  {
     this->set_value(this->value() * rhs.value());
     this->set_unit(this->unit() + rhs.unit());
     return *this;
  }

  template <typename CoeffType>
  PhysicalQuantity<CoeffType> & 
        PhysicalQuantity<CoeffType>::operator/=(const PhysicalQuantity<CoeffType> &rhs)
  {
     this->set_value(this->value() / rhs.value());
     this->set_unit(this->unit() - rhs.unit());
     return *this;
  }

  template <typename CoeffType>
  PhysicalQuantity<CoeffType> & 
        PhysicalQuantity<CoeffType>::operator*(const PhysicalQuantity<CoeffType> &rhs)
  {
     return (PhysicalQuantity<CoeffType>(*this) *= rhs);
  }

  template <typename CoeffType>
  PhysicalQuantity<CoeffType> & 
        PhysicalQuantity<CoeffType>::operator/(const PhysicalQuantity<CoeffType> &rhs)
  {
     return (PhysicalQuantity<CoeffType>(*this) /= rhs);
  }

  template <typename CoeffType>
  PhysicalQuantity<CoeffType> & 
        PhysicalQuantity<CoeffType>::operator+(const PhysicalQuantity<CoeffType> &rhs)
  {
     return (PhysicalQuantity<CoeffType>(*this) += rhs);
  }
  
  template <typename CoeffType>
  PhysicalQuantity<CoeffType> & 
        PhysicalQuantity<CoeffType>::operator-(const PhysicalQuantity<CoeffType> &rhs)
  {
     return (PhysicalQuantity<CoeffType>(*this) -= rhs);
  }

}
\end{verbatim}

}
