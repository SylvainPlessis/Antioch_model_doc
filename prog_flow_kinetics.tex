Figure~\ref{kinpf} describe the data flow from the upper
object \KineticsEvaluator\ to the rate constants objects.
The idea here is to have a light high-level object that will
be fast to build, so easily threadable. All the necessary data
and the bulk of the work is in lower-level objects.
%
\begin{figure}
\centering
\includegraphics{kinetics_relationships}
\caption{\label{kinpf}Global vision of the kinetics in \Antioch.
The gray area frames objects that should not be threaded, whereas
\KineticsEvaluator\ is light and thread-safe. The passed values of
interest are the temperature \prog{const StateType \& T}, the
particle flux \prog{const ParticleFlux<VectorCoeffType> \& pf} and
the composition \prog{const VectorStateType \& dens}.}
\end{figure}
%
\subsection{A classical case}

Let's consider a simple bimolecular reaction, elementary,
with a Kooij kinetics model. The informations we need for
a complete kinetics calculation are the temperature, the
composition and the thermodynamics. The thermodynamics are
calculated in another place of the program, what \Antioch\
will provide will be the Gibbs energy for all the concerned
species (see section~\ref{kinetics_gen}, \textbf{Going backward}
subsection). So at the \KineticsEvaluator\ level, we need
to provide \Antioch's guts with:
\begin{itemize}
\item a \prog{const StateType \& T} for the temperature,
\item a \prog{const VectorStateType \& molar\_densities} for
                the densities,
\item a \prog{const VectorStateType \& h\_RT\_minus\_s\_R} for
                the thermodynamics.
\end{itemize}

The goal is to have the quantity \doverdt{S} for all species \ce{S} concerned by the 
reactive system (see \ref{phys:kinetics_theory_integrated}).

\subsubsection{Data flow}

\paragraph{\KineticsEvaluator}
will gain from \ReactionSet\ the net rates of
the reactions,  i.e. the term \rate[r] in \ref{phys:kinetics_theory_integrated}.
Thus \KineticsEvaluator\ will, for every species, perform the loop
over the concerned reactions and multiply by the stoichiometric coefficient.
Technically, \KineticsEvaluator\ gives \ReactionSet\ a \prog{VectorCoeffType}
to fill with these terms.

\paragraph{\ReactionSet}
will equate the values of the provided \prog{VectorCoeffType} with
the returning value given by the \Reaction. To do that, it passes down to
the \Reaction\ object the temperature, the composition, the thermodynamics and
the value $\frac{\pz}{\Rg \Temp}$, used for the equilibrium constant calculation,
needed for the backward rate constant \eqref{therm:K_therm}.

\paragraph{\Reaction} will asks the \KineticsType\ object (its rate constants) to calculate
the forward rate constant of the reaction (term \fwdratecons[r]) by passing
it the temperature. Then it will be able 
to calculate the backward rate constant (term \bkwdratecons[r]), 
the forward (term \fwdrate[r]) and backward rate (term \bkwdrate[r]), 
and finally the net rate (term \rate[r]).

\paragraph{\KineticsType} will return the forward rate constant \fwdratecons[r] given
the temperature.

\subsubsection{Template flow}

\paragraph{\KineticsEvaluator} asks for a \prog{VectorCoeffType} to be filled. It
provides a \prog{StateType} for the temperature and a \prog{VectorStateType} for
the densities.

\paragraph{\ReactionSet} fills each value of the \prog{VectorCoeffType} with
the returning type of \Reaction.

\paragraph{\Reaction} returns a \prog{StateType}, which is the defined type of the
provided temperature.

\paragraph{\KineticsType} returns a \prog{StateType}, defined by the type of the
provided temperature.
