\documentclass[twoside]{report}
\usepackage{booktabs}
\usepackage{geometry}
\usepackage{amsmath}
\usepackage{multicol}
\usepackage{multirow}
\usepackage{array}
\usepackage{graphicx}
\usepackage{tikz}
\usepackage{listings}
\usepackage{longtable}
\usepackage[autolanguage]{numprint}
\usepackage[version=3,arrows=pgf-filled]{mhchem}
\usepackage{fancyhdr}
\usepackage[round]{natbib}
\usepackage[linkcolor=black,colorlinks=true,citecolor=black,urlcolor=blue]{hyperref}
\makeatletter
%first thing first
\graphicspath{{./figs/}}
%%%% meta-meta-function
\newcommand{\make@}[3][upshape]{%
\expandafter\gdef\csname font@#2\endcsname##1{% create font@something macro
{\csname#1\endcsname\csname#3\endcsname{##1}}%
}%
\expandafter\gdef\csname make@#2\endcsname##1{% create make@something meta macro
\expandafter\gdef\csname ##1\endcsname{\csname font@#2\endcsname{##1}}%
}
\expandafter\gdef\csname#2\endcsname##1{\csname font@#2\endcsname{##1}}%create the \something macro
}
%% meta font
\newcommand{\prog@type}[1]{\textcolor{green!60!black}{\object{#1}}}
%% meta function
\make@{program}{sf}
\make@{library}{tt}
\make@[bf]{object}{tt}
\make@{file}{sf}
\make@{prog}{prog@type}
%% macros
\make@program{Doxygen}
\make@library{Antioch}
\make@library{antioch}
\make@library{Boost}
\make@library{VexCL}
\make@library{ViennaCL}
\make@library{MetaPhysicL}
\make@library{GRVY}
\make@library{EIGEN}
%% for tutorial
\make@object{ReactionSet}
\make@object{KineticsEvaluator}
\make@object{ChemicalMixture}
\make@object{CEAThermodynamics}
\make@object{Units}
\make@prog{double}
\newcommand{\stdvector}{\prog{std::vector< >}}
\newcommand{\stdstring}{\prog{std::string}}
%%physics
\newcommand{\phenomenom}{\textcolor{red}{phenomenom}}
\newcommand{\quantity}{\textcolor{blue!60!black}{quantity}}
\newcommand{\Quantity}{\textcolor{blue!60!black}{Quantity}}
\newcommand{\model}{\textcolor{green!60!black}{model}}
%

%%% versions
\newcommand{\version}[3]{%
\setcounter{vmajor}{#1}
\setcounter{vmedium}{#2}
\setcounter{vminor}{#3}
}
\newcounter{vmajor}
\newcounter{vmedium}
\newcounter{vminor}
\newcommand{\theversion}{\thevmajor.\thevmedium.\thevminor}

%%% constant equation management
\newcommand{\make@meta@equation}[7]
{
\expandafter\gdef\csname#2\endcsname{\ensuremath{#3}}%                                                          \"name"     produces "symbol"
\expandafter\gdef\csname#2val\endcsname{\csname#7\endcsname{#4}\ifnum#1=0\else\,10^{#1}\fi}%                              \"name"val  produces "value"
\expandafter\gdef\csname#2dval\endcsname{\csname#7\endcsname{#5}}%                                                        \"name"dval produces "dvalue"
\expandafter\gdef\csname#2num\endcsname{\csname#7\endcsname{#4}\ifnum#5=0\else\,(\csname#2dval\endcsname)\fi\ifnum#1=0\else\,10^{#1}\fi}%   \"name"num  produces "value (dvalue)"
\expandafter\gdef\csname#2unit\endcsname{\unit{#6}}%                                                            \"name"unit produces "unit"
\expandafter\gdef\csname#2Equation\endcsname{% %% equation
 \ensuremath{\csname#2\endcsname = \csname#2num\endcsname~\csname#2unit\endcsname}}%                            \"name"Equation produces "symbol = value (dvalue)~unit"
}
\newcommand{\make@equation}[6][0] %% name, symbol, value, dvalue, unit | symbol = value (dvalue) \unit{unit}
{%
 \make@meta@equation{#1}{#2}{#3}{#4}{#5}{#6}{numprint}
}
\newcommand{\make@litteralEquation}[4] %% name, symbol, expression, unit | symbol = expression \unit{unit}
{%
 \make@meta@equation{0}{#1}{#2}{#3}{0}{#4}{relax}
}

%% fun
\newcommand{\ANTIOCH}{\font@library{A}
                      \font@library{N}ew
                      \font@library{T}emplated
                      \font@library{I}mplementation
                      \font@library{O}f
                      \font@library{C}hemistry
                      \font@library{H}ydrodynamics}
\newcommand{\ANTIOCHPhys}{\Antioch's
                         \font@library{N}oble
                         \font@library{T}ry to
                         \font@library{I}mitate the
                         \font@library{O}rdered
                         \font@library{C}osmos
                         \font@library{H}umbly}
\newcommand{\ANTIOCHTech}{\Antioch's
                         \font@library{N}ot
                         \font@library{T}emplated
                         \font@library{I}rresponsibly, this
                         \font@library{O}utstanding
                         \font@library{C}ode is
                         \font@library{H}appiness}
\newcommand{\ANTIOCHPrac}{\Antioch\
                         \font@library{N}ewbies
                         \font@library{T}empting
                         \font@library{I}nvitation to
                         \font@library{O}pen this
                         \font@library{C}hapter with
                         \font@library{H}aste}
%% I like my equations this way
\renewcommand{\theequation}{Eq.~\thesection-\arabic{equation}}
\@addtoreset{equation}{section}
%%% beauty here
\pagestyle{fancy}
\renewcommand{\chaptermark}[1]{\markboth{#1}{}}
\renewcommand{\sectionmark}[1]{\markright{\thesection~\sl #1}}
\geometry{headheight=2cm}
\fancyhead{} %empty
\fancyhead[LO]{\parbox[b]{0.45\textwidth}{\raggedright\leftmark}}
\fancyhead[RE]{\parbox[b]{0.45\textwidth}{\raggedleft\leftmark}}
\fancyhead[LE]{\parbox[b]{0.45\textwidth}{\raggedright\rightmark}}
\fancyhead[RO]{\parbox[b]{0.45\textwidth}{\raggedleft\rightmark}}
%% a little hack
\newcommand{\spacehack}{\ensuremath{\textcolor{white}{=}}}

%%%%%%%%%%%%%%%%%%%%%%%%%%%%%%%%%%%%
%%%%
%%%%  stuffs start here
%%%%
%%%%%%%%%%%%%%%%%%%%%%%%%%%%%%%%%%%%

%layout stuff
\newlength{\tocenter}
\newcommand{\graphAtPaperCenter}[2][width=\paperwidth]
{
%%%% from geometry package
%% inner/outer margin ratio 2:3
%% => \parperwidth - \textwidth = 5
%% center = (inner +) 0.5\textwidth + (\parperwidth - \textwidth)/10 %% offset from inner (odd page)
%% center = (outer +) 0.5\textwidth - (\parperwidth - \textwidth)/10 %% offset from outer (even page)
%%%%%%%%%%%
\setlength{\tocenter}{\paperwidth}
\addtolength{\tocenter}{-\textwidth}
\ifodd\number\c@page% TeX register of LaTeX counter, late one page
  \setlength{\tocenter}{0.1\tocenter}
\else% even
  \setlength{\tocenter}{-0.1\tocenter}
\fi%
\addtolength{\tocenter}{0.5\textwidth}
\noindent\hspace{\tocenter}\makebox[0pt]{\includegraphics[#1]{#2}}\par
}

%% chapter
\@addtoreset{chapter}{part}

% unit defs
\newcommand{\unitbase}{[\unit{m},\unit{kg},\unit{s},\unit{A},\unit{K},\unit{mol},\unit{cd},\unit{rad}]}
%%%%%%%
%%
%% 'cause I'm lazy 
%%
%%%%
\newcommand{\example}[1]{\textit{i.e.} #1}


%internal stuff
\newcommand{\optional@sub} [2][\relax]{\ifx#1\relax\ensuremath{#2}\else\ensuremath{{{#2}_{#1}}}\fi}
\newcommand{\optional@@sub}[2][\relax]{\ifx#1\relax\ensuremath{#2}\else\ensuremath{{#2}_{#1}}\fi}
\newcommand{\d@d} [2]{\ensuremath{\frac{\partial#2}{\partial #1}}}
\newcommand{\dd@d}[2]{\ensuremath{\frac{\dd #2}{\dd #1}}}

%%%%%%%%%%%%%%%%%%%%%%%%%%%%%
%%
%% universal constant, using
%% meta macro make@equation and
%% meta macro make@litteralEquation
%%%%%%%%%%%%%%%%%%%%%%%%%%%%%%

%% valued constant
\make@equation     {Rg}       {\mathrm{R}}            {8.3144621} {75}{J\,mol^{-1}\,K^{-1}}
\make@equation[-23]{Boltzmann}{\mathrm{k}_\text{B}}   {1.3806488} {13}{J\,K^{-1}}
\make@equation[23] {Navo}     {\mathcal{N}_\text{Avo}}{6.02214129}{27}{mol^{-1}}
\make@equation[8]  {cele}     {\mathrm{c}}            {2.99792458}{0} {m\,s^{-1}}
%% computed from other constants constant
\make@litteralEquation{vacPerme}{\mu_0}     {4\pi\,10^{-7}}{m\,kg\,s^2\,A^2}
\make@litteralEquation{vacPermi}{\epsilon_0}{\frac{1}{\vacPerme\cele^2}}{A^2\,s^4\,kg^{-1}\,m^{-3}}

%%%%%%%%%%%%%%%%%%%%%%%%%%%%%%%%%%%%
%%
%% mathematical stuff
%%
%%%%%%%%%%%%%%%%%%%%%%%%%%%%%%%%%%%%
\newcommand{\doverdt}    [1]{\d@d{t}{#1}}
\newcommand{\doverdT}    [1]{\d@d{\Temp}{#1}}
\newcommand{\ddoverdT}   [1]{\dd@d{\Temp}{#1}}
\newcommand{\doverdm}    [2][\relax]{\d@d{\mass[#1]}{#2}}
\newcommand{\doverdc}    [2][\relax]{\d@d{\conc[#1]}{#2}}
\newcommand{\doverdn}    [2][\relax]{\d@d{\Mol[#1]}{#2}}
\newcommand{\doverdext}  [2][\relax]{\d@d{\ext[#1]}{#2}}
\newcommand{\ddoverddext}[3]{\ensuremath{\frac{\partial^2#1}{\partial\ext[#2]\partial\ext[#3]}}}
%%%%%%%%%%%%%%%%%%%%%%%%%%%%%%
%%
%%  chemical definitions
%%
%%%%%%%%%%%%%%%%%%%%%%%%%%%%%
\newcommand{\pz}            {\ensuremath{\mathrm{p^0}}}
\newcommand{\Tz}            {\ensuremath{\mathrm{T^0}}}
\newcommand{\eq}            {\ensuremath{\text{eq}}}
\newcommand{\Rmix}          {\ensuremath{\mathrm{R}_\text{mix}}}
\newcommand{\reac}          {\ce{R}}
\newcommand{\product}       {\ce{P}}
\newcommand{\tc}         [1]{\ensuremath{a_{#1}}}
%%stuff for kinetics
\newcommand{\rcons}   [1][\relax]{\optional@@sub[#1]{k}}
\newcommand{\kinMod}  [1][\relax]{\optional@@sub[#1]{\alpha}\ensuremath{(\Temp)}}
\newcommand{\kinModZ}            {\ensuremath{\alpha_0(\Temp)}}
\newcommand{\kinModI}            {\ensuremath{\alpha_\infty(\Temp)}}
\newcommand{\chemProc}[1][\relax]{\optional@@sub[#1]{\chi}\ensuremath{(\kinMod,\conc[M])}}
\newcommand{\rateCons}           {\ensuremath{\rcons(T,\conc[M])}}
\newcommand{\rate}    [1][\relax]{\optional@@sub[#1]{r}}
\newcommand{\threeBody}          {\ensuremath{\sum_s \epsilon_s\conc[S]}}
\newcommand{\FLind}              {\ensuremath{F_\text{Lind}}}
\newcommand{\FTroe}              {\ensuremath{F_\text{Troe}}}
%%%% kinetics model parameter
\newcommand{\Tref}        {\ensuremath{\mathrm{T_{ref}}}}
\newcommand{\PreExp}      {\ensuremath{A}}
\newcommand{\Power}       {\ensuremath{\beta}}
\newcommand{\BerthExp}    {\ensuremath{D}}
\newcommand{\AcEn}        {\ensuremath{{E_a}}}
\newcommand{\wavelength}  {\ensuremath{{\lambda}}}
\newcommand{\crosssection}{\ensuremath{{\sigma(\wavelength)}}}
%%%% Troe falloff
\newcommand{\Troealpha} {\ensuremath{\alpha}}
\newcommand{\TroeTone}  {\ensuremath{T^{*}}}
\newcommand{\TroeTtwo}  {\ensuremath{T^{**}}}
\newcommand{\TroeTthree}{\ensuremath{T^{***}}}
\newcommand{\TroeFcent} {\ensuremath{F_\text{cent}}}
\newcommand{\Troen}     {\ensuremath{n_\text{T}}}
\newcommand{\Troed}     {\ensuremath{d_\text{T}}}
\newcommand{\Troec}     {\ensuremath{c_\text{T}}}
%
\newcommand{\conc}        [1][\relax]{\ifx#1\relax\molar\else\ce{[#1]}\fi}
\newcommand{\Mm}          [1][\relax]{\optional@sub[#1]{\mathrm{M}}}
\newcommand{\scoefabs}    [1][\relax]{\optional@sub[#1]{\mathrm{n}}}
\newcommand{\scoef}       [1][\relax]{\optional@sub[#1]{\nu}}
\newcommand{\sumscoef}    [1][\relax]{\optional@sub[#1]{\gamma}}
\newcommand{\partialOrder}[1][\relax]{\optional@sub[#1]{\mathrm{m}}}
\newcommand{\orderReac}              {\ensuremath{{\mathrm{m}}}}
\newcommand{\massfrac}    [1][\relax]{\optional@sub[#1]{y}}
\newcommand{\mass}        [1][\relax]{\optional@sub[#1]{\rho}}
\newcommand{\mdot}        [1][\relax]{\optional@sub[#1]{\dot{\omega}}}
\newcommand{\molarfrac}   [1][\relax]{\optional@sub[#1]{x}}
\newcommand{\molar}       [1][\relax]{\optional@sub[#1]{c}}
\newcommand{\Eqconst}     [1][\relax]{\optional@sub[#1]{K}}
\newcommand{\fwdratecons} [1][\relax]{\ensuremath{\rcons[#1]^{(f)}}}
\newcommand{\bkwdratecons}[1][\relax]{\ensuremath{\rcons[#1]^{(b)}}}
\newcommand{\fwdrate}     [1][\relax]{\ensuremath{\rate[#1]^{(f)}}}
\newcommand{\bkwdrate}    [1][\relax]{\ensuremath{\rate[#1]^{(b)}}}
\newcommand{\ext}         [1][\relax]{\optional@sub[#1]{\xi}}
\newcommand{\nspecies}               {\ensuremath{\mathrm{n_{species}}}}
%%thermo
%%%% phase
\newcommand{\Vol}   [1][\relax]{\optional@sub[#1]{V}}
\newcommand{\Press} [1][\relax]{\optional@sub[#1]{P}}
\newcommand{\Temp}  [1][\relax]{\optional@sub[#1]{T}}
\newcommand{\RedTemp}          {\ensuremath{T_r}}
\newcommand{\Mol}   [1][\relax]{\optional@sub[#1]{N}}
\newcommand{\Gibbs} [1][\relax]{\optional@sub[#1]{G}}
\newcommand{\GibbsZ}[1][\relax]{\ensuremath{{\Gibbs[#1]}^{0}}}
\newcommand{\Enth}  [1][\relax]{\optional@sub[#1]{H}}
\newcommand{\Entr}  [1][\relax]{\optional@sub[#1]{S}}
\newcommand{\IntEn} [1][\relax]{\optional@sub[#1]{U}}
\newcommand{\Mass}  [1][\relax]{\optional@sub[#1]{m}}
%%%% species and Delta
\newcommand{\press}   [1][\relax]{\optional@sub[#1]{p}}
\newcommand{\chempot} [1][\relax]{\optional@sub[#1]{\mu}}
\newcommand{\chempotZ}[1][\relax]{\ensuremath{\chempot[#1]^{0}}}
\newcommand{\gibbs}   [1][\relax]{\optional@sub[#1]{g}}
\newcommand{\enth}    [1][\relax]{\optional@sub[#1]{h}}
\newcommand{\entr}    [1][\relax]{\optional@sub[#1]{s}}
\newcommand{\gibbsZ}  [1][\relax]{\ensuremath{{\gibbs[#1]}^{0}}}
\newcommand{\enthZ}   [1][\relax]{\ensuremath{{\enth[#1]}^{0}}}
\newcommand{\entrZ}   [1][\relax]{\ensuremath{{\entr[#1]}^{0}}}
\newcommand{\DGibbs}  [1][\relax]{\ensuremath{\optional@sub[#1]{\Delta}G}}
\newcommand{\DGibbsZ} [1][\relax]{\ensuremath{{\DGibbs[#1]}^{0}}}
\newcommand{\Denth}   [1][\relax]{\optional@sub[#1]{\Delta}\ensuremath{H}}
\newcommand{\DenthZ}  [1][\relax]{\ensuremath{{\Denth[#1]}^{0}}}
\newcommand{\Dentr}   [1][\relax]{\optional@sub[#1]{\Delta}\ensuremath{S}}
\newcommand{\DentrZ}  [1][\relax]{\ensuremath{{\Dentr[#1]}^{0}}}

%%%%%%%%%%%%%%%%%%%%%%%%%%%%%%%%%%%%%%%
%%
%% transport definitions
%%
%%%%%%%%%%%%%%%%%%%%%%%%%%%%%%%%%%%%%%%

\newcommand{\vis}           [1][\relax]{\optional@sub[#1]{\eta}}
\newcommand{\diff}          [1][\relax]{\optional@sub[#1]{D}}
\newcommand{\thermcond}     [1][\relax]{\optional@sub[#1]{\lambda}}
\newcommand{\LJdepth}       [1][\relax]{\optional@sub[#1]{\epsilon}}
\newcommand{\LJdia}         [1][\relax]{\optional@sub[#1]{\sigma}}
\newcommand{\Stockmayer}    [1]        {\ensuremath{\left<\Omega^{(#1,#1)*}\right>}}
\newcommand{\dipole}        [1][\relax]{\optional@sub[#1]{\mu}}
\newcommand{\polarizability}[1][\relax]{\optional@sub[#1]{\alpha}}
\newcommand{\specificHeat}  [1][\relax]{\optional@sub[#1]{C}}
\newcommand{\rotRelax}      [1][\relax]{\optional@sub[\text{rot}#1]{Z}}

%misc
\newcommand{\prodReac}    {\ensuremath{\prod_{\text{reactants}} \conc[\reac]^{\scoefabs[\reac]}}}
\newcommand{\prodReacMass}{\ensuremath{\prod_{\text{reactants}} \left(\frac{\mass[\reac]}{\Mm[\reac]}\right)^{\scoefabs[\reac]}}}
\newcommand{\prodProd}    {\ensuremath{\prod_\text{products}\conc[\product]^{\scoefabs[\product]}}}
\newcommand{\Kooij} [4][1]{\ensuremath{{#2} \left(\frac{T}{#1}\right)^{#3}\exp\left(-\frac{#4}{\Rg T}\right)}}
\newcommand{\KooijEq}     {\Kooij[\mathrm{T_{ref}}]{A}{\beta}{E_a}}
\newcommand{\dd}          {\ensuremath{\mathrm{d}}}
\newcommand{\unit}     [1]{\ensuremath{\mathtt{#1}}}
\newcommand{\nounit}      {\unit{no~unit}}
%%%%%
\newcommand{\reactionEq}[1]{%
\ce{#1}%
}
\newcommand{\ArrjPar}[4]{%
\texttt{Arrhenius} model\\*[2pt]
\null\hspace{12pt}$\left\{\begin{array}{l@{~=~}l}
\PreExp & #1~\unit{#2} \\
\AcEn   & #3~\unit{#4} \\
\end{array}\right.$%
}
\newcommand{\KooijPar}[5]{%
\texttt{Kooij} model\\*[2pt]
\null\hspace{12pt}$\left\{\begin{array}{l@{~=~}l}
\PreExp & #1~\unit{#2} \\
\Power  & #3           \\
\AcEn   & #4~\unit{#5} \\
\end{array}\right.$%
}
\newcommand{\TBcoeff}[2]{%
\ensuremath{\epsilon_{\ce{#1}} = #2}%
}
\newcommand{\reactionParameter}[3][\null]{%
\vspace{5pt minus 3pt}
\begin{minipage}{5.5cm}
\null\hfill#2\hfill\null\\[3pt]
#3\\[5pt]
#1%
\end{minipage}}
%% chemistry equation 
\newcounter{chemeq}
\setcounter{chemeq}{0}
\renewcommand{\thechemeq}{$\chi$~\thesection-\arabic{chemeq}}
\newenvironment{chemicalEquation}
{\begin{displaymath}}
{\refstepcounter{chemeq}\tag{\thechemeq}%
\end{displaymath}}
\@addtoreset{chemeq}{section}

%%%%%%%%%%%%%%%%%%%%%%%%%%%%%%%%
%%
%% code
%% code + output two column
%%
%%%%%%%%%%%%%%%%%%%%%%%%%%%%%%%%
%% language loading
\lstloadlanguages{C++,XML}
\lstset{basicstyle=\footnotesize,commentstyle=\color{blue}\bf,stringstyle=\color{magenta}\bf,
        numbers=left,numbersep=10pt,numberstyle=\scriptsize}

%% c++ code one column
\lstnewenvironment{cpp}[1][\footnotesize]
{\lstset{language=C++,
         morekeywords={Scalar,CoeffType,StateType},keywordstyle=\bf,
         morestring=[b]",
         morecomment=[s]{/*}{*/},morecomment=[l]{//},
         basicstyle=#1}
}
{}

%% c++ code two column
\lstnewenvironment{cpp|}
{\lstset{linewidth=0.9\linewidth,frame=r,framesep=-5pt,
         morekeywords={Scalar,CoeffType,StateType},keywordstyle=\bf,
         morestring=[b]",
         language=C++,morecomment=[s]{/*}{*/},morecomment=[l]{//},
         numbers=none}
}
{}

%% xml code one column
\lstnewenvironment{xml}[1][\footnotesize]
{\lstset{language=XML,
         morestring=[b]",
         basicstyle=#1}
}
{}

% output in terminal
\newlength{\outdepth}
\newcommand{\terminal}[1]{%
\settototalheight{\outdepth}{#1}
\ifdim\outdepth < 4em%
  \setlength{\outdepth}{5em}%
\fi%
\fcolorbox{white}{black}{%
\begin{minipage}[c][2\outdepth][c]{0.42\linewidth}
\color{white}
\textsf{[user@here \$]./program.x}\\
\textsf{#1}
\end{minipage}}
}

\makeatother
\begin{document}

\version{0}{1}{0}
\title{\Antioch\ my love}
\date{\theversion}
\author{Paul Bauman \and Benjamin Kirk \and Todd Oliver \and Sylvain Plessis \and Roy Stogner}

\maketitle
\tableofcontents
\listoffigures
\listoftables



\part{\Antioch\ from the user's point of view}
\chapter{Intro}

\section{This manual}
\Antioch\ stands for \ANTIOCH. For your convenience,
this manual is separated in three chapters. In the
chapter~\ref{Antioch:physics}, the underlying physics
and chemistry are developed; the chapter~\ref{Antioch:technique}
is dedicated to the implementation details; the chapter~\ref{Antioch:practice}
is the \textsf{User's Guide: A Tutorial} part of the manual. Finally
the appendices brings extra useful stuff.


\chapter{Let's go physical}
\chaptermark{\ANTIOCHPhys}
\label{Antioch:physics}

\section{Preliminary knowledge}

\subsection{Notations for this manual}
\begin{itemize}
\item \massfrac[i]:       mass fraction of species $i$, \nounit;
\item \mass[i]:           mass density of species $i$, homogeneous to \unit{g\,m^{-3}};
\item \Mass[i]:           mass species $i$, homogeneous to \unit{g};
\item \molarfrac[i]:      molar fraction of species $i$, \nounit;
\item \molar[i]/\conc[I]: molar density of species $i$ of name \ce{I}, homogeneous to \unit{mol\,m^{-3}};
\item \Mm[i]:             molar mass of species $i$, homogeneous to \unit{g\,mol^{-1}};
\item \fwdratecons[r]:    forward rate constant of reaction $r$, homogeneous to \unit{(m^3\,mol^{-1})^{\text{order} - 1}\,s^{-1}},
                            (\unit{m^3\,mol^{-1}\,s^{-1}} or \unit{s^{-1}} here, only first/zeroth order reaction);
\item \bkwdratecons[r]:   backward rate constant of reaction $r$, homogeneous to \unit{(m^3\,mol^{-1})^{\text{order} - 1}\,s^{-1}},
                            (\unit{m^3\,mol^{-1}\,s^{-1}} or \unit{s^{-1}} here, only first/zeroth order reaction);
\item \fwdrate[r]:        forward rate of reaction $r$, homogeneous to \unit{m^3\,mol^{-1}\,s^{-1}}; 
\item \bkwdrate[r]:       backward rate of reaction $r$, homogeneous to \unit{m^3\,mol^{-1}\,s^{-1}}; 
\item \Eqconst[r]:        equilibrium constant of reaction $r$, \nounit;
\item \gibbs[i]:          Gibbs free energy of species $i$, homogeneous to \unit{J\,mol^{-1}};
\item \DGibbs[r]:         Gibbs free energy of reaction $r$, homogeneous to \unit{J\,mol^{-1}};
\item \enth[i]:           enthalpy of species $i$, homogeneous to \unit{J\,mol^{-1}};
\item \Denth[r]:          enthalpy of reaction $r$, homogeneous to \unit{J\,mol^{-1}};
\item \entr[i]:           entropy of species $i$, homogeneous to \unit{J\,mol^{-1}};
\item \Dentr[r]:          entropy of reaction $r$, homogeneous to \unit{J\,mol^{-1}};
\item \chempot[i]:        chemical potential of species $i$, homogeneous to \unit{J};
\item \scoef[i,r]:        stoichiometric coefficient of species $i$ in reaction $r$, \nounit;
\item \scoefabs[i,r]:     absolute value of stoichiometric coefficient of species $i$ in reaction $r$, \nounit, this is the number
                            you see in chemical equation;
\item \sumscoef[r]:       sum of the stoichiometric coefficients for reaction $r$, \nounit;
\item \Temp[i]:           temperature of thermodynamic  phase $i$ homogeneous to \unit{K}; 
\item \Press[i]:          pressure of thermodynamic  phase $i$ homogeneous to \unit{Pa}; 
\item \Vol[i]:            volume of thermodynamic  phase $i$ homogeneous to \unit{m^3}; 
\item \Mol[i]:            number of mole of components of thermodynamic  phase $i$ homogeneous to \unit{mol}; 
\item \Mass[i]:           mass of thermodynamic  phase $i$ homogeneous to \unit{g}; 
\item \Gibbs[i]:          Gibbs free energy of thermodynamic phase $i$, homogeneous to \unit{J\,mol^{-1}};
\item \Enth[i]:           enthalpy of thermodynamic phase $i$, homogeneous to \unit{J\,mol^{-1}};
\item \Entr[i]:           entropy of thermodynamic phase $i$, homogeneous to \unit{J\,mol^{-1}\,K^{-1}}.
\item \vis[i]:            viscosity of species $i$, homogeneous to \unit{Pa\,s}.
\item \dipole[i]:         dipole moment of species $i$, homogeneous to \unit{C\,m}.
\item \diff[i]:           bimolecular diffusion of species $i$, homogeneous to \unit{m^2\,s^{-1}}.
\item \thermcond[i]:      thermal conduction of species $i$, homogeneous to \unit{J\,m^2\,s^{-1}}.
\item \LJdepth[i]:        Lennard-Jones potential well's depth, usually in reduced form, that
                          is divided by Boltzmann's constant (\BoltzmannEquation)
                          homegeneous (when reduced) to \unit{K}.
\item \LJdia[i]:          Lennard-Jones potential diameter, homogeneous to \unit{m}.
\end{itemize}
%
\begin{tabular}{lcccc}\toprule
Quantity  & species-wise & phase-wise & reaction-wise    & unit \\\midrule
Enthalphy & \enth        & \Enth      & \Denth/\DenthZ   & \unit{J\,mol^{-1}}\\
Entropy   & \entr        & \Entr      & \Dentr/\DentrZ   & \unit{J\,mol^{-1}\,K^{-1}}\\
Gibbs free energy
          & \gibbs       & \Gibbs     & \DGibbs/\DGibbsZ & \unit{J\,mol^{-1}}\\
Chemical potential
          & \chempot     & \chempot   &                  & \unit{J\,mol^{-1}}\\\cmidrule(lr){2-2}
Absolute stoichiometric coefficient
          & \scoefabs    &            &                  & \nounit\\
Stoichiometric coefficient
          & \scoef       &            & \sumscoef\ (sum) & \nounit\\\cmidrule(lr){2-2}
Number of mole
          &              & \Mol       &                  & \unit{mol}\\
Mass      & \Mass        & \Mass      &                  & \unit{g}\\
Density of mass
          & \mass        & \mass      &                  & \unit{g\,m^{-3}}\\
Concentration
          & \conc/\conc[.]
                         & \conc/\conc[M]
                                      &                  & \unit{mol\,m^{-3}}\\
Mass fraction
          & \massfrac    &   1        &                  & \nounit\\
Molar fraction
          & \molarfrac   &   1        &                  & \nounit\\\cmidrule(lr){2-2}
Viscosity & \vis         &   \vis     &                  & \unit{Pa\,s}\\
Dipole moment 
          & \dipole      &            &                  & \unit{C\,m}\\
Diffusion & \diff        &  \diff     &                  & \unit{m^2\,s^{-1}}\\
Thermal conduction 
          & \thermcond   &  \thermcond 
                                      &                  & \unit{J\,m^2\,s^{-1}}\\\cmidrule(lr){2-2}
Kinetic model
          &              &            &   \kinMod        & depends\\
Chemical process
          &              &            &   \chemProc      & depends\\
Forward rate constant
          &              &            &   \fwdratecons   & depends\\
Backward rate constant
          &              &            &   \bkwdratecons  & depends\\
Forward rate 
          &              &            &   \fwdrate       & \unit{mol\,m^{-3}\,s^{-1}}\\
Backward rate
          &              &            &   \bkwdrate      & \unit{mol\,m^{-3}\,s^{-1}}\\
Equilibrium constant
          &              &            &   \Eqconst       & \nounit\\
\bottomrule
\end{tabular}


\subsection{Some thoughts on units}
\label{units_in_Antioch}
The units in \antioch\ are managed at the reading
steps. The SI system is the internal system, thus
for kinetics parameters input file.


The parameters we need to be aware of:
\begin{itemize}
\item in the file \file{species\_ascii\_parsing.h}
        \begin{itemize}
        \item molecular weight: mass per quantity of matter (SI is \unit{kg\,mol$^{-1}$}),
        \item heat of formation at 0~\unit{K}: energy per mass (SI is \unit{J\,kg$^{-1}$})
        \end{itemize}
\item in the file \file{physical\_constants.h}
        \begin{itemize}
        \item \Rg: energy per quantity of matter per unit of temperature (SI is \unit{J\,mol$^{-1}$K$^{-1}$}),
        \item Avogadro number: one over quantity of matter (SI is \unit{mol$^{-1}$})
        \end{itemize}
\end{itemize}

Exemple with \Rg. If you go to the
\href{http://physics.nist.gov/cgi-bin/cuu/Value?r}{NIST definition}
of \Rg, you obtain the advised value of \Rg:
$\Rg = \numprint{8.3144621} \pm \numprint{0.0000075}$~\unit{J\,mol$^{-1}$K$^{-1}$}.
Thanks to \href{https://en.wikipedia.org/wiki/Gas_constant}{wikipedia}
(right-side table), we have also the value in \unit{calorie}:
$\Rg = \numprint{1.9858775} \pm \numprint{0.0000034}$~\unit{cal\,mol$^{-1}$K$^{-1}$}.
We look now at the definition of the \unit{calorie} unit, again,
we go to the \href{http://physics.nist.gov/Pubs/SP811/appenB9.html#ENERGY}{NIST website}.
We have then several definitions:
\begin{enumerate}
\item International Table, \unit{calorie$_\text{IT}$},   defined as \numprint{4.1868}  \unit{Joule};
\item thermodynamic, \unit{calorie$_\text{th}$},   defined as \numprint{4.184}   \unit{Joule};
\item mean (?), \unit{calorie$_\text{mean}$}, defined as \numprint{4.19002} \unit{Joule}.
\end{enumerate}
Comparing everyone leads to table~\ref{Rwtf}.
\begin{table}
\centering
\begin{tabular}{lcc}\toprule
\null\hfill Unit \hfill\null                      & \Rg\ value                          & factor to SI \\\midrule
\unit{J\,mol$^{-1}$K$^{-1}$}                      & \numprint{8.3144621}(75)            & \numprint{1.00000} \\
\unit{cal$_\text{IT}$\,mol$^{-1}$K$^{-1}$}        & \color{red}\numprint{1.9858752}(18) & \numprint{4.18680} \\
\unit{cal$_\text{th}$\,mol$^{-1}$K$^{-1}$}        & \color{red}\numprint{1.9872041}(18) & \numprint{4.18400} \\
\unit{cal$_\text{mean}$\,mol$^{-1}$K$^{-1}$}      & \color{red}\numprint{1.9843490}(18) & \numprint{4.19002} \\
\unit{cal$_\text{wikipedia}$\,mol$^{-1}$K$^{-1}$} & \numprint{1.9858775}(34)            & \color{red}\numprint{4.18680}\\
\bottomrule
\end{tabular}
\caption{\label{Rwtf}Discrepancies in \Rg. Red are calculated values from data.}
\end{table}
It seems obvious then that wikipedia gives the International Table (IT) calorie, with a
relative difference in the factor with the given one of $\numprint{1.1806}\,10^{-6}$. 
This relative error is bigger than machine tolerance (from the \textcolor{green!60!black}{\bf double}
to more precise).

We use thus the advised value in \unit{J\,mol$^{-1}$K$^{-1}$} and the factor
\numprint{4.1868} given by the NIST to make the conversion in \unit{calorie}.
Be aware than in this case, you have a discrepancy of $\numprint{1.2}\,10^{-6}$
with the wikipedia value.


\subsection{Useful relations}
\label{relations}
\begin{itemize}
\item $\massfrac[i] = \frac{\mass[i]}{\sum_s \mass[s]}$,
\item $\molarfrac[i] = \frac{\molar[i]}{\sum_s \molar[s]}$,
\item $\Mm[i] = \frac{\mass[i]}{\molar[i]}$,
\item $\frac{1}{\Mm[\text{mix}]} = \sum_s \frac{\massfrac[s]}{\Mm[s]}$ (cf.~\ref{demo-Mm})
\item $\molarfrac[i] = \massfrac[i] \frac{\Mm[\text{mix}]}{\Mm[i]}$,
\item $\Press = \molar\Rg \Temp$,
\item $\frac{\DGibbsZ[r]}{\Rg \Temp} = \frac{\DenthZ[r](\Temp)}{\Rg \Temp} - \frac{\DentrZ[r]}{\Rg}$
\end{itemize}



\section{Chemical kinetics}

\subsection{What is a reaction?}
\label{kinetics_gen}
A chemical reaction is ``a bunch of molecules
turning into another bunch of molecules'', kinetics\footnote{%
from the greek $\kappa\iota\nu\eta\sigma\iota\varsigma$, ``kinesis'', movement, to move}
is about answering the question ``how fast?''.
Thus chemical kinetics is the mathematical model
to calculate the rate at which the molecules disappear and
appear.

\subsubsection{Going forward}

\Antioch's kinetics is based on the elementary step
hypothesis. It means that, as far as the kinetics is
concerned, every reaction is an elementary step:
the reactants get together and produce the products
immediatly. Mathematically, it means
the partial orders are the absolute
value of the stoichiometric coefficients (see next).

Let's consider a chemical reaction:
\begin{chemicalEquation}
\ce{\scoefabs[A] A + \scoefabs[B] B ->[\rcons] \scoefabs[C] C + \scoefabs[D] D}
\label{genericX}
\end{chemicalEquation}
with \rcons\ the rate constant.
We want to model the evolution of the system, that is we want to
characterize 
$\doverdt{\conc[A]}$,
$\doverdt{\conc[B]}$,
$\doverdt{\conc[C]}$,
$\doverdt{\conc[D]}$.
Using the kinetics theory, we have:
\begin{equation}
\frac{1}{\scoef[A]}\doverdt{[A]} = 
\frac{1}{\scoef[B]}\doverdt{[B]} = 
\frac{1}{\scoef[C]}\doverdt{[C]} = 
\frac{1}{\scoef[D]}\doverdt{[D]} = 
\rcons\conc[A]^{\scoefabs[A]}\conc[B]^{\scoefabs[B]}
\end{equation}
with \scoef[A]\ being the stoichiometric coefficient, which is defined by:\\[5pt]
$\left\{\begin{array}{ll}
\scoef[S] = \scoefabs[S] & \text{if \ce{S} is a product} \\
\scoef[S] = -\scoefabs[S] & \text{if \ce{S} is a reactant} \\
\end{array}\right.$\\[5pt]
So the game is to define the rate constant. 

A rate constant is characterized by two things:
\begin{itemize}
\item the kinetics model,
\item the chemical process.
\end{itemize}
The kinetics model will mathematically describe the rate constant's dependence with
the temperature, it is noted \kinMod\ in this manual, the chemical process will
possibly add a pressure dependency, it is noted \chemProc, with \conc[M]\
denoting the pressure dependence.
\Antioch\ propose six different kinetics models and five chemical processes.
The rate constant is characterized thus, for a choice of a chemical process and
a kinetics model:
\begin{equation}
\rateCons = \chemProc
\end{equation}

\subsubsection{Going backward}

Usually, a reaction will be reversible, which means, if we consider
that reaction~\ref{genericX} is reversible, we should note it:
\begin{chemicalEquation}
\ce{\scoefabs[A] A + \scoefabs[B] B <=>[\fwdratecons][\bkwdratecons] \scoefabs[C] C + \scoefabs[D] D}
\label{genericXrev}
\end{chemicalEquation}
with \fwdratecons\ the forward rate constant and \bkwdratecons\ the backward rate constant.
In a given physico-chemical environment, this reaction will eventually reach
steady state, characterized by an equilibrium constant \Eqconst. 
This equilibrium constant is given by
\begin{equation}
\Eqconst[r] = \frac{\fwdratecons[r]}{\bkwdratecons[r]}
\label{therm:K_kin}
\end{equation}
for a reaction $r$.
It is possible to estimate it from thermodynamics considerations, 
using the relation
\begin{equation}
\Eqconst[r] = \left(\frac{\pz}{\Rg \Temp}\right)^\gamma \exp\left(-\frac{\DGibbsZ[r](\Temp)}{\Rg \Temp}\right)
\label{therm:K_therm}
\end{equation}
The demonstrations are given in appendix~\ref{demo-eq_kin} and \ref{demo-eq_therm}. 
Thus the backward rate constant is therefore known given:
\begin{itemize}
\item the forward rate constant;
\item the thermodynamics of the molecules.
\end{itemize}

\subsubsection{Going nowhere: steady state, a.k.a equilibrium}
\label{phys:equilibrium}
\subsubsection{With kinetics}
A steady state is defined by
%
\begin{equation}
\forall\:s,\quad \doverdt{\conc[s]} = 0
\label{equilibrium:def}
\end{equation}
%
The system to be solved is of the form:
\begin{equation}
A\times x = b
\end{equation}
with $b$ the vector of \mdot, $A$ the matrixes of \doverdm[E]{\mdot[s]} for
the species (rows are $s$ and columns are $E$) and $x$ the vector of the solution \mass.
To close the system, we use the mass conservation equation and use a
species to ensure it:
\begin{equation}
\sum_s \mass[s] = \mathrm{mass_{tot}}
\label{mass_cons}
\end{equation}
with $\mathrm{mass_{tot}}$ being a constant, here the density of mass of the system.

So, for $N$ chemical species, we have the system:
\begin{equation}
\left[\begin{array}{cccc}
\doverdm[s_1]{\mdot[s_1]}     & \doverdm[s_2]{\mdot[s_1]}     & \cdots & \doverdm[s_N]{\mdot[s_1]} \\
\doverdm[s_1]{\mdot[s_2]}     & \doverdm[s_2]{\mdot[s_2]}     & \cdots & \doverdm[s_N]{\mdot[s_2]} \\
\vdots                        & \vdots                        & \vdots & \vdots                    \\
\doverdm[s_1]{\mdot[s_{N-1}]} & \doverdm[s_2]{\mdot[s_{N-1}]} & \cdots & \doverdm[s_N]{\mdot[s_{N-1}]}\\
1                             & 1                             & \cdots & 1\\
\end{array}\right]
\left[\begin{array}{c}
\Delta\mass[s_1]\\
\Delta\mass[s_2]\\
\vdots\\
\Delta\mass[s_N]\\
\end{array}\right]
=
\left[\begin{array}{c}
\mdot[s_1]\\
\mdot[s_2]\\
\mdot[s_1]\\
\vdots\\
\mdot[s_{N_1}]\\
\sum_{s=1}^N\mass[s] - \mathrm{mass_{tot}}
\end{array}\right]
\label{eq:matrixes}
\end{equation}
The total fixed mass is calculated thanks to the ideal gas state equation
(see section~\ref{relations})
\begin{equation}
\mathrm{mass_{tot}} = \Mm[\mathrm{mix}] \frac{P}{\Rg T}
\label{tot_mass}
\end{equation}
with \Mm[\mathrm{mix}] calculated as seen in section~\ref{relations}.
Thus an initial guess of \massfrac\ is necessary.
If you don't have any idea, let's consider the situation
\begin{chemicalEquation}
\ce{A + B ->[k_1] C + D ->[k_2] E + F}
\label{youpi}
\end{chemicalEquation}
we have
\begin{equation}
\doverdt{\conc[C]} = k_1\conc[A]\conc[B] - k_2\conc[C]\conc[D]
\end{equation}
therefore, a first approximation can be
\begin{equation}
\conc[C^{(\text{approx})}] = \frac{k_1\conc[A]\conc[B]}{k_2\conc[D]} = \frac{\mathrm{prod}}{\mathrm{loss}}
\end{equation}

This will be efficient in somewhat easy situations, meaning you're looking for the
steady state of minor species for instance.

\subsubsection{With thermodynamics}
A thermodynamic phase is at equilibrium for a minimized Gibbs energy (at \Temp, \Press\ constant), with
the relation
\begin{equation}
\dd\Gibbs = \Vol\dd\Press - \Entr\dd\Temp + \sum_s\chempot[s]\dd\Mol[s]
\end{equation}
and we have Euler's identity
\begin{equation}
\Gibbs = \sum_s \Mol[s]\chempot[s]
\label{Euler_id}
\end{equation}
with also,
\begin{equation}
\left(\doverdext[r]{\Gibbs[s]}\right)_{\Temp,\Press} = \scoef[s,r]\chempot[s]
\end{equation}
We note
\begin{equation}
\DGibbs_r = \sum_s \scoef[s,r]\chempot[s] \left[= \sum_s \left(\doverdext[r]{\Gibbs[s]}\right)_{\Temp,\Press}\right]
\end{equation}
Considering
\begin{equation}
\chempot_s = \doverdn[s]{\Gibbs[s]}
\end{equation}
We have, deduced from~\ref{Euler_id}
\begin{equation}
\chempot[s] = \gibbs[s]
\end{equation}
Thus,
\begin{equation}
\chempot[s] = \chempotZ[s] + \Rg\Temp\ln\left(\frac{\press[s]}{\pz}\right)
     \left[ = \chempotZ[s] + \Rg\Temp\ln\left(\frac{\Press}{\pz}\right) + \Rg\Temp\ln\left(\molarfrac[s]\right) \right]
\end{equation}
The story behind chemical extent is:
\begin{equation}
\Mol[s] = \Mol[s](t=0) + \sum_r \scoef[s,r] \ext[r]
\end{equation}
Using the ideal gas equation:
\begin{equation}
\Press = \conc \Rg \Temp
\end{equation}
thus
\begin{equation}
\begin{split}
\chempot[s] & = \chempotZ[s] + \Rg\Temp\ln\left(\frac{\Rg \Temp \molar[s]}{\pz}\right) \\
            & = \chempotZ[s] + \Rg\Temp\ln\left(\frac{\Rg \Temp}{\pz}\left(\molar[s](t=0) + \sum_r \scoef[s,r]\frac{\ext[r]}{\Vol}\right)\right) \\
\end{split}
\end{equation}
and therefore
\begin{equation}
\begin{split}
\doverdext[r]{\chempot[s]}      & = \frac{\pz}{\Vol}\frac{\scoef[s,r]}{\molar[s](t=0) + \sum_{r'} \scoef[s,r']\frac{\ext[r']}{\Vol}}\\
\ddoverddext{\chempot[s]}{r}{l} & = -\frac{\pz}{\Vol^2}\frac{\scoef[s,l]\scoef[s,r]}{\left(\molar[s](t=0) + \sum_{r'} \scoef[s,r']\frac{\ext[r']}{\Vol}\right)^2}\\
\end{split}
\end{equation}
Usually, it's better to consider the system per unit of volume, 
using an extent of reaction per volume, thus the equations
become:
\begin{equation}
\begin{split}
\chempot[s]                     & = \chempotZ[s] + \Rg\Temp\ln\left(\frac{\Rg \Temp}{\pz}\left(\molar[s](t=0) + \sum_r \scoef[s,r]\ext[r]\right)\right) \\
\doverdext[r]{\chempot[s]}      & = \pz\frac{\scoef[s,r]}{\molar[s](t=0) + \sum_{r'} \scoef[s,r']\ext[r']}\\
\ddoverddext{\chempot[s]}{r}{l} & = -\pz\frac{\scoef[s,l]\scoef[s,r]}{\left(\molar[s](t=0) + \sum_{r'} \scoef[s,r']\ext[r']\right)^2}\\
\end{split}
\end{equation}

Equilibrium is defined by 
\begin{equation}
\forall\; r,\; \DGibbs_r = 0
\end{equation} 
or 
\begin{equation}
\min \Gibbs(\{\chempot[s]\}_s)
\end{equation}

\subsubsection{\texorpdfstring{$\forall\;r,\;\DGibbs_r = 0$}{Reaction enthalpy}}

\begin{equation}
\begin{split}
\DGibbs_r & = \sum_s \scoef[s,r] \chempot[s]\\
\doverdext[i]{\DGibbs_r}   & = \sum_s\scoef[s,r]\doverdext[i]{\chempot[s]} \\
\ddoverddext{\DGibbs_r}{i}{j} & = \sum_s\scoef[s,r]\ddoverddext{\chempot[s]}{i}{j} \\
\end{split}
\end{equation}

For $R$ reactions:
\begin{equation}
\left[\begin{array}{cccc}
\sum_s\scoef[s,1]\doverdext[1]{\chempot[s]} & \sum_s\scoef[s,1]\doverdext[2]{\chempot[s]} & \cdots & \sum_s\scoef[s,1]\doverdext[R]{\chempot[s]}\\
\sum_s\scoef[s,2]\doverdext[1]{\chempot[s]} & \sum_s\scoef[s,2]\doverdext[2]{\chempot[s]} & \cdots & \sum_s\scoef[s,2]\doverdext[R]{\chempot[s]}\\
\vdots &\vdots &\vdots &\vdots \\
\sum_s\scoef[s,R]\doverdext[1]{\chempot[s]} & \sum_s\scoef[s,R]\doverdext[2]{\chempot[s]} & \cdots & \sum_s\scoef[s,R]\doverdext[R]{\chempot[s]}\\
\end{array}\right]
\times
\left[\begin{array}{c}
\Delta\ext[1] \\
\Delta\ext[2] \\
\vdots \\ 
\Delta\ext[R]
\end{array}\right]
=
\left[\begin{array}{c}
\sum_s \scoef[s,1] \chempot[s] \\ 
\sum_s \scoef[s,2] \chempot[s] \\ 
\vdots \\ 
\sum_s \scoef[s,R] \chempot[s]
\end{array}\right]
\end{equation}

\subsubsection{\texorpdfstring{$\min\Gibbs(\{\chempot[s]\}_s)$}{Phase enthalpy minimization}}

\begin{equation}
\begin{split}
\Gibbs & = \sum_s\Mol[s]\chempot[s]
         = \sum_s\left(\Mol[s]^0 + \sum_r\scoef[s,r]\ext[r]\right)\chempot[s] \\
\doverdext[r]{\Gibbs} & = \sum_s \left[\left(\Mol[s]^0 + \sum_{r'}\scoef[s,r']\ext[r']\right)\doverdext[r]{\chempot[s]}
                                + \scoef[s,r]\chempot[s]\right] \\
\ddoverddext{\Gibbs}{r}{l} & = \sum_s \left[\left(\Mol[s]^0 + \sum_{r'}\scoef[s,r']\ext[r']\right)\ddoverddext{\chempot[s]}{r}{l}
                                        + \scoef[s,l]\doverdext[r]{\chempot[s]}
                                        + \scoef[s,r]\doverdext[l]{\chempot[s]}\right] \\
\end{split}
\end{equation}
Considering these equations per unit of volume, one obtains:
\begin{equation}
\begin{split}
\Gibbs & = \sum_s\left(\molar[s]^0 + \sum_r\scoef[s,r]\ext[r]\right)\chempot[s] \\
\doverdext[r]{\Gibbs} & = \sum_s \left[\left(\molar[s]^0 + \sum_{r'}\scoef[s,r']\ext[r']\right)\doverdext[r]{\chempot[s]}
                                + \scoef[s,r]\chempot[s]\right] \\
\ddoverddext{\Gibbs}{r}{l} & = \sum_s \left[\left(\molar[s]^0 + \sum_{r'}\scoef[s,r']\ext[r']\right)\ddoverddext{\chempot[s]}{r}{l}
                                        + \scoef[s,l]\doverdext[r]{\chempot[s]}
                                        + \scoef[s,r]\doverdext[l]{\chempot[s]}\right] \\
\end{split}
\end{equation}




\subsection{Kinetics computing}
\label{kinetics_comput}
\subsubsection{Kinetics models}
The kinetics models are all given in Tab.~\ref{antioch::kinMod}.
The kinetics model will render the temperature evolution of
the rate constant, thus the only differential that is needed
at that level is the temperature differentiation.
\begin{table}
\centering\renewcommand{\arraystretch}{1.5}
\begin{tabular}{clr}\toprule
\multirow{2}{*}{Kinetics model}
                & Expression                     & Parameters \\
                & $\frac{\dd}{\dd T}$ Expression &\\\midrule
\multirow{2}{*}{Hercourt-Hessen} 
                & $\kinMod = \PreExp \left(\frac{\Temp}{\Tref}\right)^\Power$ 
                        & \PreExp, \Power\\
                & $\ddoverdT{\kinMod} = \kinMod \frac{\Power}{\Temp}$ \\[10pt]
\multirow{2}{*}{Berthelot}
                & $\kinMod = \PreExp \exp\left(\BerthExp\Temp\right)$ 
                        & \PreExp, \BerthExp\\
                & $\ddoverdT{\kinMod} = \kinMod \BerthExp$ \\[10pt]
\multirow{2}{*}{Arrhenius}
                & $\kinMod = \PreExp \exp\left(-\frac{\AcEn}{\Rg\Temp}\right)$ 
                        & \PreExp, \AcEn\\
                & $\ddoverdT{\kinMod} = \kinMod \frac{\AcEn}{\Rg\Temp^2}$ \\[10pt]
\multirow{2}{*}{Kooij}
                & $\kinMod = \PreExp \left(\frac{\Temp}{\Tref}\right)^\Power\exp\left(-\frac{\AcEn}{\Rg\Temp}\right)$
                        & \PreExp, \Power, \AcEn\\
                & $\ddoverdT{\kinMod} = \kinMod \left(\frac{\Power}{\Temp} + \frac{\AcEn}{\Rg\Temp^2}\right)$ \\[10pt]
\multirow{2}{*}{Berthelot Hercourt-Essen}
                & $\kinMod = \PreExp \left(\frac{\Temp}{\Tref}\right)^\Power\exp\left(\BerthExp\Temp\right)$
                        & \PreExp, \Power, \BerthExp\\
                & $\ddoverdT{\kinMod} = \kinMod \left(\frac{\Power}{\Temp} + \BerthExp\right)$ \\[10pt]
\multirow{2}{*}{Van't Hoff}
                & $\kinMod = \PreExp \left(\frac{\Temp}{\Tref}\right)^\Power\exp\left(\BerthExp\Temp-\frac{\AcEn}{\Rg\Temp}\right)$
                        & \PreExp, \Power, \BerthExp, \AcEn\\
                & $\ddoverdT{\kinMod} = \kinMod \left(\frac{\Power}{\Temp} + \BerthExp +  \frac{\AcEn}{\Rg\Temp^2}\right)$ \\
\bottomrule
\end{tabular}
\caption{\label{antioch::kinMod}Kinetics models available in \antioch.}
\end{table}

\subsubsection{Chemical processes}
\label{subsec:chem_proc}
The chemical processes is how this kinetics model behave with respect to
the densities of the molecules in the medium. Either they're all equivalent,
and only the total density matters (falloff type), or each species can 
influence the rate constant in a specific manner (three-body type).
The chemical processes are given in Tab.~\ref{antioch::chemProc}. This is where
chemical modeling begins to be stretched a little: as seen on section~\ref{kinetics_gen},
we assume that every reaction modeled is an elementary step, even in the case of
non-elementary reactions. Therefore some corrections need to be made. This is where
the chemical process happen adds, if needed, more suppleness to the rate constant
function, from adding degrees of freedom to the temperature dependency (duplicate
chemical process) to adding a pressure dependance over the temperature dependance
(falloff).
\begin{table}
\centering\renewcommand{\arraystretch}{2}
\begin{tabular}{cll}\toprule
\multirow{2}{*}{Chemical process}
                  & Expression             & $\doverdT{\text{Expression}}$ \\
                  &                        & $\doverdc{\text{Expression}}$ \\\midrule
\multirow{2}{*}{Elementary}
                  & $\chemProc = \kinMod$  & $\doverdT{\chemProc} = \ddoverdT{\kinMod}$ \\
                  &                        & $\doverdc{\chemProc} = 0$ \\[10pt]
\multirow{2}{*}{Duplicate \dag}
                  & $\chemProc = \displaystyle\sum_i^{\mathrm{N_{proc}}}\kinMod_i$
                                           & $\doverdT{\chemProc} = \displaystyle\sum_i^\mathrm{N_{proc}}\ddoverdT{\kinMod_i}$ \\
                  &                        & $\doverdc{\chemProc} = 0$ \\[10pt]
\multirow{2}{*}{Three-Body}
                  & $\chemProc = \kinMod \threeBody$
                                           & $\doverdT{\chemProc} = \ddoverdT{\kinMod}\threeBody$ \\
                  &                        & $\doverdc[I]{\chemProc} = \kinMod\epsilon_i\sum_{s\neq i}\epsilon_s\conc[S] $ \\[10pt]
\multirow{3}{*}{Lindemann falloff}
                  & $\chemProc = \frac{\conc[M]\kinModZ}{1 + \conc[M]\frac{\kinModZ}{\kinModI}}\FLind$ 
                                          & \ref{Falloff:doverdT} with $F = \FLind$\\
                  &                       & \ref{Falloff:doverdc} with $F = \FLind$\\[10pt]
\multirow{3}{*}{Troe falloff}
                  & $\chemProc = \frac{\conc[M]\kinModZ}{1 + \conc[M]\frac{\kinModZ}{\kinModI}}\FTroe$
                                          & \ref{Falloff:doverdT} with $F = \FTroe$\\
                  &                       & \ref{Falloff:doverdc} with $F = \FTroe$\\[10pt]
\bottomrule
\end{tabular}
\caption{\label{antioch::chemProc}Chemical processes available in \antioch.
\dag: the duplicate chemical process do not permit several kinetics
models to be mixed. The functions $F$ for the falloff are described in section~\ref{subsub:falloff}}
\end{table}

\subsubsection{Falloffs}
\label{subsub:falloff}
The falloffs are reaction having two different regimes: a low-pressure (\kinModZ) and
a high-pressure (\kinModI) regime. Any pressure's rate constant can be expressed in
terms if those two regimes:
\begin{equation}
\chemProc = \frac{\conc[M]\kinModZ}{1 + \conc[M]\frac{\kinModZ}{\kinModI}} F
\label{falloff:k}
\end{equation}
with \conc[M] being the total molar density and $F$ the falloff model.

The differentiations are:
\begin{equation}
\begin{split}
\doverdT{\chemProc} & = \chemProc 
                        \Bigg[\frac{1}{F}\doverdT{F} \\
                    &      + \left(\frac{1}{\kinModZ} - \frac{\conc[M]}{\kinModI}\frac{1}{1 + \conc[M]\frac{\kinModZ}{\kinModI}}\right)\doverdT{\kinModZ} \\
                    &      - \conc[M]\frac{\kinModZ}{\kinModI^2}\frac{1}{1 + \conc[M]\frac{\kinModZ}{\kinModI}}\doverdT{\kinModI}
                        \Bigg]
\end{split}
\label{Falloff:doverdT}
\end{equation}
and
\begin{equation}
\doverdc[I]{\chemProc} = \chemProc
                          \left(\frac{1}{\conc[M]} 
                               + \frac{\kinModZ}{\kinModI}\frac{1}{1 + \conc[M]\frac{\kinModZ}{\kinModI}} 
                               + \doverdT{F}\frac{1}{F}
                          \right)
\label{Falloff:doverdc}
\end{equation}
\paragraph{Lindemann}
The Lindemann falloff is the simplest falloff, it is given by:
\begin{equation}
\FLind = 1,
\label{Falloff:Lindeman}
\end{equation}
which gives:
\begin{equation}
\begin{split}
\doverdT{\FLind} & = 0 \\
\doverdc{\FLind} & = 0 \\
\end{split}
\label{Falloff:Lindemann:diff}
\end{equation}

\paragraph{Troe}
The Troe model is more elaborate and rely on the definition of three or
four parameters:
\begin{itemize}
\item \Troealpha, with \nounit,
\item \TroeTone\ in \unit{K},
\item \TroeTtwo\ in \unit{K},
\item \TroeTthree\ in \unit{K}.
\end{itemize}
Sometimes, the parameter \TroeTtwo\ is not provided, as its contribution
may be negligible.
The falloff calculations need the definition of several parameters:
\begin{equation}
\begin{split}
\TroeFcent & = \left(1 - \Troealpha\right)\exp\left(-\frac{\Temp}{\TroeTthree}\right)
              +          \Troealpha       \exp\left(-\frac{\Temp}{\TroeTone}\right)
              +                           \exp\left(-\frac{\TroeTtwo}{\Temp}\right) \\[5pt]
\Troen     & =   \numprint{0.75} - \numprint{1.27}\log_{10}\left(\TroeFcent\right)  \\[5pt]
\Troec     & = - \numprint{0.4}  - \numprint{0.64}\log_{10}\left(\TroeFcent\right)  \\[5pt]
\Troed     & =   \numprint{0.14}
\end{split}
\label{Troe:parameter}
\end{equation}
to finally have
\newcommand{\pr}{\ensuremath{\log_{10}\left(\frac{\conc[M]\kinMod_0}{\kinMod_\infty}\right)}}
\begin{equation}
\FTroe = \frac{\log_{10}\left(\TroeFcent\right)}
              {1 + \left[
                        \log_{10}\left(
                                        \frac{\pr + \Troec}
                                             {\Troen - \Troed\left(
                                                                \pr + \Troec
                                                            \right)}
                                 \right) 
                   \right]^2}
\label{Troe:F}
\end{equation}
Thus the differentials will be:
\begin{equation}
\begin{split}
\doverdT{\TroeFcent} & = - \frac{1 - \Troealpha}{\TroeTthree}\exp\left(-\frac{\Temp}{\TroeTthree}\right)
                         - \frac{\Troealpha}{\TroeTone}\exp\left(-\frac{\Temp}{\TroeTone}\right)
                         + \frac{\TroeTtwo}{\Temp^2}\exp\left(-\frac{\TroeTtwo}{\Temp}\right)\\[5pt]
\doverdT{\Troen}     & = - \frac{\numprint{1.27}}{\ln(10)\TroeFcent}\doverdT{\TroeFcent} \\[5pt]
\doverdT{\Troec}     & = - \frac{\numprint{0.64}}{\ln(10)\TroeFcent}\doverdT{\TroeFcent} \\[5pt]
\doverdT{\Troed}     & = 0
\end{split}
\label{Troe:dparameterdT}
\end{equation}
and of course
\begin{equation}
\doverdc{\TroeFcent}  = 0,\qquad
\doverdc{\Troen}      = 0,\qquad
\doverdc{\Troec}      = 0,\qquad
\doverdc{\Troed}      = 0
\label{Troe:dparameterdc}
\end{equation}
\begin{figure}
\centering
\includegraphics{falloffs}
\caption{\label{kinetics::falloffs}%
Example of falloff for the reaction \ce{CH3 + CH3 (+M) -> C2H6 (+M)}.
The rate constants are calculated for a temperature of $T = 1000~\unit{K}$. The kinetics model
used is a Kooij model. Parameters are:
$\PreExp_0      = \numprint{1.135}\,10^{36}~\unit{mol^{-2}\,cm^6\,s^{-1}}$,
$\Power_0       = \numprint{-5.245}$,
$\AcEn_0        = \numprint{1704.8}~\unit{cal\,mol^{-1}}$,
$\PreExp_\infty = \numprint{6.22}\,10^{16}~\unit{mol^-2\,cm^6\,s^{-1}}$,
$\Power_\infty  = \numprint{-1.174}$,
$\AcEn_\infty   = \numprint{653.8}~\unit{cal\,mol^{-1}}$,
$\Troealpha     = \numprint{0.405}$,
$\TroeTone      = \numprint{69.6}~\unit{K}$ and
$\TroeTthree    = \numprint{1120.0}~\unit{K}$.}
\end{figure}


\section{Full derivations for a chemical kinetics system}
\label{derivations}
What we want is 
$\mdot[S] = \doverdt{\mass[S]}$, and its derivatives:
$\doverdT{\mdot[S]}$, $\doverdm[j]{\mdot[S]}$ with respect to species $j$.

The kinetics model's value is noted \kinMod, the rate constant is noted \rateCons, the
rate of the reaction is noted \rate.

The kinetics model is Kooij 
\begin{equation}
\kinMod = \KooijEq,
\label{KooijEq}
\end{equation}
the chemical process are elementary 
\begin{equation}
\rateCons = \kinMod
\label{ChemProEl}
\end{equation} 
or three-body 
\begin{equation}
\rateCons = \kinMod \threeBody.
\label{ChemProTB}
\end{equation}

We use the equations
\begin{equation}
\mass[S] = \Mm[S] \conc[S]
\end{equation}
\begin{equation}
\dd\mass[S] = \Mm[S]\,\dd\conc[S]
\end{equation}
as everything in kinetics theory is expressed in concentrations and 
not in terms of mass.

We have thus:
\begin{equation}
\mdot[S] = \doverdt{\mass[S]} = \Mm[S] \doverdt{\conc[S]}
\end{equation}
\begin{equation}
\doverdm[E]{\mdot[S]} = \frac{\Mm[S]}{\Mm[E]}\doverdc[E]{\conc[S]}
\end{equation}
\begin{equation}
\doverdT{\mdot[S]} = \Mm[S]\doverdT{\conc[S]}
\end{equation}
but for Paul's sake, we will do the full derivation with mass.

\subsection{Forward}

The forward rate \fwdrate\ for a reaction is by definition for species \ce{S}:
\begin{equation}
\begin{split}
\fwdrate &= \frac{1}{\scoef[S]}\frac{\dd\conc[S]}{\dd t}\\
         &= \frac{1}{\scoef[S]} \fwdratecons \prodReac
\end{split}
\label{ratefDef}
\end{equation}

\paragraph{To derive with respect to \Temp.}
For any species \ce{S} participating in the reaction\footnote{remember, 
$\scoef = \left\{\begin{array}{l}-\scoefabs\text{ for reactants}\\\scoefabs\text{ for products}\end{array}\right.$}.
thus,
\begin{equation}
\doverdT{\fwdrate} = \frac{1}{\scoef[S]}
                   \doverdT{\fwdratecons}
                   \prodReac
\end{equation}
and, for the elementary process, following~\ref{ChemProEl} and \ref{KooijEq}:
\begin{equation}
\begin{split}
\doverdT{\fwdratecons} & = \doverdT{\kinMod} \\
                       & = \frac{\kinMod}{\Temp} \left(\frac{E_a}{\Rg \Temp} + \beta\right)
\end{split}
\end{equation}
For the three-body process, following~\ref{ChemProTB} and \ref{KooijEq}:
\begin{equation}
\begin{split}
\doverdT{\fwdratecons} & = \doverdT{\kinMod} \threeBody \\
                       & = \frac{\kinMod}{\Temp} \left(\frac{E_a}{\Rg \Temp} + \beta\right) \threeBody
\end{split}
\end{equation}

Finally,
for the elementary processes
\begin{equation}
\doverdT{\fwdrate} = \frac{1}{\scoef[S]} \frac{\kinMod}{\Temp} \left(\frac{E_a}{\Rg \Temp} + \beta\right)
                                                \prodReac
\label{derivTEP}
\end{equation}
and the three-body processes
\begin{equation}
\doverdT{\fwdrate} = \frac{1}{\scoef[S]} \frac{\kinMod}{\Temp} \left(\frac{E_a}{\Rg \Temp} + \beta\right) \threeBody
                                                \prodReac
\label{derivTTB}
\end{equation}

\paragraph{To derive with respect to \mass.}
Knowing that $\conc = \frac{\mass}{\Mm}$, we substitute in~\ref{ratefDef}
\begin{equation}
\fwdrate = \frac{1}{\scoef[S]} \fwdratecons \prod_\text{reactants}  \left(\frac{\mass[\reac]}{\Mm[\reac]}\right)^{\scoefabs[\reac]}
\end{equation}
We obtain for a derivation with respect to \mass[E]\ of species \ce{E}:
\begin{equation}
\doverdm[E]{\fwdrate} = \fwdrate \left[
                                \doverdm[E]{\fwdratecons} \frac{1}{\fwdratecons} +
                                \underbrace{\frac{\scoefabs[E]}{\mass[E]}}_{\text{if \ce{E} is a reactant}}
                              \right]
\end{equation}

For elementary process:
\begin{equation}
\doverdm[E]{\fwdratecons} = 0
\end{equation}
For a three-body process:
\begin{equation}
\doverdm[E]{\fwdratecons} = \kinMod \frac{\epsilon_{\ce{E}}}{\Mm[E]}
\end{equation}

Finally, for an elementary process:
\begin{equation}
\doverdm[E]{\fwdrate} = \left\{\begin{array}{ll}
                        \fwdrate \frac{\scoefabs[E]}{\mass[E]} & \text{if \ce{E} is a reactant} \\
                        0                                    & \text{if \ce{E} is not a reactant} \\
                      \end{array}\right.
\end{equation}
and for a three-body process:
\begin{equation}
\renewcommand{\arraystretch}{1.5}
\doverdm[E]{\fwdrate} = \left\{\begin{array}{ll}
                        \fwdrate \left[\frac{\epsilon_{\ce{E}}}{\Mm[E]\threeBody} + \frac{\scoefabs[E]}{\mass[E]} \right] 
                                                & \text{if \ce{E} is a reactant} \\
                        \fwdrate \frac{\epsilon_{\ce{E}}}{\Mm[E]\threeBody} 
                                                & \text{if \ce{E} is not a reactant} \\
                      \end{array}\right.
\end{equation}

\subsection{Backward}

The backward rate is
\begin{equation}
\bkwdrate = \frac{1}{\scoef[S]}\bkwdratecons\prodProd
\end{equation}

The backward rate constant is given by
\begin{equation}
\bkwdratecons = \frac{\fwdratecons}{\Eqconst}
\end{equation}

\paragraph{Derive with respect to \Temp.}
It makes:
\begin{equation}
\doverdT{\bkwdrate} = \frac{1}{\scoef[S]}\prodProd\doverdT{\bkwdratecons}
\end{equation}
Decomposition gives
\begin{equation}
\doverdT{\bkwdratecons} = \doverdT{\fwdratecons}\Eqconst^{-1} - \doverdT{\Eqconst}\frac{\bkwdratecons}{\Eqconst^{2}}
\label{rateb-decomp}
\end{equation}
For \Eqconst, we have
\begin{equation}
\begin{split}
\doverdT{\Eqconst} & = \doverdT{\left[\left(\frac{\pz}{\Rg \Temp}\right)^{\sum_s \scoef[s]} \exp\left(-\frac{\DGibbsZ(\Temp)}{\Rg \Temp}\right)\right]} \\
                   & = \Eqconst\left[-\frac{\sum_s\scoef[s]}{\Temp} - \doverdT{\left[\frac{\DGibbsZ(\Temp)}{\Rg \Temp}\right]}\right]
\end{split}
\end{equation}
The derived value of \DGibbsZ(\Temp) is easily given, see~\ref{data-thermo}
\begin{equation}
\begin{split}
\doverdT{\left[\frac{\DGibbsZ(\Temp)}{\Rg \Temp}\right]} 
        &= \doverdT{\left[\frac{\DenthZ(T)}{\Rg \Temp}\right]} - \doverdT{\left[\frac{\DenthZ(\Temp)}{\Rg \Temp}\right]} \\
        &= 2\tc{0}\Temp^{-3} + \tc{1}\Temp^{-2}\left(1+\ln(\Temp)\right) + \frac{\tc{3}}{2} + 
           \frac{2}{3}\tc{4}\Temp + \frac{3}{4}\tc{5}\Temp^{2} + \frac{4}{5}\tc{6}\Temp^3 - \tc{8}\Temp^{-2}\\
        &+ \tc{0}\Temp^{-3} + \tc{1}\Temp^{-2} + \tc{2}\Temp^{-1} + \tc{3} + \tc{4}\Temp + \tc{5} \Temp^{2} + \tc{6} \Temp^{3}
\end{split}
\end{equation}
Finally,
\begin{equation}
\doverdT{\left[\frac{\DGibbsZ(\Temp)}{\Rg \Temp}\right]} =
        3\tc{0}\Temp^{-3} + \tc{1} \Temp^{-2} \left(2 + \ln(\Temp)\right) + \frac{3}{2} \tc{3} + \frac{5}{3} \tc{4} \Temp
        + \frac{7}{4}\tc{5} \Temp^2 + \frac{9}{5}\tc{6} \Temp^3 - \tc{8} \Temp^{-2}
\end{equation}
\emph{Note:} $\ln(\Temp)$ should be understood as $\ln\left(\frac{\Temp}{\mathrm{\Temp_{ref}}}\right)$ with 
$\mathrm{\Temp_{ref}} = 1$~K.

Thus, for elementary processes
\begin{equation}
\begin{split}
\doverdT{\bkwdrate} = & \frac{1}{\scoef[S]} \prodProd \\
&\bigg[
        \frac{\kinMod}{\Temp} \left(\frac{E_a}{\Rg \Temp} + \beta\right)\prodReac \Eqconst^{-1}
                  - 3\tc{0}\Temp^{-3} - \tc{1} \Temp^{-2} \left(2 + \ln(\Temp)\right) - \frac{3}{2} \tc{3} \\
&                 - \frac{5}{3} \tc{4} \Temp - \frac{7}{4}\tc{5} \Temp^2 - \frac{9}{5}\tc{6} \Temp^3 + \tc{8} \Temp^{-2} \bigg]
\end{split}
\end{equation}
and for third-body processes
\begin{equation}
\begin{split}
\doverdT{\bkwdrate} = & \frac{1}{\scoef[S]} \prodProd \\
&\bigg[
        \frac{\kinMod}{\Temp} \left(\frac{E_a}{\Rg \Temp} + \beta\right) \threeBody \Eqconst^{-1} 
                  - 3\tc{0}\Temp^{-3} - \tc{1} \Temp^{-2} \left(2 + \ln(\Temp)\right) - \frac{3}{2} \tc{3} \\
&                 - \frac{5}{3} \tc{4} \Temp - \frac{7}{4}\tc{5} \Temp^2 - \frac{9}{5}\tc{6} \Temp^3 + \tc{8} \Temp^{-2} \bigg]
\end{split}
\end{equation}

\paragraph{Derivation with respect to mass.}
Using~\ref{rateb-decomp}
\begin{equation}
\begin{split}
\doverdm[E]{\bkwdratecons} & = \doverdm[E]{\fwdratecons}\Eqconst^{-1} - \doverdm[E]{\Eqconst}\frac{\bkwdratecons}{\Eqconst^{2}} \\
                           & = \doverdm[E]{\fwdratecons}\Eqconst^{-1}
\end{split}
\label{rateb-decomp-m}
\end{equation}
thus, for elementary processes
\begin{equation}
\doverdm[E]{\rate} = \left\{\begin{array}{ll}
                        \fwdrate \frac{\scoefabs[E]}{\mass[E]} \Eqconst^{-1} & \text{if \ce{E} is a reactant} \\
                          0                                                  & \text{if \ce{E} is not a reactant} \\
                      \end{array}\right.
\end{equation}
and for three-body processes
\begin{equation}
\renewcommand{\arraystretch}{1.5}
\doverdm[E]{\fwdrate} = \left\{\begin{array}{ll}
                        \fwdrate \left[\frac{\epsilon_{\ce{E}}}{\Mm[E]\threeBody} + \frac{\scoefabs[E]}{\mass[E]} \right] \Eqconst^{-1}
                                                & \text{if \ce{E} is a reactant} \\
                        \fwdrate \frac{\epsilon_{\ce{E}}}{\Mm[E]\threeBody} \Eqconst^{-1}
                                                & \text{if \ce{E} is not a reactant} \\
                      \end{array}\right.
\end{equation}

\subsection{Net rate}

The net rate is
\begin{equation}
\rate = \fwdrate - \bkwdrate
\end{equation}
thus,
\begin{equation}
\doverdT{\rate} = \doverdT{\fwdrate} - \doverdT{\bkwdrate}
\end{equation}
and
\begin{equation}
\doverdm{\rate} = \doverdm{\fwdrate} - \doverdm{\bkwdrate}
\end{equation}
thus, for elementary processes
\begin{equation}
\begin{split}
\doverdT{\rate[{\text{net}}]} &= \frac{1}{\scoef[S]} \frac{\kinMod}{\Temp} \left(\frac{E_a}{\Rg \Temp} + \beta\right) \prodReac\\
& -
\frac{1}{\scoef[S]} \prodProd 
\bigg[
        \frac{\kinMod}{\Temp} \left(\frac{E_a}{\Rg \Temp} + \beta\right)\prodReac \Eqconst^{-1} \\
                  & \hspace{3cm} - 3\tc{0}\Temp^{-3} - \tc{1} \Temp^{-2} \left(2 + \ln(\Temp)\right) - \frac{3}{2} \tc{3} \\
                  & \hspace{3cm} - \frac{5}{3} \tc{4} \Temp - \frac{7}{4}\tc{5} \Temp^2 - \frac{9}{5}\tc{6} \Temp^3 + \tc{8} \Temp^{-2} \bigg]
\end{split}
\end{equation}
and for three-body processes
\begin{equation}
\begin{split}
\doverdT{\rate[{\text{net}}]} & = \frac{1}{\scoef[S]} \frac{\kinMod}{\Temp} \left(\frac{E_a}{\Rg \Temp} + \beta\right) \threeBody\prodReac\\
& -
\frac{1}{\scoef[S]} \prodProd 
\bigg[
        \frac{\kinMod}{\Temp} \left(\frac{E_a}{\Rg \Temp} + \beta\right)\threeBody\prodReac \Eqconst^{-1} \\
                  & \hspace{3cm} - 3\tc{0}\Temp^{-3} - \tc{1} \Temp^{-2} \left(2 + \ln(\Temp)\right) - \frac{3}{2} \tc{3} \\
                  & \hspace{3cm} - \frac{5}{3} \tc{4} \Temp - \frac{7}{4}\tc{5} \Temp^2 - \frac{9}{5}\tc{6} \Temp^3 + \tc{8} \Temp^{-2} \bigg]
\end{split}
\end{equation}
and for the mass, elementary processes
\begin{equation}
\doverdm[E]{\rate} = \left\{\begin{array}{ll}
                        \fwdrate \frac{\scoefabs[E]}{\mass[E]}\left(1-\Eqconst^{-1}\right) & \text{if \ce{E} is a reactant} \\
                          0                                                                & \text{if \ce{E} is not a reactant} \\
                      \end{array}\right.
\end{equation}
and for the three-body processes
\begin{equation}
\renewcommand{\arraystretch}{1.5}
\doverdm[E]{\rate} = \left\{\begin{array}{ll}
                        \fwdrate \left[\frac{\epsilon_{\ce{E}}}{\Mm[E]\threeBody} + \frac{\scoefabs[E]}{\mass[E]} \right] \left(1 - \Eqconst^{-1}\right)
                                                & \text{if \ce{E} is a reactant} \\
                        \fwdrate \frac{\epsilon_{\ce{E}}}{\Mm[E]\threeBody}  \left(1 - \Eqconst^{-1}\right)
                                                & \text{if \ce{E} is not a reactant} \\
                      \end{array}\right.
\end{equation}

Back to \mdot:
\begin{equation}
\begin{split}
\mdot[S] &= \Mm[S] \doverdt{\ce{S}} \\
         &= \Mm[S] \scoef[S]\rate
\end{split}
\end{equation}

\begin{equation}
\renewcommand{\arraystretch}{1.5}
\left\{\begin{array}{l}
\doverdT{\mdot[S]} = \Mm[S] \scoef[S]\doverdT{\rate}\\
\doverdm[E]{\mdot[S]} = \Mm[S] \scoef[S]\doverdm[E]{\rate}
\end{array}\right.
\end{equation}



\section{Transport}
\label{transport}

\subsection{Viscosity}
\label{transport:viscosity}
\Antioch\ has three viscosity models available:
\begin{itemize}
\item Blottner viscosity model;
\item Sutherland viscosity model;
\item viscosity as derived from kinetics theory under the ideal gas assumption.
\end{itemize}

\subsubsection{Blottner}
\label{transport:viscosity:blottner}

This viscosity model is a fit of the form:
\begin{equation}
\vis = \numprint{0.1} \exp\left((a \ln(\Temp) + b) \ln(\Temp) + c\right)
\label{blottner:eq}
\end{equation}
$a$, $b$, $c$ being the Blottner parameter for the considered
species.

\subsubsection{Sutherland}
\label{transport:viscosity:sutherland}

This viscosity model is of the form:
\begin{equation}
\vis = \vis_\text{ref} \frac{\Temp^{\numprint{1.5}}}{\Temp + \text{T}_\text{ref}}
\label{sutherland:eq}
\end{equation}

\subsubsection{Kinetics Theory}
\label{transport:viscosity:kin_theo}

The kinetics theory uses the assumption of ideality in a 
gaseous medium. This assumption means basically that the
state law is
\begin{equation}
\Press = \conc\Rg \Temp
\end{equation}
with \Press\ the pressure (\unit{Pa}), \conc\ the molar concentration
(\unit{mol\,m^{-3}}), \Rg\ the universal gas constant (\RgEquation)
and \Temp\ the temperature (\unit{K}).

We use the following formul\ae\ for the species $s$, from \citet{Monchick1961}:
\begin{equation}
\vis[s] = \frac{5}{16} \sqrt{\frac{\Mass[s]\Boltzmann\Temp}{\pi}}\frac{f_{\vis}}{\LJdia[s]^2\Stockmayer{2}}
\label{visc_kin_theory:eq}
\end{equation}
with \Mass\ the molecular mass of the species (\unit{kg}), \LJdia\ the Lennard-Jones
potential diameter associated to the species (\unit{m}), \Boltzmann\ the Boltzmann
constant (\BoltzmannEquation), \Stockmayer{2}\ is the
value of the dimensionless integrated integral and $f_{\vis}$ a factor close
to one (see Eq.~5 in \citet{Monchick1961}). We use the 
the hard sphere approximation ($f_{\vis} = 1$), and a polynomial fit to \Stockmayer{2}.


\subsection{Diffusion}
\label{transport:diffusion}
Diffusion to be put here


\subsection{Thermal conduction}
\label{transport:thermal}
thermal conduction to be put here


\subsection{Stockmayer potentials}
\label{transport:Stockmayer}
The Stokmayer potential is a generalization of the Lennard-Jones potential
(see Fig.~\ref{viscosity:Stockmayer_potential} and \citet{Jasper2014} for a
full analysis of the Lennard-Jones potential):
\begin{equation}
\varphi(r) = \underbrace{4 \LJdepth \left[\left(\frac{\LJdia[0]}{r}\right)^{12} - \left(\frac{\LJdia[0]}{r}\right)^{6}\right]}_\text{Lennard-Jones}
             - \frac{\dipole[i]\dipole[j]}{r^3}\xi
\label{Stockmayer:equation}
\end{equation}
with 
\begin{equation}
\xi = 2 \cos(\theta_1)\cos(\theta_2) - \sin(\theta_1)\sin(\theta_2)\cos(\phi)
\end{equation}
\begin{figure}
\centering
\includegraphics{Stockmayer_angles}
\caption[Stockmayer/Lennard-Jones potential]{\label{viscosity:Stockmayer_potential}The angles considered in the
Stockmayer potential are the axis inclination ($\theta_i$) and the azimuthal
angle between the dipoles ($\phi = \phi_2 - \phi_1$). Below is given the
Lennard-Jones potential.}
\end{figure}

The integrated collision integrals (\Stockmayer{1} and \Stockmayer{2}) are tabulated
in function of the reduced temperature and reduced dipole moment:
\begin{equation}
\Temp_\text{red} = \frac{\Boltzmann\Temp}{\LJdepth}
\end{equation}
and
\begin{equation}
\dipole_\text{red} = \frac{1}{2}\frac{\dipole^2}{\LJdepth\LJdia^3}
\end{equation}

For a pair a species, needed for the molecular binary diffusion, we have the relations
%%
\begin{equation}
\begin{array}{l@{\qquad}l}
\renewcommand{\arraystretch}{1.2}
\left\{\begin{array}{l@{\quad}l}
  \LJdia_{ij}                      & = \frac{1}{2}\left(\LJdia_{i} + \LJdia_{j}\right) \\
  \frac{\LJdepth_{ij}}{\Boltzmann} & = \sqrt{\left(\frac{\LJdepth_{i}}{\Boltzmann}\right)\left(\frac{\LJdepth_{j}}{\Boltzmann}\right)} \\
  \dipole_{ij}                     & = \sqrt{\dipole_j\dipole_k}\\
\end{array}\right. & 
  \text{if $i$ and $j$ are both polar or non polar} \\\\
\left\{\begin{array}{l@{\quad}l}
  \LJdia_{ij}                      & = \frac{1}{2}\left(\LJdia_{i} + \LJdia_{j}\right) \xi^2 \\
  \frac{\LJdepth_{ij}}{\Boltzmann} & = \sqrt{\left(\frac{\LJdepth_{i}}{\Boltzmann}\right)\left(\frac{\LJdepth_{j}}{\Boltzmann}\right)}\; \xi^{-\frac{1}{6}} \\
  \dipole_{ij}                     & = 0\\
\end{array}\right. & 
  \text{else} \\
\end{array}
\label{LJ_reduced_parameters}
\end{equation}
%%
with
%%
\begin{equation}
\xi = 1 + \frac{1}{4}\polarizability^{(\not{\dipole})}_\text{red}\dipole^{(\dipole)}_\text{red}\sqrt{\frac{\LJdepth^{(\dipole)}}{\LJdepth^{(\not{\dipole})}}}
\label{reduced_parameter:xi}
\end{equation}
%%
with the notation $(\not\!\!\dipole)$ denoting the non polar
species and $(\dipole)$ the polar species.
%%
\begin{equation}
\polarizability_\text{red}^{(\not{\dipole})} = \frac{\polarizability}{\LJdia^3}
\label{reduced_parameter:pol_non_pol:polarizability}
\end{equation}
%%%%
and
%%%%
\begin{equation}
\dipole_\text{red}^{(\dipole)} = \frac{\dipole}{\sqrt{\LJdepth\LJdia^3}}
\label{reduced_parameter:pol_non_pol:dipole}
\end{equation}


\chapter{Let's go technical}
\chaptermark{\ANTIOCHTech}
\label{Antioch:technique}

\section{So many templates}

\section{C++ 11}

\section{One tuple, two tuples, three tuples,\dots}

\section{\EIGEN}

\section{\Boost}

\section{\MetaPhysicL}

\section{Off to the GPU with \VexCL}

\section{\ViennaCL}

\section{\GRVY}

\section{Don't forget \Doxygen}
\label{dox}
The implementation details, in terms of file, class,
and algorithm are in the \Doxygen\ documentation.
Once you've boostrapped your \Antioch\ version, a
simple \verb!make doc! will give you joy, happiness
and insights into \Antioch.


\chapter{Let's go practical}
\chaptermark{\ANTIOCHPrac}
\label{Antioch:practice}

\section{Let's get data}

\subsection{Data}
\label{kin:data}
Mixture: 
\ce{N2}, \ce{O2}, \ce{N}, \ce{O}, \ce{NO}.
(see thermo tab. at~\ref{data-thermo}).

Reactions:\\*
\begin{tabular}{*{3}{m{4cm}}}
\ce{N2 + M <=> 2 N + M} & \ce{O2 + M <=> 2 O + M} & \ce{NO + M <=> N + O + M} \\[5pt]
\parbox{4cm}{%
$A = 7\,10^{18}~\unit{m^3\,kmol^{-1}\,s^{-1}}$,\\
$\beta = -1.6$,\\
$E_a = 224801.3~\unit{cal\,mol^{-1}}$\\
$\epsilon_\ce{N2} = 1.0$,  $\epsilon_\ce{O2} = 1.0$,  
      $\epsilon_\ce{NO} = 1.0$,  $\epsilon_\ce{N} = 4.2857$,  
      $\epsilon_\ce{O} = 4.2857$.}
&
\parbox{4cm}{%
$A = 2\,10^{18}~\unit{m^3\,kmol^{-1}\,s^{-1}}$,\\
$\beta = -1.5$,\\
$E_a = 117881.7~\unit{cal\,mol^{-1}}$\\
$\epsilon_\ce{N2} = 1.0$,  $\epsilon_\ce{O2} = 1.0$,  
      $\epsilon_\ce{NO} = 1.0$,  $\epsilon_\ce{N} = 5.0$,  
      $\epsilon_\ce{O} = 5.0$.}
&
\parbox{4cm}{%
$A = 5\,10^{12}~\unit{m^3\,kmol^{-1}\,s^{-1}}$,\\
$\beta = 0.0$,\\
$E_a = 149943.0~\unit{cal\,mol^{-1}}$
$\epsilon_\ce{N2} = 1.0$,  $\epsilon_\ce{O2} = 1.0$,  
      $\epsilon_\ce{NO} = 22.0$,  $\epsilon_\ce{N} = 22.0$,  
      $\epsilon_\ce{O} = 22.0$.}
\\\\
\ce{N2 + O <=> NO + N} & \ce{NO + O <=> O2 + N}\\[5pt]
\parbox{4cm}{%
$A = 5.7\,10^{9}~\unit{m^3\,kmol^{-1}\,s^{-1}}$,\\
$\beta = 0.42$,\\
$E_a = 85269.6~\unit{cal\,mol^{-1}}$}
&       
\parbox{4cm}{%
$A = 8.4\,10^{9}~\unit{m^3\,kmol^{-1}\,s^{-1}}$,\\
$\beta = 0.0$,\\
$E_a = 38526.0~\unit{cal\,mol^{-1}}$}
\end{tabular}
\medskip

Data used:\\*
\begin{tabular}{l}
$\Mm_\ce{N} = 14.008~\unit{g\,mol^{-1}}$, \\
$\Mm_\ce{O} = 16.000~\unit{g\,mol^{-1}}$, \\
$\Rg = 8314.4621~\unit{J\,kmol^{-1}\,K^{-1}}$.
\end{tabular}

Conditions:\\*
\begin{tabular}{l}
$P = 10^5~\unit{Pa}$, \\
$T = 1500.0~\unit{K}$.
\end{tabular}


\subsection{First thing first: the input file}
At this point, \antioch-\theversion, only the \verb!xml! format is
supported, but soon some \verb!hdf5! joy will be added.

\subsubsection{The file}

Thus you want to have a proper input file, therefore providing
\Antioch\ with the proper data, as seen in subsection~\ref{kin:data}.
This is pretty straightforward, here what it should look like:
\begin{xml}
<?xml version="1.0"?>
<ctml>
  <reactionData id="some name, does not matter">
    <!-- reaction 0001    -->
    <reaction reversible="yes" type="ThreeBody" id="0001">
      <equation>N2 + M [=] 2 N + M</equation>
      <rateCoeff>
        <Kooij>
           <A units="m3/kmol/s">7.e+18</A>
           <b>-1.6</b>
           <E units="cal/mol">224801.3</E>
        </Kooij>
        <efficiencies default="1.0">N:4.2857 O:4.2857 </efficiencies>
      </rateCoeff>
      <reactants>N2:1.0</reactants>
      <products>N:2.0</products>
    </reaction>

    <!-- reaction 0002    -->
    <reaction reversible="yes" type="ThreeBody" id="0002">
      <equation>O2 + M [=] 2 O + M</equation>
      <rateCoeff>
        <Kooij>
           <A units="m3/kmol/s">2.e+18</A>
           <b>-1.5</b>
           <E units="cal/mol">117881.7</E>
        </Kooij>
        <efficiencies default="1.0">N:5.0 O:5.0</efficiencies>
      </rateCoeff>
      <reactants>O2:1.0</reactants>
      <products>O:2.0</products>
    </reaction>

    <!-- reaction 0003    -->
    <reaction reversible="yes" type="ThreeBody" id="0003">
      <equation>NO + M [=] N + O + M</equation>
      <rateCoeff>
        <Arrhenius>
           <A units="m3/kmol/s">5.e+12</A>
           <E units="cal/mol">149943.0</E>
        </Arrhenius>
        <efficiencies default="1.0">NO:22 N:22 O:22</efficiencies>
      </rateCoeff>
      <reactants>NO:1</reactants>
      <products>N:1 O:1</products>
    </reaction>

    <!-- reaction 0004    -->
    <reaction reversible="yes" type="Elementary" id="0004">
      <equation>N2 + O [=] NO + N</equation>
      <rateCoeff>
        <Kooij>
           <A units="m3/kmol/s">5.7e+9</A>
           <b>0.42</b>
           <E units="cal/mol">85269.6</E>
        </Kooij>
      </rateCoeff>
      <reactants>N2:1 O:1</reactants>
      <products>NO:1 N:1</products>
    </reaction>

    <!-- reaction 0005    -->
    <reaction reversible="yes" type="Elementary" id="0005">
      <equation>NO + O [=] O2 + N</equation>
      <rateCoeff>
        <Arrhenius>
           <A units="m3/kmol/s">8.4e+09</A>
           <E units="cal/mol">38526.0</E>
        </Arrhenius>
      </rateCoeff>
      <reactants>NO:1 O:1</reactants>
      <products>O2:1 N:1</products>
    </reaction>
  </reactionData>
<\ctml>
\end{xml}

The things you want to know about this file are simple:
\begin{itemize}
\item \verb!reactionData! needs to be called \verb!reactionData!, 
        this is hard-coded and imposed;
\item if you place several \verb!reactionData! groups, only the first
        one is considered;
\item every keyword is actually hard-coded and imposed, check
        throughfully your input file before launching it;
\item \Antioch's a little touchy with the formatting of the inputs.
        If you don't provide the units for example, she\footnote{Yes, \Antioch's feminine,
        if you don't like it, quit whining and get over it.} will use
        default units (see Tab.~\ref{unit:default}), but she'll complain.
\item Historically, a \verb!GRI-Mech! file format was used, thus defining
        the defaults. Using \verb!GRI-Mech! files will give you the expected
        behavior, but she'll complain.
\end{itemize}

\subsubsection{Formatting your inputs}

Letting alone the \verb!xml! imposed rules, once you've put
the lines
\begin{xml}
<?xml version="1.0"?>
<ctml>
</ctml>
\end{xml}
you enter into \Antioch's realm, her rules apply from there, don't
resist, there's no hero, you will eventually abide by her rules.
Again, let's insist on the \verb!reactionData! environment, only
a \verb!reactionData! is searched, and only the first is read.
Any meta data given in the \verb!reactionData! tag is ignored.

Before starting, a little vocabulary for this section:
\begin{itemize}
\item an \emph{environment} requires an opening tag
        and a closing tag. Anything in between belongs to it.
\item an \emph{attribute} is within the opening tag of an
        environment. 
\item a \emph{value} is the value associated with either
        an environment or an attribute.
\end{itemize}
An attribute's value is in between quotation marks, an environment's
value is between its tags, with nothing else.
%%%%%% horrible, make it better some time
\begin{center}
\begin{minipage}{9cm}
\tt
\tikz[overlay,baseline={(0,-3pt)}]\draw[red,stealth-] (0,0) -- (-2,0)node[left]{Opening tag};<anEnvironment %
   anAttr\tikz[overlay,baseline={(0,-6pt)}]\draw[red,stealth-](0,0) |- (2,0.5)node[right]{Attribute} ;ibute="attribu%
               \tikz[overlay,baseline={(0,2pt)}]\draw[red,stealth-] (0,0) |- (2,-0.5)node[right]{Attribute value};te value">\\
\tikz[overlay,baseline={(0,-3pt)}]\draw[red,stealth-,shorten <=-2em](0,0) -- (-2,0)node[left]{Environment value};%
\null\hspace{2em} environment value\\
\tikz[overlay,baseline={(0,-3pt)}]\draw[red,stealth-](0,0) -- (-2,0)node[left]{Closing tag};</anEnvironment>
\end{minipage}
\end{center}

Let's consider a reaction closely:
\begin{xml}
    <!-- reaction 0005    -->
    <reaction reversible="yes" type="Elementary" id="0005">
      <equation>NO + O [=] O2 + N</equation>
      <rateCoeff>
        <Arrhenius>
           <A units="m3/kmol/s">8.4e+09</A>
           <E units="cal/mol">38526.0</E>
        </Arrhenius>
      </rateCoeff>
      <reactants>NO:1 O:1</reactants>
      <products>O2:1 N:1</products>
    </reaction>
\end{xml}
%
\begin{table}
\centering
\begin{tabular}{cccc}\toprule
Parameter         & keyword(s)              & Default value  & Type\\\midrule
Chemical reaction & \verb!reaction!         &                & environment \\
Chemical process  & \verb!type!             & Elementary     & attribute   \\
Elementary        & \verb!Elementary!       &                & value       \\
Duplicate         & \verb!Duplicate!        &                & value       \\
Three-body        & \verb!ThreeBody! or
                    \verb!threeBody!        &                & value       \\
Lindemann falloff & \verb!LindemannFalloff! &                & value       \\
Troe Falloff      & \verb!TroeFalloff!      &                & value       \\[5pt]
id                & \verb!id!               &                & attribute   \\
Reversible?       & \verb!reversible!       & yes            & attribute   \\[5pt]
Chemical equation & \verb!equation!         &                & environment \\
Reactants         & \verb!reactants!        &                & environment \\
Products          & \verb!products!         &                & environment \\[5pt]
Rate constant     & \verb!rateCoeff!        &                & environment \\
Rate constant's name
                  & \verb!name!             &                & attribute   \\
\kinModZ          & \verb!k0!               &                & value       \\
Efficiencies      & \verb!efficiency!       & 1.0            & environment \\[5pt]
Hercourt-Essen    & \verb!HercourtEssen!    &                & environment \\
Arrhenius         & \verb!Arrhenius!        &                & environment \\
Berthelot         & \verb!Berthelot!        &                & environment \\
Berthelot Hercourt-Essen
                  & \verb!BerthelotHercourtEssen!  &         & environment \\
Kooij             & \verb!Kooij!            &                & environment \\
Van't Hoff        & \verb!VantHoff!         &                & environment \\
Photochemistry    & \verb!photochemistry!   &                & environment \\[5pt]
\Tref             & \verb!Tref!             & 1~\unit{K}     & environment \\
\PreExp           & \verb!A!                &                & environment \\
\Power            & \verb!b!                &                & environment \\
\AcEn             & \verb!E!                &                & environment \\
\BerthExp         & \verb!D!                &                & environment \\
\wavelength       & \verb!lambda!           &                & environment \\
\crosssection     & \verb!cross_section!    &                & environment \\[5pt]
Unit              & \verb!units!            & see Tab. \ref{unit:default} 
                                                             & attribute \\
\bottomrule
\end{tabular}
\caption{\label{antioch:keyword_reading}Keyword for the \texttt{xml} input file.}
\end{table}

\begin{table}
\centering
\begin{tabular}{lcccccc}\toprule
                   & \verb!A!                                        & \verb!b! & \verb!E!             & \verb!D!       & \verb!lambda! & \verb!cross_section! \\\midrule
Elementary         & \unit{(m^3\,mol^{-1})^{\orderReac - 1}\,s^{-1}} & \nounit  & \unit{cal\,mol^{-1}} & \unit{K^{-1}}  &  \unit{nm}    &  \unit{cm^2\,nm^{-1}} \\
Duplicate          & \unit{(m^3\,mol^{-1})^{\orderReac - 1}\,s^{-1}} & \nounit  & \unit{cal\,mol^{-1}} & \unit{K^{-1}}  &   --          & -- \\
ThreeBody          & \unit{(m^3\,mol^{-1})^{\orderReac}\,s^{-1}}     & \nounit  & \unit{cal\,mol^{-1}} & \unit{K^{-1}}  &   --          & -- \\
Falloff (\kinModZ) & \unit{(m^3\,mol^{-1})^{\orderReac}\,s^{-1}}     & \nounit  & \unit{cal\,mol^{-1}} & \unit{K^{-1}}  &   --          & -- \\
Falloff (\kinModI) & \unit{(m^3\,mol^{-1})^{\orderReac - 1}\,s^{-1}} & \nounit  & \unit{cal\,mol^{-1}} & \unit{K^{-1}}  &   --          & -- \\
\bottomrule
\end{tabular}
\caption{\label{unit:default}Default units given to the parameters by \Antioch\ 
if not provided in the input file.}
\end{table}

Let's build it together, asking and answering every question that
arises along the way.
So we start from an \verb!xml! empty reaction database:
%
\begin{xml}
<?xml version="1.0"?>
<ctml>
  <reactionData>
  </reactionData>
</ctml>
\end{xml}
%
We want to add a reaction, with a commentary for the
reader:
\begin{xml}
<?xml version="1.0"?>
<ctml>
  <reactionData>
    <!-- my very first xml reaction, how cute! -->
    <reaction>
    </reaction>
  </reactionData>
</ctml>
\end{xml}
This reaction, as can be seen in chapter~\ref{Antioch:physics},
is characterized by its chemical process, its kinetics model and
if it is reversible or not. So, as attribute, we will provide
the chemical process and the reversibility of the reaction. An id
can be provided too, \Antioch\ will use it for warning or error
message purpose at the reading step.
\begin{xml}
<?xml version="1.0"?>
<ctml>
  <reactionData>
    <!-- my very first xml reaction, how cute! -->
    <reaction reversible="yes" type="Elementary" id="Reaction1">
    </reaction>
  </reactionData>
</ctml>
\end{xml}
Next thing we want, is the equation, we simply put it in the
appropriate environment:
\begin{xml}
<?xml version="1.0"?>
<ctml>
  <reactionData>
    <!-- my very first xml reaction, how cute! -->
    <reaction reversible="yes" type="Elementary" id="Reaction1">
      <equation>NO + O [=] O2 + N </equation>
    </reaction>
  </reactionData>
</ctml>
\end{xml}
Along with the equation, comes the description of the reaction
in terms of reactants and products. There we need to define
each reactant (resp. product) with its stoichiometric
coefficient:
\begin{xml}
<?xml version="1.0"?>
<ctml>
  <reactionData>
    <!-- my very first xml reaction, how cute! -->
    <reaction reversible="yes" type="Elementary" id="Reaction1">
      <equation>NO + O [=] O2 + N </equation>
      <reactants>NO:1 O:1 </reactants>
      <products>O2:1 N:1 </products>
    </reaction>
  </reactionData>
</ctml>
\end{xml}
This step is needed because \Antioch\ does not parse the
chemical equation (so basically you can get creative there),
\Antioch\ parses these environments to internally define the
reaction.

Next step is to define the rate constant. All these information
are going into a \verb!rateCoeff! environment. Several cases arise
there. In our case, an elementary reaction, we but need to define
the kinetics models and its parameters, each with its unit. But
if we had a three-body reaction, we then should define the efficiencies,
if we had a falloff, we need to define two rate constants.
The kinetics model is an environment, each parameter is an environment
too, having a unit attribute (except \Power).
Thus the general layout is:
\begin{xml}
      <rateCoeff>
        <kinModel name="k0">
          <Parameter units="appropriate unit"> value of parameter</A>
        </kinModel>
        <efficiencies> molecule:efficiency ... </efficiencies>
      </rateCoeff>
\end{xml}
with the \verb!name! attribute having a sense only in a falloff
reaction, and the \verb!efficiencies! for three-body reactions.
The efficiencies are formatted as shown, the molecule, a colon, the
stoichiometric coefficient. If a molecule has no efficiency
given, the default is 1.0.
Thus in our case, we have an Arrhenius kinetics model with the
following parameters:
\begin{itemize}
\item $\PreExp = 8.4\,10^9~\unit{m^3\,kmol^{-1}\,s^{-1}}$,
\item $\AcEn = \numprint{38526.0}~\unit{cal\,mol^{-1}}$.
\end{itemize}
Thus, our final reaction is therefore:
\begin{xml}
<?xml version="1.0"?>
<ctml>
  <reactionData>
    <!-- my very first xml reaction, how cute! -->
    <reaction reversible="yes" type="Elementary" id="Reaction1">
      <equation>NO + O [=] O2 + N </equation>
      <reactants>NO:1 O:1 </reactants>
      <products>O2:1 N:1 </products>
      <rateCoeff>
        <Arrhenius>
           <A units="m3/kmol/s">8.4e+09</A>
           <E units="cal/mol">38526.0</E>
        </Arrhenius>
      </rateCoeff>
    </reaction>
  </reactionData>
</ctml>
\end{xml}
Note that the order of environments inside environment have little
interest, at the slight exception of falloff. If ever you are
as bold as to provide a falloff without precision as to what rate
constant is the low pressure limit, the so-called \kinModZ, then
\Antioch\ will consider that the first one encountered is the
low-pressure limit.

Congratulations, you have a nice input file.


\section{Let's code some kinetics regression}
Let's face it, \Antioch\ is mainly about calculating
efficiently, with elegance and a sexy attitude, rates,
be them with respect to mass or molar concentration. 

The first thing to clear out is the template choice. 
\Antioch\ is very suple about this, 
The objects we will need are
a \prog{ReactionSet} and a \prog{KineticsEvaluator}. We will
use as \prog{double} a \prog{double} and as vector the standard 
\prog{std::vector< >} 

\begin{verbatim}
// C++
#include <string>
#include <vector>

// Antioch
#include "antioch/vector_utils.h"

#include "antioch/antioch_asserts.h"
#include "antioch/chemical_species.h"
#include "antioch/chemical_mixture.h"
#include "antioch/reaction_set.h"
#include "antioch/read_reaction_set_data_xml.h"
#include "antioch/cea_thermo.h"
#include "antioch/kinetics_evaluator.h"

int compute_rates(const std::string& input_name,
  std::vector<double> & omega_dot,
  std::vector<double> & domega_dot_dT,
  std::vector<std::vector<double> > & domega_dot_drho_s)
{
  std::vector<std::string> species_str_list;
  const unsigned int n_species = 5;
  species_str_list.reserve(n_species);
  species_str_list.push_back( "N2" );
  species_str_list.push_back( "O2" );
  species_str_list.push_back( "N" );
  species_str_list.push_back( "O" );
  species_str_list.push_back( "NO" );

  Antioch::ChemicalMixture<double> chem_mixture( species_str_list );
  Antioch::ReactionSet<double> reaction_set( chem_mixture );
  Antioch::CEAThermodynamics<double> thermo( chem_mixture );

  Antioch::read_reaction_set_data_xml<double>( input_name, true, 
                                               reaction_set );

  const double T = 1500.0; // K
  const double P = 1.0e5; // Pa

  // Mass fractions
  std::vector<double> Y(n_species,0.2);

  const double R_mix = chem_mixture.R(Y); //get R_tot in J.kg-1.K-1

  const double rho = P/(R_mix*T); // kg.m-3

  std::vector<double> molar_densities(n_species,0.0);
  chem_mixture.molar_densities(rho,Y,molar_densities);

  std::vector<double> h_RT_minus_s_R(n_species);
  std::vector<double> dh_RT_minus_s_R_dT(n_species);

  typedef typename Antioch::CEAThermodynamics<double>::template 
                Cache<double> Cache;
  thermo.h_RT_minus_s_R(Cache(T),h_RT_minus_s_R);
  thermo.dh_RT_minus_s_R_dT(Cache(T),dh_RT_minus_s_R_dT);

  Antioch::KineticsEvaluator<double> kinetics( reaction_set, 0 );

  omega_dot.resize(n_species,0.L);
  domega_dot_dT(n_species,0.L);
  domega_dot_drho_s(n_species);
  for( unsigned int s = 0; s < n_species; s++ )
    {
      domega_dot_drho_s[s].resize(n_species);
    }
  
  kinetics.compute_mass_sources( T, molar_densities, 
                                 h_RT_minus_s_R, omega_dot);

  kinetics.compute_mass_sources_and_derivs( T, molar_densities, 
                                            h_RT_minus_s_R, dh_RT_minus_s_R_dT,
                                            omega_dot, domega_dot_dT, 
                                            domega_dot_drho_s );
  return 0;
}
\end{verbatim}


\section{The \Units\ object}
\Antioch\ takes care of the units for every
parameter thanks to the \Units\ object. This object
enables powerful unit evaluation and conversion,
of a physical quantity as well as an equation.

A description of this object and its dependencies is
given in the \Doxygen\ documentation, thus only a
brief description and some examples will be given here.

The \Units\ objects projects an unit on a unit basis,
defined by the SI system plus radian for angles. There are
three different parts:
\begin{itemize}
\item the symbol of the unit,
\item the power array (basis projection),
\item the coefficient to convert the unit to the basis.
\end{itemize}

All is defined so as to provide an easy use should the user
want to play (or test/analyze) the dimension of some physical
quantity.

The basis is defined as: \unitbase.

The rules for a unit are simple:
\begin{itemize}
\item ``.'' and ``/'' are the separators 
        (thus \verb!m.s-1! and \verb!m/s! are equivalent),
\item it supports grouping with parenthesises.
\end{itemize}

\subsection{Simple example}

Let's consider an energy for instance, say parameter
$E$ is given in \unit{cal}. You would initialize a \Units\
object by 
\begin{cpp}
Units<Scalar> E_unit("cal")
\end{cpp}
\prog{Scalar} being the precision you want, typically \prog{float},
\prog{double} or \prog{long double}. Then you can obtain factors
to any unit that is homogeneous to an energy:
\begin{cpp}
Scalar j = E_unit.factor_to_some_unit("J"); // j = 4.1868
Scalar strange = E_unit.factor_to_some_unit("g.m2/s2"); // strange = 4.1868e3
Scalar weird = E_unit.factor_to_some_unit("Da.ang2.min-2"); // weird = 9.07686e50
Scalar error = E_unit.factor_to_some_unit("Da.ang.min-2"); 
/*
terminate called after throwing an instance of 'Antioch::UnitError'
  what():  Units are not homogeneous:
"cal" and "Da.ang.min-2".
*/ 
\end{cpp}

\subsection{Combining units}

The \Units\ object enables to combine unit
as they would be combined by an equation.
\begin{itemize}
\item A multiplication is a unit addition, 
        $A \cdot B \Rightarrow \unit{U_{A\cdot B}} = \unit{U_A} + \unit{U_B}$;
\item a division       is a unit substraction, 
        $\frac{A}{B} \Rightarrow \unit{U_{\frac{A}{B}}} = \unit{U_A} - \unit{U_B}$;
\item an addition      is a unit homogeneity test, 
        $A + B \Rightarrow \unit{U_A} \text{ is homogeneous to } \unit{U_B}$;
\item a substraction   is a unit homogeneity test, 
        $A - B \Rightarrow \unit{U_A} \text{ is homogeneous to } \unit{U_B}$.
\end{itemize}
Mathematical functions have specifics requirements:
\begin{itemize}
\item $\log$, $\exp$, $\mathrm{pow}$, $\arccos$, $\arcsin$, $\arctan$ require unitless parameters;
\item $\cos$, $\sin$, $\tan$ require angled-united parameters;
\item \mbox{\vdots}
\end{itemize}

The \Units\ object supports all these operations.
It is therefore easy to define an object to automatically get the unit
from an equation. One such object is given in appendix~\ref{unit_goodie},
some capabilities are given here.


\begin{minipage}{0.55\linewidth}
\begin{cpp|} 
// 1 gram 
// light celerity
// Energy contained in 1 gram
Antioch::PhysicalQuantity<double> 
        m(1.,"g"),
        c(299792458.,"m.s-1"),
        E = m * c * c; 

std::cout << E.value() << " " 
          << E.unit() << std::endl;

E.change_unit_to("J");
std::cout << E.value() << " " 
          << E.unit() << std::endl;

E.change_unit_to("cal");
std::cout << E.value() << " " 
          << E.unit() << std::endl;

E.change_unit_to_SI();
std::cout << E.value() << " " 
          << E.unit() << std::endl;
\end{cpp|}
\end{minipage}
\terminal{%
8.987552e+16 g.m.s-1.m.s-1\\
8.987552e+13 J\\
2.146640e+13 cal \\
8.987552e+13 kg.m2.s-2%
}

\subsection{Restrictions}
No partial power! The physical senses of it is questionnable
anyway, \Antioch\ will not let you define in any way
a partial power.


\part{\Antioch\ from the developper's point of view}
\chapter{Understanding the physics}
The scientific goal is to model a physical phenomenom. This
phenomenom can be affected by several physical quantities, which
in turn are usually modeled using a model chosen within several
existing models.

Therefore a natural hierarchy emerges:
a phenomenom (\ie{transport}) requires physical quantities
(\ie{diffusion, viscosity, thermal conductivity}) which are
represented by models (\ie{Eucken viscosity, kinetics theory}).

The code structure respects this hierarchy, for each
phenomenom, there is a \phenomenom\ folder (\ie{\file{src/transport}}),
containing the files representing the mixture: 
\file{\phenomenom\_mixture.h},
\file{\phenomenom\_set.h},
\file{\phenomenom\_evaluator.h}.
The choice between \file{mixture} and/or \file{set}
file is up to the developper, as the phenomenom characterized
may be better represented by a mixture or a set. The
\file{evaluator} is mandatory, a lightweight user-friendly object
to be spawned all over the cluster.

The quantities modeled are stored in \quantity\ folder (\ie{\file{src/viscosity}}):
\file{\model\_\quantity.h}, derived from
\file{\quantity.h}, sorted with
\file{\quantity\_enum.h} and constructed with
\file{\quantity\_parsing.h}, stored in
\file{\quantity\_set.h}.

And in the \file{src/parsing} folder
\file{\model\_ascii\_parsing.h} that fills a \object{\Quantity Set}


\begin{tabular}{ll}\toprule
file                             & example                                   \\\midrule
\phenomenom                      & transport                                 \\
folder                           & \file{src/transport/include/antioch}      \\
\file{\phenomenom\_mixture.h}    & -                                         \\
\file{\phenomenom\_set.h}        & \file{transport\_set.h}                   \\
\file{\phenomenom\_evaluator.h}  & \file{transport\_evaluator.h}             \\[10pt]
\quantity                        & viscosity                                 \\
folder                           & \file{src/viscosity}                      \\
\file{\quantity.h}               & \file{viscosity.h}                        \\
\file{\quantity\_enum.h}         & \file{viscosity\_enum.h}                  \\
\file{\quantity\_parsing.h}      & \file{viscosity\_parsing.h}               \\
\file{\quantity\_set.h}          & \file{viscosity\_set.h}                   \\[10pt]
\multirow{3}{*}{\model}
                                 & Blottner                                  \\ 
                                 & Sutherland                                \\ 
                                 & kinetics theory                           \\\cmidrule(r@{2cm}l@{1cm}){2-2}
\multirow{3}{*}{\file{\model\_\quantity.h}}
                                 & \file{blottner\_viscosity.h}              \\ 
                                 & \file{sutherland\_viscosity.h}            \\ 
                                 & \file{kinetics\_theory\_viscosity.h}      \\\cmidrule(r@{2cm}l@{1cm}){2-2}
\multirow{3}{*}{\file{\model\_ascii\_parsing.h}}
                                 & \file{blottner\_ascii\_parsing.h}         \\ 
                                 & \file{sutherland\_ascii\_parsing.h}       \\ 
                                 & \file{kinetics\_theory\_ascii\_parsing.h} \\\bottomrule
\end{tabular}


\chapter{Global code layout}
\graphAtPaperCenter{code_graph}
The global layout can be understood as different
parts working together:
\begin{itemize}
\item the core part;
\item the thermodynamics part;
\item the chemistry part composed of;
        \begin{itemize}
        \item the kinetics model part;
        \item the chemical process part.
        \end{itemize}
\end{itemize}

%templ/.style={rounded corners,blue,-stealth},
%point/.style={rounded corners,violet,-stealth,double}
%The relationships between the objects are as follow:
\begin{itemize}
\item \tikz[baseline=(a.base)]\draw[red,-stealth,text=black] (0,0)node(a)[left]{A} -- (1,0) node[right]{B};: 
                        B contains a reference to A;
\item \tikz[baseline=(a.base)]\draw[stealth-,densely dotted,text=black] (0,0)node(a)[left]{A} -- (1,0) node[right]{B};: 
                        A derives from B;
\item \tikz[baseline=(a.base)]\draw[blue,-stealth,text=black] (0,0)node(a)[left]{A} -- (1,0) node[right]{B};: 
                        B is templated around objects of type A;
\item \tikz[baseline=(a.base)]\draw[violet,double,-stealth,text=black] (0,0)node(a)[left]{A} -- (1,0) node[right]{B};: 
                        B contains a/several pointers of type A;
\item \tikz[baseline=(a.base)]\draw[violet,double,-stealth,text=black] (0,0)node(a)[left]{A} -- (1,0) node[right]{B};: 
                        B contains a/several pointers of type A;
\item \tikz[baseline=(a.base)]\draw[)-,text=black] (0,0)node(a)[left]{A} -- (1,0) node[right]{B};: 
                        B contains an instanciation of A;
\end{itemize}


\chapter{Core}
\graphAtPaperCenter{core_graph}

\chapter{Thermodynamics}
\graphAtPaperCenter{thermo_graph}

\chapter{Chemistry}
\graphAtPaperCenter{chemistry_graph}
\section{The particle flux}

A particle flux is the quantity of particles through
a surface per second (\unit{particle\,cm^{-2}\,s^{-1}}). This
value depends on some characteristics of the considered
particles, usually their energy. For instance, a photochemical
reaction will be expressed with respect to a certain photons
flux, \textit{i.e.} a photon spectrum. Such a spectrum is
expressed in terms of particles flux per energy. Thus the
``classic'' units you'll find for this will be
  \unit{photon\,cm^{-2}\,s^{-1}} or \unit{W\,m^{-2}} 
for the photon flux, and
  \unit{eV} or \unit{nm}
for the energy. 

Now typically, you have instrument resolution here, and you
measure those thing per windows of resolution, per bin. Thus,
your measured flux is per unit of energy (or equivalent). Thus,
informations on these quantities are typically of the form
(\unit{photon\,cm^{-2}\,s^{-1}\,nm^{-1}}~,~\unit{nm}).

The rate constant is calculated by integration of the
cross-section with the incoming particle flux 
(Tab.~\ref{antioch::kinMod}), thus a rescaling
is due. \Antioch\ rescale cross-sections to given
flux.

The principle is simple, let's put some units first. We
consider a photon flux in \unit{photon\,cm^{-2}\,s^{-1}\,nm}
on a grid of \unit{nm}. We have a cross-section on \unit{cm^2\,\AA}
on a grid of \unit{\AA}. The principle will be to rescale
the cross-section to obtain the wanted units, then to calculate
the integrated value on each photon flux bin, and then deduced
the corresponding value of the cross-section on each photon
flux bin. Fig.~\ref{dev:particle_flux_rescaling} pictures
this protocol.

\begin{figure}
\centering
\includegraphics{particle_flux}
\caption{\label{dev:particle_flux_rescaling}The rescaling of the cross section (blue) on
a new grid (black) requires to calculate the surface (dotted/stripped areas) to rescale
on this new grid (red).}
\end{figure}


\chapter{Parsing}

\appendix
\part{Appendixes}
\chapter{Demonstrations}
\label{demo}
\subsection{\texorpdfstring{$\frac{1}{\Mm_\text{mix}} = \sum_s \frac{y_s}{\Mm_s}$}{Mixture molar mass}}
\label{demo-Mm}
\[
\begin{split}
\Mm_\text{mix} & = \frac{m_\text{mix}}{c_\text{mix}}
                 = \frac{m_\text{mix}}{\sum_s c_s}
                 = \frac{m_\text{mix}}{\sum_s \frac{m_s}{\Mm_s}}\\
\Rightarrow
\frac{1}{\Mm_\text{mix}}
               & = \frac{\sum_s\frac{m_s}{\Mm_s}}{m_\text{mix}}
                 = \sum_s\frac{\frac{m_s}{\Mm_s}}{m_\text{mix}}
                 = \sum_s\frac{\frac{m_s}{m_\text{mix}}}{\Mm_s}
                 = \sum_s\frac{y_s}{\Mm_s}
\end{split}
\]

\subsection{Equilibrium--inverse rate constant}

for the reaction:
\begin{chemicalEquation}
\ce{\scoefabs[A] A + \scoefabs[B] B <=>[k_1(T,p)][k_{-1}(T,p)] \scoefabs[C] C + \scoefabs[D] D}
\label{eq:equat}
\end{chemicalEquation}

We consider this equilibrium being composed of elemental steps
(%
\ce{\scoefabs[A] A + \scoefabs[B] B -> \scoefabs[C] C + \scoefabs[D] D} and
(\ce{\scoefabs[C] C + \scoefabs[D] D -> \scoefabs[A] A + \scoefabs[B] B}
are elementary processes). 

\subsubsection{Thermo}
\label{demo-eq}

By definition, the equilibrium constant will be
\begin{equation}
K = \prod_s \conc[S]_{\eq}^{\scoef[s]}
\label{eq:Kdef}
\end{equation}
$\conc[S]_{\eq}$ being the concentration of species \ce{S} at equilibrium.
For ideal gas: $p = \conc\Rg T \Rightarrow \conc = \frac{p}{\Rg T}$, with \Rg\ the ideal
gas constant and \conc\ the concentration. Thus
\begin{equation}
\begin{split}
K & = \prod_s \left(\frac{p^{(\eq)}_s}{\Rg T}\right)^{\scoef[s]} \\
  & = \prod_s \left(\frac{p^{(\eq)}_s}{\pz}\frac{\pz}{\Rg T}\right)^{\scoef[s]} \\
  & = \left(\frac{\pz}{\Rg T}\right)^{\sum_s\scoef[s]} \prod_s \left(\frac{p^{(\eq)}_s}{\pz}\right)^{\scoef[s]} \label{eq:K}
\end{split}
\end{equation}
Also, for ideal gas:
\begin{equation}
g(T) = g^0(T) + \Rg T \ln\left(\frac{p}{\pz}\right)
\end{equation}
wich gives
\begin{equation}
\frac{p}{\pz} = \exp\left(\frac{g(T) - g^0(T)}{\Rg T}\right)
\end{equation}
replacing in \ref{eq:K}
\begin{equation}
\begin{split}
K & = \left(\frac{\pz}{\Rg T}\right)^{\sum_s\scoef[s]} \prod_s \exp\left(\scoef[s] \frac{g(T)_s^{(\eq)} - g^0(T)}{\Rg T}\right) \\
  & = \left(\frac{\pz}{\Rg T}\right)^{\sum_s\scoef[s]} \exp\left(\frac{\Delta_rG(T)^{(\eq)}}{\Rg T} - \frac{\Delta_rG^0(T)}{\Rg T}\right) \\
\end{split}
\end{equation}
and, by definition, at equilibrium
\begin{equation}
\Delta_rG(T) = 0
\end{equation}
we obtain, noting $\gamma = \sum_s \scoef[s]$,
\begin{equation}
 K = \left(\frac{\pz}{\Rg T}\right)^\gamma \exp\left(-\frac{\Delta_rG^0(T)}{\Rg T}\right) \\
\end{equation}

\subsubsection{Kinetics}
\label{demo-eqrate}

By definition: $\frac{\dd \conc}{\dd t} = 0$ for any species concerned by the equilibrium.

Thus, for any species \ce{S}: 
$\frac{1}{\scoef[S]}\frac{\dd\conc[S]}{\dd t} =  0$, which can be rewritten as
\begin{equation}
\fwdrate \prodReac = \bkwdrate \prodProd
\label{eq-conccond}
\end{equation}
Developping \ref{eq:Kdef}
\begin{equation}
\Eqconst = \prod_{s}\conc[S]^{\scoef[S]} 
          = \frac{\displaystyle\prodProd}
                 {\displaystyle\prodReac}
\label{eq-defK}
\end{equation}
We deduce, from \ref{eq-conccond} and \ref{eq-defK}
\begin{equation}
\Eqconst = \frac{\fwdrate}{\bkwdrate}
\end{equation}


\chapter{Program flow}
\label{progflow}

\section{Kinetics}
\label{progflow:kinetics}
Figure~\ref{kinpf} describe the data flow from the upper
object \KineticsEvaluator\ to the rate constants objects.
The idea here is to have a light high-level object that will
be fast to build, so easily threadable. All the necessary data
and the bulk of the work is in lower-level objects.
%
\begin{figure}
\centering
\includegraphics{kinetics_relationships}
\caption{\label{kinpf}Global vision of the kinetics in \Antioch.
The gray area frames objects that should not be threaded, whereas
\KineticsEvaluator\ is light and thread-safe. The passed values of
interest are the temperature \prog{const StateType \& T}, the
particle flux \prog{const ParticleFlux<VectorCoeffType> \& pf} and
the composition \prog{const VectorStateType \& dens}.}
\end{figure}
%
\subsection{A classical case}

Let's consider a simple bimolecular reaction, elementary,
with a Kooij kinetics model. The informations we need for
a complete kinetics calculation are the temperature, the
composition and the thermodynamics. The thermodynamics are
calculated in another place of the program, what \Antioch\
will provide will be the Gibbs energy for all the concerned
species (see section~\ref{kinetics_gen}, \textbf{Going backward}
subsection). So at the \KineticsEvaluator\ level, we need
to provide \Antioch's guts with:
\begin{itemize}
\item a \prog{const StateType \& T} for the temperature,
\item a \prog{const VectorStateType \& molar\_densities} for
                the densities,
\item a \prog{const VectorStateType \& h\_RT\_minus\_s\_R} for
                the thermodynamics.
\end{itemize}

The goal is to have the quantity \doverdt{S} for all species \ce{S} concerned by the 
reactive system (see \ref{phys:kinetics_theory_integrated}).

\subsubsection{Data flow}

\paragraph{\KineticsEvaluator}
will gain from \ReactionSet\ the net rates of
the reactions,  i.e. the term \rate[r] in \ref{phys:kinetics_theory_integrated}.
Thus \KineticsEvaluator\ will, for every species, perform the loop
over the concerned reactions and multiply by the stoichiometric coefficient.
Technically, \KineticsEvaluator\ gives \ReactionSet\ a \prog{VectorCoeffType}
to fill with these terms.

\paragraph{\ReactionSet}
will equate the values of the provided \prog{VectorCoeffType} with
the returning value given by the \Reaction. To do that, it passes down to
the \Reaction\ object the temperature, the composition, the thermodynamics and
the value $\frac{\pz}{\Rg \Temp}$, used for the equilibrium constant calculation,
needed for the backward rate constant \eqref{therm:K_therm}.

\paragraph{\Reaction} will asks the \KineticsType\ object (its rate constants) to calculate
the forward rate constant of the reaction (term \fwdratecons[r]) by passing
it the temperature. Then it will be able 
to calculate the backward rate constant (term \bkwdratecons[r]), 
the forward (term \fwdrate[r]) and backward rate (term \bkwdrate[r]), 
and finally the net rate (term \rate[r]).

\paragraph{\KineticsType} will return the forward rate constant \fwdratecons[r] given
the temperature.

\subsubsection{Template flow}

\paragraph{\KineticsEvaluator} asks for a \prog{VectorCoeffType} to be filled. It
provides a \prog{StateType} for the temperature and a \prog{VectorStateType} for
the densities.

\paragraph{\ReactionSet} fills each value of the \prog{VectorCoeffType} with
the returning type of \Reaction.

\paragraph{\Reaction} returns a \prog{StateType}, which is the defined type of the
provided temperature.

\paragraph{\KineticsType} returns a \prog{StateType}, defined by the type of the
provided temperature.

\subsection{A particle flux reaction}

A reaction occuring with a particle flux is another type of reaction than the one
previously considered. Basically, instead of having molecules bumping into each
other and joyfully transforming into other molecules, this kind of reaction is
a molecule shot by particles, and transforming into another molecule(s). A simple
example is photoinduced ionization, a photon will hit a molecule, be absorbed, and
the excedental energy will pop out an electron, thus describing the reaction
\ce{A ->[\mathrm{h}\nu] A+}. In the classical view, this means that the rate constant
is constant, as it does not depend on the temperature. \Antioch\ supports at this
moment only photochemical reaction, but what will be describe here holds for any
particle flux-induced reaction.

The only change so is that instead of passing down a temperature, \KineticsEvaluator,
\ReactionSet\ and \Reaction\ need to pass down a particle flux (\ParticleFlux). The
big difference is the type of the data. A temperature is an atomic type (typically
a \prog{double} or a \prog{float}, while the \ParticleFlux\ is a \ParticleFlux, and
contains a vector type (the full description is \ParticleFlux\prog{<VectorCoeffType>}),
as is needs to store a flux intensity and a grid (wavelength, energy, \dots) corresponding
to those intensity. A photon flux will be characterized by an intensity per wavelength.

Thus the program flow stays the same, just replace the temperature with the particle
flux. Where it gets tricky is at the \Reaction\ level, where it returns the net reaction
rate value. Two things to consider here: a reaction occuring through particle flux
chemistry is irreversible in nature, so no backward rate consideration (nor thermodynamics),
we need to extract a \prog{StateType} out of a \ParticleFlux\prog{<VectorCoeffType>}.
The concerned method is (templates left out)
\begin{center}
\verb!StateType Reaction::compute_rate_of_progress() const!
\end{center}
with \prog{StateType} provided by the passed temperature. In this case however, we but
need the molecular densities (\prog{VectorStateType}) and the \ParticleFlux. So the
question is, how do we get the \prog{StateType} type?

Several possibilities here:
\begin{enumerate}
\item\label{type_pass} we pass down the temperature anyway, just for the type;
\item\label{meta_paradise} heck man, meta computing is so great, let's just refactore those method!
\end{enumerate}

The method \ref{type_pass} is simple, we would need to overload the methods:
\begin{verbatim}
  template<typename CoeffType, typename VectorCoeffType>
  template <typename StateType, typename VectorStateType>
  inline
  StateType Reaction<CoeffType,VectorCoeffType>::compute_rate_of_progress( 
                const VectorStateType& molar_densities,
                const StateType& T,  
                const StateType& P0_RT,  
                const VectorStateType& h_RT_minus_s_R) const;
\end{verbatim}
with
\begin{verbatim}
  template<typename CoeffType, typename VectorCoeffType>
  template <typename StateType, typename VectorStateType>
  inline
  StateType Reaction<CoeffType,VectorCoeffType>::compute_rate_of_progress( 
                const VectorStateType& molar_densities,
                const ParticleFlux<VectorStateType>& pf,
                const StateType & /*T*/) const;
                
\end{verbatim}
and so forth down to \KineticsType.

Method \ref{meta_paradise} would require a little bit more perversion, but no
overloading. Basically the idea would be to have the definition
\begin{verbatim}
  template<typename CoeffType, typename VectorCoeffType>
  template <typename DataType, typename VectorStateType, typename StateType>
  inline
  typedef typename Antioch::returned_type<DataType>::type
        Reaction<CoeffType,VectorCoeffType>::compute_rate_of_progress( 
                const VectorStateType& molar_densities,
                const DataType& T,  
                const StateType& P0_RT,  
                const VectorStateType& h_RT_minus_s_R) const;
\end{verbatim}
with the meta function
\begin{verbatim}
  template <typename T, typename Enable=void>
  struct returned_type;
\end{verbatim}
that would be defined as
\begin{verbatim}
template <typename T>
struct returned_type<T>
{
  typedef T type;
};
\end{verbatim}
for all there is (eigen, metaphysicl, valarray, vector, vexcl, ANTIOCH\_PLAIN\_SCALAR), and
as
\begin{verbatim}
template <typename T>
struct returned_type<ParticleFlux<T> >
{
  typedef typename Antioch::value_type<T>::type type;
};
\end{verbatim}
for the particle flux.


\chapter{Units}
\section{List of units supported by \Antioch}
\label{list_units}
\begin{longtable}{cccc}\toprule
Unit symbol & Unit name & Factor & Translator \\\midrule
\endhead
\multicolumn{4}{l}{\textsf{SI units}} \\\cmidrule(rl){1-2}
m   & meter    & $1$ & $0$ \\
kg  & kilogram & $1$ & $0$ \\
s   & second   & $1$ & $0$ \\
A   & ampere   & $1$ & $0$ \\
K   & kelvin   & $1$ & $0$ \\
mol & mole     & $1$ & $0$ \\
cd  & candela  & $1$ & $0$ \\
rad & radian   & $1$ & $0$ \\[5pt]
%%%
\multicolumn{4}{l}{\textsf{Length}} \\\cmidrule(rl){1-2}
in  &  inch              & $\numprint{0.0254}$                 & $0$ \\
ft  &  foot              & $\numprint{0.3048}$                 & $0$ \\
ua  &  astronomical unit & $\numprint{1.49597870700}\,10^{11}$ & $0$ \\
ang &  angstrom          & $1\,10^{-10}$                       & $0$ \\[5pt]
%%%
\multicolumn{4}{l}{\textsf{Mass}} \\\cmidrule(rl){1-2}
Da  &  dalton               & $\numprint{1.660538921}\,10^{-27}$ & $0$ \\
u   &  unified atomic mass  & $\numprint{1.660538921}\,10^{-27}$ & $0$ \\[5pt]
%%%
\multicolumn{4}{l}{\textsf{Time}} \\\cmidrule(rl){1-2}
min   &  minute & $\numprint{60}$   & $0$ \\
hour  &  hour   & $\numprint{3600}$ & $0$ \\[5pt]
%%%
\multicolumn{4}{l}{\textsf{Temperature}} \\\cmidrule(rl){1-2}
degC  &  degree Celsius   & $1$           & $\numprint{273.15}$ \\
degF  &  degree Farenheit & $\frac{5}{9}$ & $\numprint{459.57} \frac{5}{9}$ \\[5pt]
%%%
\multicolumn{4}{l}{\textsf{Angle}} \\\cmidrule(rl){1-2}
deg & degree    &  $\frac{\pi}{180}$                   & $0$ \\
'   & Arcminute &  $\frac{\pi}{180 \cdot 60}$          & $0$ \\
''  & Arcsecond &  $\frac{\pi}{180 \cdot 60 \cdot 60}$ & $0$ \\[5pt]
%%%
\multicolumn{4}{l}{\textsf{Volume} (\unit{m^3})} \\\cmidrule(rl){1-2}
l  &  litre  & $1\,10^{-3}$ & $0$ \\
L  &  litre  & $1\,10^{-3}$ & $0$ \\[5pt]
%%%
\multicolumn{4}{l}{Force (\unit{m\,kg\,s^{-2}})} \\\cmidrule(rl){1-2}
N   & newton & $1$          & $0$ \\
dyn & dyne   & $1\,10^{-5}$ & $0$ \\[5pt]
%%%
\multicolumn{4}{l}{\textsf{Pressure} (\unit{m^{-1}\,kg\,s^{-2}})} \\\cmidrule(rl){1-2}
Pa   & pascal                 & $1$                                          & $0$ \\
bar  & bar                    & $1\,10^5$                                    & $0$ \\
at   & technical atmosphere   & $\numprint{9.80665}\,10^4$                   & $0$ \\
atm  & atmosphere             & $\numprint{1.01325}\,10^5$                   & $0$ \\
Torr & Torr                   & $\frac{\numprint{101325.0}}{\numprint{760}}$ & $0$ \\
psi  & pound per square inch  & $\numprint{6.895}\,10^3$                     & $0$ \\
mmHg & millimeter of mercury  & $\numprint{133.322387415}$                   & $0$ \\[5pt]
%%%
\multicolumn{4}{l}{\textsf{Viscosity} (\unit{m^{-1}\,kg\,s^{-1}})} \\\cmidrule(rl){1-2}
P & poise & $\numprint{0.1}$ & $0$ \\[5pt]
%%%
\multicolumn{4}{l}{\textsf{Power} (\unit{m^{2}\,kg\,s^{-3}})} \\\cmidrule(rl){1-2}
W & watt & $1$ & $0$ \\[5pt]
%%%
\multicolumn{4}{l}{\textsf{Energy} (\unit{m^{2}\,kg\,s^{-2}})} \\\cmidrule(rl){1-2}
J     & joule                 & $1$                                & $0$ \\
cal   & calorie               & $\numprint{4.1868}$                & $0$ \\
calth & calorie thermodynamic & $\numprint{4.184}$                 & $0$ \\
eV    & electronVolt          & $\numprint{1.602176565}\,10^{-19}$ & $0$ \\
erg   & erg                   & $1\,10^{-7}$                       & $0$ \\
Ha    & hartree               & $\numprint{4.35974434}\,10^{-18}$  & $0$ \\[5pt]
%%%
\multicolumn{4}{l}{\textsf{Electric charge} (\unit{A\,s})} \\\cmidrule(rl){1-2}
C & coulomb & $1$ & $0$ \\[5pt]
%%%
\multicolumn{4}{l}{\textsf{Frequency} (\unit{s^{-1}})} \\\cmidrule(rl){1-2}
Hz  &  herzt     & $1$                       & $0$ \\
Ci  &  curie     & $\numprint{3.7}\,10^{10}$ & $0$ \\
Bq  &  becquerel & $1$                       & $0$ \\[5pt]
%%%
\multicolumn{4}{l}{\textsf{No unit}} \\\cmidrule(rl){1-2}
molecule & molecule & $1$ & $0$ \\
photon   & photon   & $1$ & $0$ \\
\bottomrule
\end{longtable}


\section{Goodie code}
\label{unit_goodie}
{\footnotesize
\begin{verbatim}

namespace Antioch{

  template <typename CoeffType>
  class PhysicalQuantity{

    public:
        PhysicalQuantity();
        PhysicalQuantity(const CoeffType & value, const std::string & unit = std::string() );

        ~PhysicalQuantity();

        /* getter/setter */

        void set_value(const CoeffType & value);

        void set_unit(const std::string & unit);

        const CoeffType value() const;

        const Units<CoeffType> & unit() const;

        /* operators */

        PhysicalQuantity<CoeffType> & operator+=(const PhysicalQuantity<CoeffType> &rhs);

        PhysicalQuantity<CoeffType> & operator-=(const PhysicalQuantity<CoeffType> &rhs);

        PhysicalQuantity<CoeffType> & operator*=(const PhysicalQuantity<CoeffType> &rhs);

        PhysicalQuantity<CoeffType> & operator/=(const PhysicalQuantity<CoeffType> &rhs);

        PhysicalQuantity<CoeffType>   operator* (const PhysicalQuantity<CoeffType> &rhs) const;

        PhysicalQuantity<CoeffType>   operator/ (const PhysicalQuantity<CoeffType> &rhs) const;

        PhysicalQuantity<CoeffType>   operator+ (const PhysicalQuantity<CoeffType> &rhs) const;

        PhysicalQuantity<CoeffType>   operator- (const PhysicalQuantity<CoeffType> &rhs) const;

    private:
        Units<CoeffType> _unit;
        CoeffType        _value;

  }

  template <typename CoeffType>
  PhysicalQuantity<CoeffType>::PhysicalQuantity():
        _unit(),
        _value(0.)
  {
     return;
  }

  template <typename CoeffType>
  PhysicalQuantity<CoeffType>::PhysicalQuantity(const std::string & unit, const CoeffType & value):
        _unit(unit),
        _value(value)
  {
     return;
  }

  template <typename CoeffType>
  void PhysicalQuantity<CoeffType>::set_value(const CoeffType & value)
  {
     _value = value;
  }

  template <typename CoeffType>
  void PhysicalQuantity<CoeffType>::set_unit(const std::string & unit)
  {
     _unit = unit;
  }

  template <typename CoeffType>
  const CoeffType PhysicalQuantity<CoeffType>::value() const
  {
     return _value;
  }

  template <typename CoeffType>
  const Units<CoeffType> & PhysicalQuantity<CoeffType>::unit() const
  {
     return _unit;
  }

  template <typename CoeffType>
  PhysicalQuantity<CoeffType> & 
        PhysicalQuantity<CoeffType>::operator+=(const PhysicalQuantity<CoeffType> &rhs)
  {
     if(!rhs.unit().is_homogeneous(this->unit()))antioch_error();
     this->set_value(this->value() + rhs.value());
     return *this;
  }

  template <typename CoeffType>
  PhysicalQuantity<CoeffType> & 
        PhysicalQuantity<CoeffType>::operator-=(const PhysicalQuantity<CoeffType> &rhs)
  {
     if(!rhs.unit().is_homogeneous(this->unit()))antioch_error();
     this->set_value(this->value() - rhs.value());
     return *this;
  }

  template <typename CoeffType>
  PhysicalQuantity<CoeffType> & 
        PhysicalQuantity<CoeffType>::operator*=(const PhysicalQuantity<CoeffType> &rhs)
  {
     this->set_value(this->value() * rhs.value());
     this->set_unit(this->unit() + rhs.unit());
     return *this;
  }

  template <typename CoeffType>
  PhysicalQuantity<CoeffType> & 
        PhysicalQuantity<CoeffType>::operator/=(const PhysicalQuantity<CoeffType> &rhs)
  {
     this->set_value(this->value() / rhs.value());
     this->set_unit(this->unit() - rhs.unit());
     return *this;
  }

  template <typename CoeffType>
  PhysicalQuantity<CoeffType> & 
        PhysicalQuantity<CoeffType>::operator*(const PhysicalQuantity<CoeffType> &rhs)
  {
     return (PhysicalQuantity<CoeffType>(*this) *= rhs);
  }

  template <typename CoeffType>
  PhysicalQuantity<CoeffType> & 
        PhysicalQuantity<CoeffType>::operator/(const PhysicalQuantity<CoeffType> &rhs)
  {
     return (PhysicalQuantity<CoeffType>(*this) /= rhs);
  }

  template <typename CoeffType>
  PhysicalQuantity<CoeffType> & 
        PhysicalQuantity<CoeffType>::operator+(const PhysicalQuantity<CoeffType> &rhs)
  {
     return (PhysicalQuantity<CoeffType>(*this) += rhs);
  }
  
  template <typename CoeffType>
  PhysicalQuantity<CoeffType> & 
        PhysicalQuantity<CoeffType>::operator-(const PhysicalQuantity<CoeffType> &rhs)
  {
     return (PhysicalQuantity<CoeffType>(*this) -= rhs);
  }

}
\end{verbatim}

}

\bibliography{ref}
\addcontentsline{toc}{part}{\bibname}
\bibliographystyle{unsrtnat}


\end{document}
