\documentclass{article}
\usepackage{booktabs,geometry,amsmath,multicol,multirow,array,graphicx}
\usepackage[autolanguage]{numprint}
\usepackage[version=3,arrows=pgf-filled]{mhchem}
\usepackage[linkcolor=black,colorlinks=true,citecolor=black,urlcolor=blue]{hyperref}
\makeatletter
%first thing first
\graphicspath{{./figs/}}
%%%% meta-meta-function
\newcommand{\make@}[3][upshape]{%
\expandafter\gdef\csname font@#2\endcsname##1{% create font@something macro
{\csname#1\endcsname\csname#3\endcsname{##1}}%
}%
\expandafter\gdef\csname make@#2\endcsname##1{% create make@something meta macro
\expandafter\gdef\csname ##1\endcsname{\csname font@#2\endcsname{##1}}%
}
\expandafter\gdef\csname#2\endcsname##1{\csname font@#2\endcsname{##1}}%create the \something macro
}
%% meta font
\newcommand{\prog@type}[1]{\textcolor{green!60!black}{\object{#1}}}
%% meta function
\make@{program}{sf}
\make@{library}{tt}
\make@[bf]{object}{tt}
\make@{file}{sf}
\make@{prog}{prog@type}
%% macros
\make@program{Doxygen}
\make@library{Antioch}
\make@library{antioch}
\make@library{Boost}
\make@library{VexCL}
\make@library{ViennaCL}
\make@library{MetaPhysicL}
\make@library{GRVY}
\make@library{EIGEN}
%% for tutorial
\make@object{ReactionSet}
\make@object{KineticsEvaluator}
\make@object{ChemicalMixture}
\make@object{CEAThermodynamics}
\make@object{Units}
\make@prog{double}
\newcommand{\stdvector}{\prog{std::vector< >}}
\newcommand{\stdstring}{\prog{std::string}}
%%physics
\newcommand{\phenomenom}{\textcolor{red}{phenomenom}}
\newcommand{\quantity}{\textcolor{blue!60!black}{quantity}}
\newcommand{\Quantity}{\textcolor{blue!60!black}{Quantity}}
\newcommand{\model}{\textcolor{green!60!black}{model}}
%

%%% versions
\newcommand{\version}[3]{%
\setcounter{vmajor}{#1}
\setcounter{vmedium}{#2}
\setcounter{vminor}{#3}
}
\newcounter{vmajor}
\newcounter{vmedium}
\newcounter{vminor}
\newcommand{\theversion}{\thevmajor.\thevmedium.\thevminor}

%%% constant equation management
\newcommand{\make@meta@equation}[7]
{
\expandafter\gdef\csname#2\endcsname{\ensuremath{#3}}%                                                          \"name"     produces "symbol"
\expandafter\gdef\csname#2val\endcsname{\csname#7\endcsname{#4}\ifnum#1=0\else\,10^{#1}\fi}%                              \"name"val  produces "value"
\expandafter\gdef\csname#2dval\endcsname{\csname#7\endcsname{#5}}%                                                        \"name"dval produces "dvalue"
\expandafter\gdef\csname#2num\endcsname{\csname#7\endcsname{#4}\ifnum#5=0\else\,(\csname#2dval\endcsname)\fi\ifnum#1=0\else\,10^{#1}\fi}%   \"name"num  produces "value (dvalue)"
\expandafter\gdef\csname#2unit\endcsname{\unit{#6}}%                                                            \"name"unit produces "unit"
\expandafter\gdef\csname#2Equation\endcsname{% %% equation
 \ensuremath{\csname#2\endcsname = \csname#2num\endcsname~\csname#2unit\endcsname}}%                            \"name"Equation produces "symbol = value (dvalue)~unit"
}
\newcommand{\make@equation}[6][0] %% name, symbol, value, dvalue, unit | symbol = value (dvalue) \unit{unit}
{%
 \make@meta@equation{#1}{#2}{#3}{#4}{#5}{#6}{numprint}
}
\newcommand{\make@litteralEquation}[4] %% name, symbol, expression, unit | symbol = expression \unit{unit}
{%
 \make@meta@equation{0}{#1}{#2}{#3}{0}{#4}{relax}
}

%% fun
\newcommand{\ANTIOCH}{\font@library{A}
                      \font@library{N}ew
                      \font@library{T}emplated
                      \font@library{I}mplementation
                      \font@library{O}f
                      \font@library{C}hemistry
                      \font@library{H}ydrodynamics}
\newcommand{\ANTIOCHPhys}{\Antioch's
                         \font@library{N}oble
                         \font@library{T}ry to
                         \font@library{I}mitate the
                         \font@library{O}rdered
                         \font@library{C}osmos
                         \font@library{H}umbly}
\newcommand{\ANTIOCHTech}{\Antioch's
                         \font@library{N}ot
                         \font@library{T}emplated
                         \font@library{I}rresponsibly, this
                         \font@library{O}utstanding
                         \font@library{C}ode is
                         \font@library{H}appiness}
\newcommand{\ANTIOCHPrac}{\Antioch\
                         \font@library{N}ewbies
                         \font@library{T}empting
                         \font@library{I}nvitation to
                         \font@library{O}pen this
                         \font@library{C}hapter with
                         \font@library{H}aste}
%% I like my equations this way
\renewcommand{\theequation}{Eq.~\thesection-\arabic{equation}}
\@addtoreset{equation}{section}
%%% beauty here
\pagestyle{fancy}
\renewcommand{\chaptermark}[1]{\markboth{#1}{}}
\renewcommand{\sectionmark}[1]{\markright{\thesection~\sl #1}}
\geometry{headheight=2cm}
\fancyhead{} %empty
\fancyhead[LO]{\parbox[b]{0.45\textwidth}{\raggedright\leftmark}}
\fancyhead[RE]{\parbox[b]{0.45\textwidth}{\raggedleft\leftmark}}
\fancyhead[LE]{\parbox[b]{0.45\textwidth}{\raggedright\rightmark}}
\fancyhead[RO]{\parbox[b]{0.45\textwidth}{\raggedleft\rightmark}}
%% a little hack
\newcommand{\spacehack}{\ensuremath{\textcolor{white}{=}}}

%%%%%%%%%%%%%%%%%%%%%%%%%%%%%%%%%%%%
%%%%
%%%%  stuffs start here
%%%%
%%%%%%%%%%%%%%%%%%%%%%%%%%%%%%%%%%%%

%layout stuff
\newlength{\tocenter}
\newcommand{\graphAtPaperCenter}[2][width=\paperwidth]
{
%%%% from geometry package
%% inner/outer margin ratio 2:3
%% => \parperwidth - \textwidth = 5
%% center = (inner +) 0.5\textwidth + (\parperwidth - \textwidth)/10 %% offset from inner (odd page)
%% center = (outer +) 0.5\textwidth - (\parperwidth - \textwidth)/10 %% offset from outer (even page)
%%%%%%%%%%%
\setlength{\tocenter}{\paperwidth}
\addtolength{\tocenter}{-\textwidth}
\ifodd\number\c@page% TeX register of LaTeX counter, late one page
  \setlength{\tocenter}{0.1\tocenter}
\else% even
  \setlength{\tocenter}{-0.1\tocenter}
\fi%
\addtolength{\tocenter}{0.5\textwidth}
\noindent\hspace{\tocenter}\makebox[0pt]{\includegraphics[#1]{#2}}\par
}

%% chapter
\@addtoreset{chapter}{part}

% unit defs
\newcommand{\unitbase}{[\unit{m},\unit{kg},\unit{s},\unit{A},\unit{K},\unit{mol},\unit{cd},\unit{rad}]}
%%%%%%%
%%
%% 'cause I'm lazy 
%%
%%%%
\newcommand{\example}[1]{\textit{i.e.} #1}


%internal stuff
\newcommand{\optional@sub} [2][\relax]{\ifx#1\relax\ensuremath{#2}\else\ensuremath{{{#2}_{#1}}}\fi}
\newcommand{\optional@@sub}[2][\relax]{\ifx#1\relax\ensuremath{#2}\else\ensuremath{{#2}_{#1}}\fi}
\newcommand{\d@d} [2]{\ensuremath{\frac{\partial#2}{\partial #1}}}
\newcommand{\dd@d}[2]{\ensuremath{\frac{\dd #2}{\dd #1}}}

%%%%%%%%%%%%%%%%%%%%%%%%%%%%%
%%
%% universal constant, using
%% meta macro make@equation and
%% meta macro make@litteralEquation
%%%%%%%%%%%%%%%%%%%%%%%%%%%%%%

%% valued constant
\make@equation     {Rg}       {\mathrm{R}}            {8.3144621} {75}{J\,mol^{-1}\,K^{-1}}
\make@equation[-23]{Boltzmann}{\mathrm{k}_\text{B}}   {1.3806488} {13}{J\,K^{-1}}
\make@equation[23] {Navo}     {\mathcal{N}_\text{Avo}}{6.02214129}{27}{mol^{-1}}
\make@equation[8]  {cele}     {\mathrm{c}}            {2.99792458}{0} {m\,s^{-1}}
%% computed from other constants constant
\make@litteralEquation{vacPerme}{\mu_0}     {4\pi\,10^{-7}}{m\,kg\,s^2\,A^2}
\make@litteralEquation{vacPermi}{\epsilon_0}{\frac{1}{\vacPerme\cele^2}}{A^2\,s^4\,kg^{-1}\,m^{-3}}

%%%%%%%%%%%%%%%%%%%%%%%%%%%%%%%%%%%%
%%
%% mathematical stuff
%%
%%%%%%%%%%%%%%%%%%%%%%%%%%%%%%%%%%%%
\newcommand{\doverdt}    [1]{\d@d{t}{#1}}
\newcommand{\doverdT}    [1]{\d@d{\Temp}{#1}}
\newcommand{\ddoverdT}   [1]{\dd@d{\Temp}{#1}}
\newcommand{\doverdm}    [2][\relax]{\d@d{\mass[#1]}{#2}}
\newcommand{\doverdc}    [2][\relax]{\d@d{\conc[#1]}{#2}}
\newcommand{\doverdn}    [2][\relax]{\d@d{\Mol[#1]}{#2}}
\newcommand{\doverdext}  [2][\relax]{\d@d{\ext[#1]}{#2}}
\newcommand{\ddoverddext}[3]{\ensuremath{\frac{\partial^2#1}{\partial\ext[#2]\partial\ext[#3]}}}
%%%%%%%%%%%%%%%%%%%%%%%%%%%%%%
%%
%%  chemical definitions
%%
%%%%%%%%%%%%%%%%%%%%%%%%%%%%%
\newcommand{\pz}            {\ensuremath{\mathrm{p^0}}}
\newcommand{\Tz}            {\ensuremath{\mathrm{T^0}}}
\newcommand{\eq}            {\ensuremath{\text{eq}}}
\newcommand{\Rmix}          {\ensuremath{\mathrm{R}_\text{mix}}}
\newcommand{\reac}          {\ce{R}}
\newcommand{\product}       {\ce{P}}
\newcommand{\tc}         [1]{\ensuremath{a_{#1}}}
%%stuff for kinetics
\newcommand{\rcons}   [1][\relax]{\optional@@sub[#1]{k}}
\newcommand{\kinMod}  [1][\relax]{\optional@@sub[#1]{\alpha}\ensuremath{(\Temp)}}
\newcommand{\kinModZ}            {\ensuremath{\alpha_0(\Temp)}}
\newcommand{\kinModI}            {\ensuremath{\alpha_\infty(\Temp)}}
\newcommand{\chemProc}[1][\relax]{\optional@@sub[#1]{\chi}\ensuremath{(\kinMod,\conc[M])}}
\newcommand{\rateCons}           {\ensuremath{\rcons(T,\conc[M])}}
\newcommand{\rate}    [1][\relax]{\optional@@sub[#1]{r}}
\newcommand{\threeBody}          {\ensuremath{\sum_s \epsilon_s\conc[S]}}
\newcommand{\FLind}              {\ensuremath{F_\text{Lind}}}
\newcommand{\FTroe}              {\ensuremath{F_\text{Troe}}}
%%%% kinetics model parameter
\newcommand{\Tref}        {\ensuremath{\mathrm{T_{ref}}}}
\newcommand{\PreExp}      {\ensuremath{A}}
\newcommand{\Power}       {\ensuremath{\beta}}
\newcommand{\BerthExp}    {\ensuremath{D}}
\newcommand{\AcEn}        {\ensuremath{{E_a}}}
\newcommand{\wavelength}  {\ensuremath{{\lambda}}}
\newcommand{\crosssection}{\ensuremath{{\sigma(\wavelength)}}}
%%%% Troe falloff
\newcommand{\Troealpha} {\ensuremath{\alpha}}
\newcommand{\TroeTone}  {\ensuremath{T^{*}}}
\newcommand{\TroeTtwo}  {\ensuremath{T^{**}}}
\newcommand{\TroeTthree}{\ensuremath{T^{***}}}
\newcommand{\TroeFcent} {\ensuremath{F_\text{cent}}}
\newcommand{\Troen}     {\ensuremath{n_\text{T}}}
\newcommand{\Troed}     {\ensuremath{d_\text{T}}}
\newcommand{\Troec}     {\ensuremath{c_\text{T}}}
%
\newcommand{\conc}        [1][\relax]{\ifx#1\relax\molar\else\ce{[#1]}\fi}
\newcommand{\Mm}          [1][\relax]{\optional@sub[#1]{\mathrm{M}}}
\newcommand{\scoefabs}    [1][\relax]{\optional@sub[#1]{\mathrm{n}}}
\newcommand{\scoef}       [1][\relax]{\optional@sub[#1]{\nu}}
\newcommand{\sumscoef}    [1][\relax]{\optional@sub[#1]{\gamma}}
\newcommand{\partialOrder}[1][\relax]{\optional@sub[#1]{\mathrm{m}}}
\newcommand{\orderReac}              {\ensuremath{{\mathrm{m}}}}
\newcommand{\massfrac}    [1][\relax]{\optional@sub[#1]{y}}
\newcommand{\mass}        [1][\relax]{\optional@sub[#1]{\rho}}
\newcommand{\mdot}        [1][\relax]{\optional@sub[#1]{\dot{\omega}}}
\newcommand{\molarfrac}   [1][\relax]{\optional@sub[#1]{x}}
\newcommand{\molar}       [1][\relax]{\optional@sub[#1]{c}}
\newcommand{\Eqconst}     [1][\relax]{\optional@sub[#1]{K}}
\newcommand{\fwdratecons} [1][\relax]{\ensuremath{\rcons[#1]^{(f)}}}
\newcommand{\bkwdratecons}[1][\relax]{\ensuremath{\rcons[#1]^{(b)}}}
\newcommand{\fwdrate}     [1][\relax]{\ensuremath{\rate[#1]^{(f)}}}
\newcommand{\bkwdrate}    [1][\relax]{\ensuremath{\rate[#1]^{(b)}}}
\newcommand{\ext}         [1][\relax]{\optional@sub[#1]{\xi}}
\newcommand{\nspecies}               {\ensuremath{\mathrm{n_{species}}}}
%%thermo
%%%% phase
\newcommand{\Vol}   [1][\relax]{\optional@sub[#1]{V}}
\newcommand{\Press} [1][\relax]{\optional@sub[#1]{P}}
\newcommand{\Temp}  [1][\relax]{\optional@sub[#1]{T}}
\newcommand{\RedTemp}          {\ensuremath{T_r}}
\newcommand{\Mol}   [1][\relax]{\optional@sub[#1]{N}}
\newcommand{\Gibbs} [1][\relax]{\optional@sub[#1]{G}}
\newcommand{\GibbsZ}[1][\relax]{\ensuremath{{\Gibbs[#1]}^{0}}}
\newcommand{\Enth}  [1][\relax]{\optional@sub[#1]{H}}
\newcommand{\Entr}  [1][\relax]{\optional@sub[#1]{S}}
\newcommand{\IntEn} [1][\relax]{\optional@sub[#1]{U}}
\newcommand{\Mass}  [1][\relax]{\optional@sub[#1]{m}}
%%%% species and Delta
\newcommand{\press}   [1][\relax]{\optional@sub[#1]{p}}
\newcommand{\chempot} [1][\relax]{\optional@sub[#1]{\mu}}
\newcommand{\chempotZ}[1][\relax]{\ensuremath{\chempot[#1]^{0}}}
\newcommand{\gibbs}   [1][\relax]{\optional@sub[#1]{g}}
\newcommand{\enth}    [1][\relax]{\optional@sub[#1]{h}}
\newcommand{\entr}    [1][\relax]{\optional@sub[#1]{s}}
\newcommand{\gibbsZ}  [1][\relax]{\ensuremath{{\gibbs[#1]}^{0}}}
\newcommand{\enthZ}   [1][\relax]{\ensuremath{{\enth[#1]}^{0}}}
\newcommand{\entrZ}   [1][\relax]{\ensuremath{{\entr[#1]}^{0}}}
\newcommand{\DGibbs}  [1][\relax]{\ensuremath{\optional@sub[#1]{\Delta}G}}
\newcommand{\DGibbsZ} [1][\relax]{\ensuremath{{\DGibbs[#1]}^{0}}}
\newcommand{\Denth}   [1][\relax]{\optional@sub[#1]{\Delta}\ensuremath{H}}
\newcommand{\DenthZ}  [1][\relax]{\ensuremath{{\Denth[#1]}^{0}}}
\newcommand{\Dentr}   [1][\relax]{\optional@sub[#1]{\Delta}\ensuremath{S}}
\newcommand{\DentrZ}  [1][\relax]{\ensuremath{{\Dentr[#1]}^{0}}}

%%%%%%%%%%%%%%%%%%%%%%%%%%%%%%%%%%%%%%%
%%
%% transport definitions
%%
%%%%%%%%%%%%%%%%%%%%%%%%%%%%%%%%%%%%%%%

\newcommand{\vis}           [1][\relax]{\optional@sub[#1]{\eta}}
\newcommand{\diff}          [1][\relax]{\optional@sub[#1]{D}}
\newcommand{\thermcond}     [1][\relax]{\optional@sub[#1]{\lambda}}
\newcommand{\LJdepth}       [1][\relax]{\optional@sub[#1]{\epsilon}}
\newcommand{\LJdia}         [1][\relax]{\optional@sub[#1]{\sigma}}
\newcommand{\Stockmayer}    [1]        {\ensuremath{\left<\Omega^{(#1,#1)*}\right>}}
\newcommand{\dipole}        [1][\relax]{\optional@sub[#1]{\mu}}
\newcommand{\polarizability}[1][\relax]{\optional@sub[#1]{\alpha}}
\newcommand{\specificHeat}  [1][\relax]{\optional@sub[#1]{C}}
\newcommand{\rotRelax}      [1][\relax]{\optional@sub[\text{rot}#1]{Z}}

%misc
\newcommand{\prodReac}    {\ensuremath{\prod_{\text{reactants}} \conc[\reac]^{\scoefabs[\reac]}}}
\newcommand{\prodReacMass}{\ensuremath{\prod_{\text{reactants}} \left(\frac{\mass[\reac]}{\Mm[\reac]}\right)^{\scoefabs[\reac]}}}
\newcommand{\prodProd}    {\ensuremath{\prod_\text{products}\conc[\product]^{\scoefabs[\product]}}}
\newcommand{\Kooij} [4][1]{\ensuremath{{#2} \left(\frac{T}{#1}\right)^{#3}\exp\left(-\frac{#4}{\Rg T}\right)}}
\newcommand{\KooijEq}     {\Kooij[\mathrm{T_{ref}}]{A}{\beta}{E_a}}
\newcommand{\dd}          {\ensuremath{\mathrm{d}}}
\newcommand{\unit}     [1]{\ensuremath{\mathtt{#1}}}
\newcommand{\nounit}      {\unit{no~unit}}
%%%%%
\newcommand{\reactionEq}[1]{%
\ce{#1}%
}
\newcommand{\ArrjPar}[4]{%
\texttt{Arrhenius} model\\*[2pt]
\null\hspace{12pt}$\left\{\begin{array}{l@{~=~}l}
\PreExp & #1~\unit{#2} \\
\AcEn   & #3~\unit{#4} \\
\end{array}\right.$%
}
\newcommand{\KooijPar}[5]{%
\texttt{Kooij} model\\*[2pt]
\null\hspace{12pt}$\left\{\begin{array}{l@{~=~}l}
\PreExp & #1~\unit{#2} \\
\Power  & #3           \\
\AcEn   & #4~\unit{#5} \\
\end{array}\right.$%
}
\newcommand{\TBcoeff}[2]{%
\ensuremath{\epsilon_{\ce{#1}} = #2}%
}
\newcommand{\reactionParameter}[3][\null]{%
\vspace{5pt minus 3pt}
\begin{minipage}{5.5cm}
\null\hfill#2\hfill\null\\[3pt]
#3\\[5pt]
#1%
\end{minipage}}
%% chemistry equation 
\newcounter{chemeq}
\setcounter{chemeq}{0}
\renewcommand{\thechemeq}{$\chi$~\thesection-\arabic{chemeq}}
\newenvironment{chemicalEquation}
{\begin{displaymath}}
{\refstepcounter{chemeq}\tag{\thechemeq}%
\end{displaymath}}
\@addtoreset{chemeq}{section}

%%%%%%%%%%%%%%%%%%%%%%%%%%%%%%%%
%%
%% code
%% code + output two column
%%
%%%%%%%%%%%%%%%%%%%%%%%%%%%%%%%%
%% language loading
\lstloadlanguages{C++,XML}
\lstset{basicstyle=\footnotesize,commentstyle=\color{blue}\bf,stringstyle=\color{magenta}\bf,
        numbers=left,numbersep=10pt,numberstyle=\scriptsize}

%% c++ code one column
\lstnewenvironment{cpp}[1][\footnotesize]
{\lstset{language=C++,
         morekeywords={Scalar,CoeffType,StateType},keywordstyle=\bf,
         morestring=[b]",
         morecomment=[s]{/*}{*/},morecomment=[l]{//},
         basicstyle=#1}
}
{}

%% c++ code two column
\lstnewenvironment{cpp|}
{\lstset{linewidth=0.9\linewidth,frame=r,framesep=-5pt,
         morekeywords={Scalar,CoeffType,StateType},keywordstyle=\bf,
         morestring=[b]",
         language=C++,morecomment=[s]{/*}{*/},morecomment=[l]{//},
         numbers=none}
}
{}

%% xml code one column
\lstnewenvironment{xml}[1][\footnotesize]
{\lstset{language=XML,
         morestring=[b]",
         basicstyle=#1}
}
{}

% output in terminal
\newlength{\outdepth}
\newcommand{\terminal}[1]{%
\settototalheight{\outdepth}{#1}
\ifdim\outdepth < 4em%
  \setlength{\outdepth}{5em}%
\fi%
\fcolorbox{white}{black}{%
\begin{minipage}[c][2\outdepth][c]{0.42\linewidth}
\color{white}
\textsf{[user@here \$]./program.x}\\
\textsf{#1}
\end{minipage}}
}

\makeatother
\begin{document}

\title{libantioch my love}
\date{version number to be printed here}
\author{author list}

\maketitle
\tableofcontents

\section{Intro}

\subsection{General stuff on libantioch}
\Antioch\ stands for \ANTIOCH. For your convenience,
this manual is separated in three chapters. In the
chapter~\ref{Antioch:physics}, the underlying physics
and chemistry are developed; the chapter~\ref{Antioch:technique}
is dedicated to the implementation details; the chapter~\ref{Antioch:practice}
is the \textsf{User's Guide: A Tutorial} part of the manual. Finally
the appendices brings extra useful stuff.


\subsection{Notations for this manual}
\begin{itemize}
\item \massfrac[i]:       mass fraction of species $i$, \nounit;
\item \mass[i]:           mass density of species $i$, homogeneous to \unit{g\,m^{-3}};
\item \Mass[i]:           mass species $i$, homogeneous to \unit{g};
\item \molarfrac[i]:      molar fraction of species $i$, \nounit;
\item \molar[i]/\conc[I]: molar density of species $i$ of name \ce{I}, homogeneous to \unit{mol\,m^{-3}};
\item \Mm[i]:             molar mass of species $i$, homogeneous to \unit{g\,mol^{-1}};
\item \fwdratecons[r]:    forward rate constant of reaction $r$, homogeneous to \unit{(m^3\,mol^{-1})^{\text{order} - 1}\,s^{-1}},
                            (\unit{m^3\,mol^{-1}\,s^{-1}} or \unit{s^{-1}} here, only first/zeroth order reaction);
\item \bkwdratecons[r]:   backward rate constant of reaction $r$, homogeneous to \unit{(m^3\,mol^{-1})^{\text{order} - 1}\,s^{-1}},
                            (\unit{m^3\,mol^{-1}\,s^{-1}} or \unit{s^{-1}} here, only first/zeroth order reaction);
\item \fwdrate[r]:        forward rate of reaction $r$, homogeneous to \unit{m^3\,mol^{-1}\,s^{-1}}; 
\item \bkwdrate[r]:       backward rate of reaction $r$, homogeneous to \unit{m^3\,mol^{-1}\,s^{-1}}; 
\item \Eqconst[r]:        equilibrium constant of reaction $r$, \nounit;
\item \gibbs[i]:          Gibbs free energy of species $i$, homogeneous to \unit{J\,mol^{-1}};
\item \DGibbs[r]:         Gibbs free energy of reaction $r$, homogeneous to \unit{J\,mol^{-1}};
\item \enth[i]:           enthalpy of species $i$, homogeneous to \unit{J\,mol^{-1}};
\item \Denth[r]:          enthalpy of reaction $r$, homogeneous to \unit{J\,mol^{-1}};
\item \entr[i]:           entropy of species $i$, homogeneous to \unit{J\,mol^{-1}};
\item \Dentr[r]:          entropy of reaction $r$, homogeneous to \unit{J\,mol^{-1}};
\item \chempot[i]:        chemical potential of species $i$, homogeneous to \unit{J};
\item \scoef[i,r]:        stoichiometric coefficient of species $i$ in reaction $r$, \nounit;
\item \scoefabs[i,r]:     absolute value of stoichiometric coefficient of species $i$ in reaction $r$, \nounit, this is the number
                            you see in chemical equation;
\item \sumscoef[r]:       sum of the stoichiometric coefficients for reaction $r$, \nounit;
\item \Temp[i]:           temperature of thermodynamic  phase $i$ homogeneous to \unit{K}; 
\item \Press[i]:          pressure of thermodynamic  phase $i$ homogeneous to \unit{Pa}; 
\item \Vol[i]:            volume of thermodynamic  phase $i$ homogeneous to \unit{m^3}; 
\item \Mol[i]:            number of mole of components of thermodynamic  phase $i$ homogeneous to \unit{mol}; 
\item \Mass[i]:           mass of thermodynamic  phase $i$ homogeneous to \unit{g}; 
\item \Gibbs[i]:          Gibbs free energy of thermodynamic phase $i$, homogeneous to \unit{J\,mol^{-1}};
\item \Enth[i]:           enthalpy of thermodynamic phase $i$, homogeneous to \unit{J\,mol^{-1}};
\item \Entr[i]:           entropy of thermodynamic phase $i$, homogeneous to \unit{J\,mol^{-1}\,K^{-1}}.
\item \vis[i]:            viscosity of species $i$, homogeneous to \unit{Pa\,s}.
\item \dipole[i]:         dipole moment of species $i$, homogeneous to \unit{C\,m}.
\item \diff[i]:           bimolecular diffusion of species $i$, homogeneous to \unit{m^2\,s^{-1}}.
\item \thermcond[i]:      thermal conduction of species $i$, homogeneous to \unit{J\,m^2\,s^{-1}}.
\item \LJdepth[i]:        Lennard-Jones potential well's depth, usually in reduced form, that
                          is divided by Boltzmann's constant (\BoltzmannEquation)
                          homegeneous (when reduced) to \unit{K}.
\item \LJdia[i]:          Lennard-Jones potential diameter, homogeneous to \unit{m}.
\end{itemize}
%
\begin{tabular}{lcccc}\toprule
Quantity  & species-wise & phase-wise & reaction-wise    & unit \\\midrule
Enthalphy & \enth        & \Enth      & \Denth/\DenthZ   & \unit{J\,mol^{-1}}\\
Entropy   & \entr        & \Entr      & \Dentr/\DentrZ   & \unit{J\,mol^{-1}\,K^{-1}}\\
Gibbs free energy
          & \gibbs       & \Gibbs     & \DGibbs/\DGibbsZ & \unit{J\,mol^{-1}}\\
Chemical potential
          & \chempot     & \chempot   &                  & \unit{J\,mol^{-1}}\\\cmidrule(lr){2-2}
Absolute stoichiometric coefficient
          & \scoefabs    &            &                  & \nounit\\
Stoichiometric coefficient
          & \scoef       &            & \sumscoef\ (sum) & \nounit\\\cmidrule(lr){2-2}
Number of mole
          &              & \Mol       &                  & \unit{mol}\\
Mass      & \Mass        & \Mass      &                  & \unit{g}\\
Density of mass
          & \mass        & \mass      &                  & \unit{g\,m^{-3}}\\
Concentration
          & \conc/\conc[.]
                         & \conc/\conc[M]
                                      &                  & \unit{mol\,m^{-3}}\\
Mass fraction
          & \massfrac    &   1        &                  & \nounit\\
Molar fraction
          & \molarfrac   &   1        &                  & \nounit\\\cmidrule(lr){2-2}
Viscosity & \vis         &   \vis     &                  & \unit{Pa\,s}\\
Dipole moment 
          & \dipole      &            &                  & \unit{C\,m}\\
Diffusion & \diff        &  \diff     &                  & \unit{m^2\,s^{-1}}\\
Thermal conduction 
          & \thermcond   &  \thermcond 
                                      &                  & \unit{J\,m^2\,s^{-1}}\\\cmidrule(lr){2-2}
Kinetic model
          &              &            &   \kinMod        & depends\\
Chemical process
          &              &            &   \chemProc      & depends\\
Forward rate constant
          &              &            &   \fwdratecons   & depends\\
Backward rate constant
          &              &            &   \bkwdratecons  & depends\\
Forward rate 
          &              &            &   \fwdrate       & \unit{mol\,m^{-3}\,s^{-1}}\\
Backward rate
          &              &            &   \bkwdrate      & \unit{mol\,m^{-3}\,s^{-1}}\\
Equilibrium constant
          &              &            &   \Eqconst       & \nounit\\
\bottomrule
\end{tabular}


\subsection{Some thoughts on units}
\label{units_in_Antioch}
The units in \antioch\ are managed at the reading
steps. The SI system is the internal system, thus
for kinetics parameters input file.


The parameters we need to be aware of:
\begin{itemize}
\item in the file \file{species\_ascii\_parsing.h}
        \begin{itemize}
        \item molecular weight: mass per quantity of matter (SI is \unit{kg\,mol$^{-1}$}),
        \item heat of formation at 0~\unit{K}: energy per mass (SI is \unit{J\,kg$^{-1}$})
        \end{itemize}
\item in the file \file{physical\_constants.h}
        \begin{itemize}
        \item \Rg: energy per quantity of matter per unit of temperature (SI is \unit{J\,mol$^{-1}$K$^{-1}$}),
        \item Avogadro number: one over quantity of matter (SI is \unit{mol$^{-1}$})
        \end{itemize}
\end{itemize}

Exemple with \Rg. If you go to the
\href{http://physics.nist.gov/cgi-bin/cuu/Value?r}{NIST definition}
of \Rg, you obtain the advised value of \Rg:
$\Rg = \numprint{8.3144621} \pm \numprint{0.0000075}$~\unit{J\,mol$^{-1}$K$^{-1}$}.
Thanks to \href{https://en.wikipedia.org/wiki/Gas_constant}{wikipedia}
(right-side table), we have also the value in \unit{calorie}:
$\Rg = \numprint{1.9858775} \pm \numprint{0.0000034}$~\unit{cal\,mol$^{-1}$K$^{-1}$}.
We look now at the definition of the \unit{calorie} unit, again,
we go to the \href{http://physics.nist.gov/Pubs/SP811/appenB9.html#ENERGY}{NIST website}.
We have then several definitions:
\begin{enumerate}
\item International Table, \unit{calorie$_\text{IT}$},   defined as \numprint{4.1868}  \unit{Joule};
\item thermodynamic, \unit{calorie$_\text{th}$},   defined as \numprint{4.184}   \unit{Joule};
\item mean (?), \unit{calorie$_\text{mean}$}, defined as \numprint{4.19002} \unit{Joule}.
\end{enumerate}
Comparing everyone leads to table~\ref{Rwtf}.
\begin{table}
\centering
\begin{tabular}{lcc}\toprule
\null\hfill Unit \hfill\null                      & \Rg\ value                          & factor to SI \\\midrule
\unit{J\,mol$^{-1}$K$^{-1}$}                      & \numprint{8.3144621}(75)            & \numprint{1.00000} \\
\unit{cal$_\text{IT}$\,mol$^{-1}$K$^{-1}$}        & \color{red}\numprint{1.9858752}(18) & \numprint{4.18680} \\
\unit{cal$_\text{th}$\,mol$^{-1}$K$^{-1}$}        & \color{red}\numprint{1.9872041}(18) & \numprint{4.18400} \\
\unit{cal$_\text{mean}$\,mol$^{-1}$K$^{-1}$}      & \color{red}\numprint{1.9843490}(18) & \numprint{4.19002} \\
\unit{cal$_\text{wikipedia}$\,mol$^{-1}$K$^{-1}$} & \numprint{1.9858775}(34)            & \color{red}\numprint{4.18680}\\
\bottomrule
\end{tabular}
\caption{\label{Rwtf}Discrepancies in \Rg. Red are calculated values from data.}
\end{table}
It seems obvious then that wikipedia gives the International Table (IT) calorie, with a
relative difference in the factor with the given one of $\numprint{1.1806}\,10^{-6}$. 
This relative error is bigger than machine tolerance (from the \textcolor{green!60!black}{\bf double}
to more precise).

We use thus the advised value in \unit{J\,mol$^{-1}$K$^{-1}$} and the factor
\numprint{4.1868} given by the NIST to make the conversion in \unit{calorie}.
Be aware than in this case, you have a discrepancy of $\numprint{1.2}\,10^{-6}$
with the wikipedia value.


\subsection{What is a reaction?}
\label{kinetics_gen}
A chemical reaction is ``a bunch of molecules
turning into another bunch of molecules'', kinetics\footnote{%
from the greek $\kappa\iota\nu\eta\sigma\iota\varsigma$, ``kinesis'', movement, to move}
is about answering the question ``how fast?''.
Thus chemical kinetics is the mathematical model
to calculate the rate at which the molecules disappear and
appear.

\subsubsection{Going forward}

\Antioch's kinetics is based on the elementary step
hypothesis. It means that, as far as the kinetics is
concerned, every reaction is an elementary step:
the reactants get together and produce the products
immediatly. Mathematically, it means
the partial orders are the absolute
value of the stoichiometric coefficients (see next).

Let's consider a chemical reaction:
\begin{chemicalEquation}
\ce{\scoefabs[A] A + \scoefabs[B] B ->[\rcons] \scoefabs[C] C + \scoefabs[D] D}
\label{genericX}
\end{chemicalEquation}
with \rcons\ the rate constant.
We want to model the evolution of the system, that is we want to
characterize 
$\doverdt{\conc[A]}$,
$\doverdt{\conc[B]}$,
$\doverdt{\conc[C]}$,
$\doverdt{\conc[D]}$.
Using the kinetics theory, we have:
\begin{equation}
\frac{1}{\scoef[A]}\doverdt{[A]} = 
\frac{1}{\scoef[B]}\doverdt{[B]} = 
\frac{1}{\scoef[C]}\doverdt{[C]} = 
\frac{1}{\scoef[D]}\doverdt{[D]} = 
\rcons\conc[A]^{\scoefabs[A]}\conc[B]^{\scoefabs[B]}
\end{equation}
with \scoef[A]\ being the stoichiometric coefficient, which is defined by:\\[5pt]
$\left\{\begin{array}{ll}
\scoef[S] = \scoefabs[S] & \text{if \ce{S} is a product} \\
\scoef[S] = -\scoefabs[S] & \text{if \ce{S} is a reactant} \\
\end{array}\right.$\\[5pt]
So the game is to define the rate constant. 

A rate constant is characterized by two things:
\begin{itemize}
\item the kinetics model,
\item the chemical process.
\end{itemize}
The kinetics model will mathematically describe the rate constant's dependence with
the temperature, it is noted \kinMod\ in this manual, the chemical process will
possibly add a pressure dependency, it is noted \chemProc, with \conc[M]\
denoting the pressure dependence.
\Antioch\ propose six different kinetics models and five chemical processes.
The rate constant is characterized thus, for a choice of a chemical process and
a kinetics model:
\begin{equation}
\rateCons = \chemProc
\end{equation}

\subsubsection{Going backward}

Usually, a reaction will be reversible, which means, if we consider
that reaction~\ref{genericX} is reversible, we should note it:
\begin{chemicalEquation}
\ce{\scoefabs[A] A + \scoefabs[B] B <=>[\fwdratecons][\bkwdratecons] \scoefabs[C] C + \scoefabs[D] D}
\label{genericXrev}
\end{chemicalEquation}
with \fwdratecons\ the forward rate constant and \bkwdratecons\ the backward rate constant.
In a given physico-chemical environment, this reaction will eventually reach
steady state, characterized by an equilibrium constant \Eqconst. 
This equilibrium constant is given by
\begin{equation}
\Eqconst[r] = \frac{\fwdratecons[r]}{\bkwdratecons[r]}
\label{therm:K_kin}
\end{equation}
for a reaction $r$.
It is possible to estimate it from thermodynamics considerations, 
using the relation
\begin{equation}
\Eqconst[r] = \left(\frac{\pz}{\Rg \Temp}\right)^\gamma \exp\left(-\frac{\DGibbsZ[r](\Temp)}{\Rg \Temp}\right)
\label{therm:K_therm}
\end{equation}
The demonstrations are given in appendix~\ref{demo-eq_kin} and \ref{demo-eq_therm}. 
Thus the backward rate constant is therefore known given:
\begin{itemize}
\item the forward rate constant;
\item the thermodynamics of the molecules.
\end{itemize}

\subsubsection{Going nowhere: steady state, a.k.a equilibrium}
\label{phys:equilibrium}
\subsubsection{With kinetics}
A steady state is defined by
%
\begin{equation}
\forall\:s,\quad \doverdt{\conc[s]} = 0
\label{equilibrium:def}
\end{equation}
%
The system to be solved is of the form:
\begin{equation}
A\times x = b
\end{equation}
with $b$ the vector of \mdot, $A$ the matrixes of \doverdm[E]{\mdot[s]} for
the species (rows are $s$ and columns are $E$) and $x$ the vector of the solution \mass.
To close the system, we use the mass conservation equation and use a
species to ensure it:
\begin{equation}
\sum_s \mass[s] = \mathrm{mass_{tot}}
\label{mass_cons}
\end{equation}
with $\mathrm{mass_{tot}}$ being a constant, here the density of mass of the system.

So, for $N$ chemical species, we have the system:
\begin{equation}
\left[\begin{array}{cccc}
\doverdm[s_1]{\mdot[s_1]}     & \doverdm[s_2]{\mdot[s_1]}     & \cdots & \doverdm[s_N]{\mdot[s_1]} \\
\doverdm[s_1]{\mdot[s_2]}     & \doverdm[s_2]{\mdot[s_2]}     & \cdots & \doverdm[s_N]{\mdot[s_2]} \\
\vdots                        & \vdots                        & \vdots & \vdots                    \\
\doverdm[s_1]{\mdot[s_{N-1}]} & \doverdm[s_2]{\mdot[s_{N-1}]} & \cdots & \doverdm[s_N]{\mdot[s_{N-1}]}\\
1                             & 1                             & \cdots & 1\\
\end{array}\right]
\left[\begin{array}{c}
\Delta\mass[s_1]\\
\Delta\mass[s_2]\\
\vdots\\
\Delta\mass[s_N]\\
\end{array}\right]
=
\left[\begin{array}{c}
\mdot[s_1]\\
\mdot[s_2]\\
\mdot[s_1]\\
\vdots\\
\mdot[s_{N_1}]\\
\sum_{s=1}^N\mass[s] - \mathrm{mass_{tot}}
\end{array}\right]
\label{eq:matrixes}
\end{equation}
The total fixed mass is calculated thanks to the ideal gas state equation
(see section~\ref{relations})
\begin{equation}
\mathrm{mass_{tot}} = \Mm[\mathrm{mix}] \frac{P}{\Rg T}
\label{tot_mass}
\end{equation}
with \Mm[\mathrm{mix}] calculated as seen in section~\ref{relations}.
Thus an initial guess of \massfrac\ is necessary.
If you don't have any idea, let's consider the situation
\begin{chemicalEquation}
\ce{A + B ->[k_1] C + D ->[k_2] E + F}
\label{youpi}
\end{chemicalEquation}
we have
\begin{equation}
\doverdt{\conc[C]} = k_1\conc[A]\conc[B] - k_2\conc[C]\conc[D]
\end{equation}
therefore, a first approximation can be
\begin{equation}
\conc[C^{(\text{approx})}] = \frac{k_1\conc[A]\conc[B]}{k_2\conc[D]} = \frac{\mathrm{prod}}{\mathrm{loss}}
\end{equation}

This will be efficient in somewhat easy situations, meaning you're looking for the
steady state of minor species for instance.

\subsubsection{With thermodynamics}
A thermodynamic phase is at equilibrium for a minimized Gibbs energy (at \Temp, \Press\ constant), with
the relation
\begin{equation}
\dd\Gibbs = \Vol\dd\Press - \Entr\dd\Temp + \sum_s\chempot[s]\dd\Mol[s]
\end{equation}
and we have Euler's identity
\begin{equation}
\Gibbs = \sum_s \Mol[s]\chempot[s]
\label{Euler_id}
\end{equation}
with also,
\begin{equation}
\left(\doverdext[r]{\Gibbs[s]}\right)_{\Temp,\Press} = \scoef[s,r]\chempot[s]
\end{equation}
We note
\begin{equation}
\DGibbs_r = \sum_s \scoef[s,r]\chempot[s] \left[= \sum_s \left(\doverdext[r]{\Gibbs[s]}\right)_{\Temp,\Press}\right]
\end{equation}
Considering
\begin{equation}
\chempot_s = \doverdn[s]{\Gibbs[s]}
\end{equation}
We have, deduced from~\ref{Euler_id}
\begin{equation}
\chempot[s] = \gibbs[s]
\end{equation}
Thus,
\begin{equation}
\chempot[s] = \chempotZ[s] + \Rg\Temp\ln\left(\frac{\press[s]}{\pz}\right)
     \left[ = \chempotZ[s] + \Rg\Temp\ln\left(\frac{\Press}{\pz}\right) + \Rg\Temp\ln\left(\molarfrac[s]\right) \right]
\end{equation}
The story behind chemical extent is:
\begin{equation}
\Mol[s] = \Mol[s](t=0) + \sum_r \scoef[s,r] \ext[r]
\end{equation}
Using the ideal gas equation:
\begin{equation}
\Press = \conc \Rg \Temp
\end{equation}
thus
\begin{equation}
\begin{split}
\chempot[s] & = \chempotZ[s] + \Rg\Temp\ln\left(\frac{\Rg \Temp \molar[s]}{\pz}\right) \\
            & = \chempotZ[s] + \Rg\Temp\ln\left(\frac{\Rg \Temp}{\pz}\left(\molar[s](t=0) + \sum_r \scoef[s,r]\frac{\ext[r]}{\Vol}\right)\right) \\
\end{split}
\end{equation}
and therefore
\begin{equation}
\begin{split}
\doverdext[r]{\chempot[s]}      & = \frac{\pz}{\Vol}\frac{\scoef[s,r]}{\molar[s](t=0) + \sum_{r'} \scoef[s,r']\frac{\ext[r']}{\Vol}}\\
\ddoverddext{\chempot[s]}{r}{l} & = -\frac{\pz}{\Vol^2}\frac{\scoef[s,l]\scoef[s,r]}{\left(\molar[s](t=0) + \sum_{r'} \scoef[s,r']\frac{\ext[r']}{\Vol}\right)^2}\\
\end{split}
\end{equation}
Usually, it's better to consider the system per unit of volume, 
using an extent of reaction per volume, thus the equations
become:
\begin{equation}
\begin{split}
\chempot[s]                     & = \chempotZ[s] + \Rg\Temp\ln\left(\frac{\Rg \Temp}{\pz}\left(\molar[s](t=0) + \sum_r \scoef[s,r]\ext[r]\right)\right) \\
\doverdext[r]{\chempot[s]}      & = \pz\frac{\scoef[s,r]}{\molar[s](t=0) + \sum_{r'} \scoef[s,r']\ext[r']}\\
\ddoverddext{\chempot[s]}{r}{l} & = -\pz\frac{\scoef[s,l]\scoef[s,r]}{\left(\molar[s](t=0) + \sum_{r'} \scoef[s,r']\ext[r']\right)^2}\\
\end{split}
\end{equation}

Equilibrium is defined by 
\begin{equation}
\forall\; r,\; \DGibbs_r = 0
\end{equation} 
or 
\begin{equation}
\min \Gibbs(\{\chempot[s]\}_s)
\end{equation}

\subsubsection{\texorpdfstring{$\forall\;r,\;\DGibbs_r = 0$}{Reaction enthalpy}}

\begin{equation}
\begin{split}
\DGibbs_r & = \sum_s \scoef[s,r] \chempot[s]\\
\doverdext[i]{\DGibbs_r}   & = \sum_s\scoef[s,r]\doverdext[i]{\chempot[s]} \\
\ddoverddext{\DGibbs_r}{i}{j} & = \sum_s\scoef[s,r]\ddoverddext{\chempot[s]}{i}{j} \\
\end{split}
\end{equation}

For $R$ reactions:
\begin{equation}
\left[\begin{array}{cccc}
\sum_s\scoef[s,1]\doverdext[1]{\chempot[s]} & \sum_s\scoef[s,1]\doverdext[2]{\chempot[s]} & \cdots & \sum_s\scoef[s,1]\doverdext[R]{\chempot[s]}\\
\sum_s\scoef[s,2]\doverdext[1]{\chempot[s]} & \sum_s\scoef[s,2]\doverdext[2]{\chempot[s]} & \cdots & \sum_s\scoef[s,2]\doverdext[R]{\chempot[s]}\\
\vdots &\vdots &\vdots &\vdots \\
\sum_s\scoef[s,R]\doverdext[1]{\chempot[s]} & \sum_s\scoef[s,R]\doverdext[2]{\chempot[s]} & \cdots & \sum_s\scoef[s,R]\doverdext[R]{\chempot[s]}\\
\end{array}\right]
\times
\left[\begin{array}{c}
\Delta\ext[1] \\
\Delta\ext[2] \\
\vdots \\ 
\Delta\ext[R]
\end{array}\right]
=
\left[\begin{array}{c}
\sum_s \scoef[s,1] \chempot[s] \\ 
\sum_s \scoef[s,2] \chempot[s] \\ 
\vdots \\ 
\sum_s \scoef[s,R] \chempot[s]
\end{array}\right]
\end{equation}

\subsubsection{\texorpdfstring{$\min\Gibbs(\{\chempot[s]\}_s)$}{Phase enthalpy minimization}}

\begin{equation}
\begin{split}
\Gibbs & = \sum_s\Mol[s]\chempot[s]
         = \sum_s\left(\Mol[s]^0 + \sum_r\scoef[s,r]\ext[r]\right)\chempot[s] \\
\doverdext[r]{\Gibbs} & = \sum_s \left[\left(\Mol[s]^0 + \sum_{r'}\scoef[s,r']\ext[r']\right)\doverdext[r]{\chempot[s]}
                                + \scoef[s,r]\chempot[s]\right] \\
\ddoverddext{\Gibbs}{r}{l} & = \sum_s \left[\left(\Mol[s]^0 + \sum_{r'}\scoef[s,r']\ext[r']\right)\ddoverddext{\chempot[s]}{r}{l}
                                        + \scoef[s,l]\doverdext[r]{\chempot[s]}
                                        + \scoef[s,r]\doverdext[l]{\chempot[s]}\right] \\
\end{split}
\end{equation}
Considering these equations per unit of volume, one obtains:
\begin{equation}
\begin{split}
\Gibbs & = \sum_s\left(\molar[s]^0 + \sum_r\scoef[s,r]\ext[r]\right)\chempot[s] \\
\doverdext[r]{\Gibbs} & = \sum_s \left[\left(\molar[s]^0 + \sum_{r'}\scoef[s,r']\ext[r']\right)\doverdext[r]{\chempot[s]}
                                + \scoef[s,r]\chempot[s]\right] \\
\ddoverddext{\Gibbs}{r}{l} & = \sum_s \left[\left(\molar[s]^0 + \sum_{r'}\scoef[s,r']\ext[r']\right)\ddoverddext{\chempot[s]}{r}{l}
                                        + \scoef[s,l]\doverdext[r]{\chempot[s]}
                                        + \scoef[s,r]\doverdext[l]{\chempot[s]}\right] \\
\end{split}
\end{equation}




\subsection{Useful relations}
\label{relations}
\begin{itemize}
\item $\massfrac[i] = \frac{\mass[i]}{\sum_s \mass[s]}$,
\item $\molarfrac[i] = \frac{\molar[i]}{\sum_s \molar[s]}$,
\item $\Mm[i] = \frac{\mass[i]}{\molar[i]}$,
\item $\frac{1}{\Mm[\text{mix}]} = \sum_s \frac{\massfrac[s]}{\Mm[s]}$ (cf.~\ref{demo-Mm})
\item $\molarfrac[i] = \massfrac[i] \frac{\Mm[\text{mix}]}{\Mm[i]}$,
\item $\Press = \molar\Rg \Temp$,
\item $\frac{\DGibbsZ[r]}{\Rg \Temp} = \frac{\DenthZ[r](\Temp)}{\Rg \Temp} - \frac{\DentrZ[r]}{\Rg}$
\end{itemize}



\section{Full derivations for a chemical kinetics system}
\label{derivations}
What we want is 
$\mdot[S] = \doverdt{\mass[S]}$, and its derivatives:
$\doverdT{\mdot[S]}$, $\doverdm[j]{\mdot[S]}$ with respect to species $j$.

The kinetics model's value is noted \kinMod, the rate constant is noted \rateCons, the
rate of the reaction is noted \rate.

The kinetics model is Kooij 
\begin{equation}
\kinMod = \KooijEq,
\label{KooijEq}
\end{equation}
the chemical process are elementary 
\begin{equation}
\rateCons = \kinMod
\label{ChemProEl}
\end{equation} 
or three-body 
\begin{equation}
\rateCons = \kinMod \threeBody.
\label{ChemProTB}
\end{equation}

We use the equations
\begin{equation}
\mass[S] = \Mm[S] \conc[S]
\end{equation}
\begin{equation}
\dd\mass[S] = \Mm[S]\,\dd\conc[S]
\end{equation}
as everything in kinetics theory is expressed in concentrations and 
not in terms of mass.

We have thus:
\begin{equation}
\mdot[S] = \doverdt{\mass[S]} = \Mm[S] \doverdt{\conc[S]}
\end{equation}
\begin{equation}
\doverdm[E]{\mdot[S]} = \frac{\Mm[S]}{\Mm[E]}\doverdc[E]{\conc[S]}
\end{equation}
\begin{equation}
\doverdT{\mdot[S]} = \Mm[S]\doverdT{\conc[S]}
\end{equation}
but for Paul's sake, we will do the full derivation with mass.

\subsection{Forward}

The forward rate \fwdrate\ for a reaction is by definition for species \ce{S}:
\begin{equation}
\begin{split}
\fwdrate &= \frac{1}{\scoef[S]}\frac{\dd\conc[S]}{\dd t}\\
         &= \frac{1}{\scoef[S]} \fwdratecons \prodReac
\end{split}
\label{ratefDef}
\end{equation}

\paragraph{To derive with respect to \Temp.}
For any species \ce{S} participating in the reaction\footnote{remember, 
$\scoef = \left\{\begin{array}{l}-\scoefabs\text{ for reactants}\\\scoefabs\text{ for products}\end{array}\right.$}.
thus,
\begin{equation}
\doverdT{\fwdrate} = \frac{1}{\scoef[S]}
                   \doverdT{\fwdratecons}
                   \prodReac
\end{equation}
and, for the elementary process, following~\ref{ChemProEl} and \ref{KooijEq}:
\begin{equation}
\begin{split}
\doverdT{\fwdratecons} & = \doverdT{\kinMod} \\
                       & = \frac{\kinMod}{\Temp} \left(\frac{E_a}{\Rg \Temp} + \beta\right)
\end{split}
\end{equation}
For the three-body process, following~\ref{ChemProTB} and \ref{KooijEq}:
\begin{equation}
\begin{split}
\doverdT{\fwdratecons} & = \doverdT{\kinMod} \threeBody \\
                       & = \frac{\kinMod}{\Temp} \left(\frac{E_a}{\Rg \Temp} + \beta\right) \threeBody
\end{split}
\end{equation}

Finally,
for the elementary processes
\begin{equation}
\doverdT{\fwdrate} = \frac{1}{\scoef[S]} \frac{\kinMod}{\Temp} \left(\frac{E_a}{\Rg \Temp} + \beta\right)
                                                \prodReac
\label{derivTEP}
\end{equation}
and the three-body processes
\begin{equation}
\doverdT{\fwdrate} = \frac{1}{\scoef[S]} \frac{\kinMod}{\Temp} \left(\frac{E_a}{\Rg \Temp} + \beta\right) \threeBody
                                                \prodReac
\label{derivTTB}
\end{equation}

\paragraph{To derive with respect to \mass.}
Knowing that $\conc = \frac{\mass}{\Mm}$, we substitute in~\ref{ratefDef}
\begin{equation}
\fwdrate = \frac{1}{\scoef[S]} \fwdratecons \prod_\text{reactants}  \left(\frac{\mass[\reac]}{\Mm[\reac]}\right)^{\scoefabs[\reac]}
\end{equation}
We obtain for a derivation with respect to \mass[E]\ of species \ce{E}:
\begin{equation}
\doverdm[E]{\fwdrate} = \fwdrate \left[
                                \doverdm[E]{\fwdratecons} \frac{1}{\fwdratecons} +
                                \underbrace{\frac{\scoefabs[E]}{\mass[E]}}_{\text{if \ce{E} is a reactant}}
                              \right]
\end{equation}

For elementary process:
\begin{equation}
\doverdm[E]{\fwdratecons} = 0
\end{equation}
For a three-body process:
\begin{equation}
\doverdm[E]{\fwdratecons} = \kinMod \frac{\epsilon_{\ce{E}}}{\Mm[E]}
\end{equation}

Finally, for an elementary process:
\begin{equation}
\doverdm[E]{\fwdrate} = \left\{\begin{array}{ll}
                        \fwdrate \frac{\scoefabs[E]}{\mass[E]} & \text{if \ce{E} is a reactant} \\
                        0                                    & \text{if \ce{E} is not a reactant} \\
                      \end{array}\right.
\end{equation}
and for a three-body process:
\begin{equation}
\renewcommand{\arraystretch}{1.5}
\doverdm[E]{\fwdrate} = \left\{\begin{array}{ll}
                        \fwdrate \left[\frac{\epsilon_{\ce{E}}}{\Mm[E]\threeBody} + \frac{\scoefabs[E]}{\mass[E]} \right] 
                                                & \text{if \ce{E} is a reactant} \\
                        \fwdrate \frac{\epsilon_{\ce{E}}}{\Mm[E]\threeBody} 
                                                & \text{if \ce{E} is not a reactant} \\
                      \end{array}\right.
\end{equation}

\subsection{Backward}

The backward rate is
\begin{equation}
\bkwdrate = \frac{1}{\scoef[S]}\bkwdratecons\prodProd
\end{equation}

The backward rate constant is given by
\begin{equation}
\bkwdratecons = \frac{\fwdratecons}{\Eqconst}
\end{equation}

\paragraph{Derive with respect to \Temp.}
It makes:
\begin{equation}
\doverdT{\bkwdrate} = \frac{1}{\scoef[S]}\prodProd\doverdT{\bkwdratecons}
\end{equation}
Decomposition gives
\begin{equation}
\doverdT{\bkwdratecons} = \doverdT{\fwdratecons}\Eqconst^{-1} - \doverdT{\Eqconst}\frac{\bkwdratecons}{\Eqconst^{2}}
\label{rateb-decomp}
\end{equation}
For \Eqconst, we have
\begin{equation}
\begin{split}
\doverdT{\Eqconst} & = \doverdT{\left[\left(\frac{\pz}{\Rg \Temp}\right)^{\sum_s \scoef[s]} \exp\left(-\frac{\DGibbsZ(\Temp)}{\Rg \Temp}\right)\right]} \\
                   & = \Eqconst\left[-\frac{\sum_s\scoef[s]}{\Temp} - \doverdT{\left[\frac{\DGibbsZ(\Temp)}{\Rg \Temp}\right]}\right]
\end{split}
\end{equation}
The derived value of \DGibbsZ(\Temp) is easily given, see~\ref{data-thermo}
\begin{equation}
\begin{split}
\doverdT{\left[\frac{\DGibbsZ(\Temp)}{\Rg \Temp}\right]} 
        &= \doverdT{\left[\frac{\DenthZ(T)}{\Rg \Temp}\right]} - \doverdT{\left[\frac{\DenthZ(\Temp)}{\Rg \Temp}\right]} \\
        &= 2\tc{0}\Temp^{-3} + \tc{1}\Temp^{-2}\left(1+\ln(\Temp)\right) + \frac{\tc{3}}{2} + 
           \frac{2}{3}\tc{4}\Temp + \frac{3}{4}\tc{5}\Temp^{2} + \frac{4}{5}\tc{6}\Temp^3 - \tc{8}\Temp^{-2}\\
        &+ \tc{0}\Temp^{-3} + \tc{1}\Temp^{-2} + \tc{2}\Temp^{-1} + \tc{3} + \tc{4}\Temp + \tc{5} \Temp^{2} + \tc{6} \Temp^{3}
\end{split}
\end{equation}
Finally,
\begin{equation}
\doverdT{\left[\frac{\DGibbsZ(\Temp)}{\Rg \Temp}\right]} =
        3\tc{0}\Temp^{-3} + \tc{1} \Temp^{-2} \left(2 + \ln(\Temp)\right) + \frac{3}{2} \tc{3} + \frac{5}{3} \tc{4} \Temp
        + \frac{7}{4}\tc{5} \Temp^2 + \frac{9}{5}\tc{6} \Temp^3 - \tc{8} \Temp^{-2}
\end{equation}
\emph{Note:} $\ln(\Temp)$ should be understood as $\ln\left(\frac{\Temp}{\mathrm{\Temp_{ref}}}\right)$ with 
$\mathrm{\Temp_{ref}} = 1$~K.

Thus, for elementary processes
\begin{equation}
\begin{split}
\doverdT{\bkwdrate} = & \frac{1}{\scoef[S]} \prodProd \\
&\bigg[
        \frac{\kinMod}{\Temp} \left(\frac{E_a}{\Rg \Temp} + \beta\right)\prodReac \Eqconst^{-1}
                  - 3\tc{0}\Temp^{-3} - \tc{1} \Temp^{-2} \left(2 + \ln(\Temp)\right) - \frac{3}{2} \tc{3} \\
&                 - \frac{5}{3} \tc{4} \Temp - \frac{7}{4}\tc{5} \Temp^2 - \frac{9}{5}\tc{6} \Temp^3 + \tc{8} \Temp^{-2} \bigg]
\end{split}
\end{equation}
and for third-body processes
\begin{equation}
\begin{split}
\doverdT{\bkwdrate} = & \frac{1}{\scoef[S]} \prodProd \\
&\bigg[
        \frac{\kinMod}{\Temp} \left(\frac{E_a}{\Rg \Temp} + \beta\right) \threeBody \Eqconst^{-1} 
                  - 3\tc{0}\Temp^{-3} - \tc{1} \Temp^{-2} \left(2 + \ln(\Temp)\right) - \frac{3}{2} \tc{3} \\
&                 - \frac{5}{3} \tc{4} \Temp - \frac{7}{4}\tc{5} \Temp^2 - \frac{9}{5}\tc{6} \Temp^3 + \tc{8} \Temp^{-2} \bigg]
\end{split}
\end{equation}

\paragraph{Derivation with respect to mass.}
Using~\ref{rateb-decomp}
\begin{equation}
\begin{split}
\doverdm[E]{\bkwdratecons} & = \doverdm[E]{\fwdratecons}\Eqconst^{-1} - \doverdm[E]{\Eqconst}\frac{\bkwdratecons}{\Eqconst^{2}} \\
                           & = \doverdm[E]{\fwdratecons}\Eqconst^{-1}
\end{split}
\label{rateb-decomp-m}
\end{equation}
thus, for elementary processes
\begin{equation}
\doverdm[E]{\rate} = \left\{\begin{array}{ll}
                        \fwdrate \frac{\scoefabs[E]}{\mass[E]} \Eqconst^{-1} & \text{if \ce{E} is a reactant} \\
                          0                                                  & \text{if \ce{E} is not a reactant} \\
                      \end{array}\right.
\end{equation}
and for three-body processes
\begin{equation}
\renewcommand{\arraystretch}{1.5}
\doverdm[E]{\fwdrate} = \left\{\begin{array}{ll}
                        \fwdrate \left[\frac{\epsilon_{\ce{E}}}{\Mm[E]\threeBody} + \frac{\scoefabs[E]}{\mass[E]} \right] \Eqconst^{-1}
                                                & \text{if \ce{E} is a reactant} \\
                        \fwdrate \frac{\epsilon_{\ce{E}}}{\Mm[E]\threeBody} \Eqconst^{-1}
                                                & \text{if \ce{E} is not a reactant} \\
                      \end{array}\right.
\end{equation}

\subsection{Net rate}

The net rate is
\begin{equation}
\rate = \fwdrate - \bkwdrate
\end{equation}
thus,
\begin{equation}
\doverdT{\rate} = \doverdT{\fwdrate} - \doverdT{\bkwdrate}
\end{equation}
and
\begin{equation}
\doverdm{\rate} = \doverdm{\fwdrate} - \doverdm{\bkwdrate}
\end{equation}
thus, for elementary processes
\begin{equation}
\begin{split}
\doverdT{\rate[{\text{net}}]} &= \frac{1}{\scoef[S]} \frac{\kinMod}{\Temp} \left(\frac{E_a}{\Rg \Temp} + \beta\right) \prodReac\\
& -
\frac{1}{\scoef[S]} \prodProd 
\bigg[
        \frac{\kinMod}{\Temp} \left(\frac{E_a}{\Rg \Temp} + \beta\right)\prodReac \Eqconst^{-1} \\
                  & \hspace{3cm} - 3\tc{0}\Temp^{-3} - \tc{1} \Temp^{-2} \left(2 + \ln(\Temp)\right) - \frac{3}{2} \tc{3} \\
                  & \hspace{3cm} - \frac{5}{3} \tc{4} \Temp - \frac{7}{4}\tc{5} \Temp^2 - \frac{9}{5}\tc{6} \Temp^3 + \tc{8} \Temp^{-2} \bigg]
\end{split}
\end{equation}
and for three-body processes
\begin{equation}
\begin{split}
\doverdT{\rate[{\text{net}}]} & = \frac{1}{\scoef[S]} \frac{\kinMod}{\Temp} \left(\frac{E_a}{\Rg \Temp} + \beta\right) \threeBody\prodReac\\
& -
\frac{1}{\scoef[S]} \prodProd 
\bigg[
        \frac{\kinMod}{\Temp} \left(\frac{E_a}{\Rg \Temp} + \beta\right)\threeBody\prodReac \Eqconst^{-1} \\
                  & \hspace{3cm} - 3\tc{0}\Temp^{-3} - \tc{1} \Temp^{-2} \left(2 + \ln(\Temp)\right) - \frac{3}{2} \tc{3} \\
                  & \hspace{3cm} - \frac{5}{3} \tc{4} \Temp - \frac{7}{4}\tc{5} \Temp^2 - \frac{9}{5}\tc{6} \Temp^3 + \tc{8} \Temp^{-2} \bigg]
\end{split}
\end{equation}
and for the mass, elementary processes
\begin{equation}
\doverdm[E]{\rate} = \left\{\begin{array}{ll}
                        \fwdrate \frac{\scoefabs[E]}{\mass[E]}\left(1-\Eqconst^{-1}\right) & \text{if \ce{E} is a reactant} \\
                          0                                                                & \text{if \ce{E} is not a reactant} \\
                      \end{array}\right.
\end{equation}
and for the three-body processes
\begin{equation}
\renewcommand{\arraystretch}{1.5}
\doverdm[E]{\rate} = \left\{\begin{array}{ll}
                        \fwdrate \left[\frac{\epsilon_{\ce{E}}}{\Mm[E]\threeBody} + \frac{\scoefabs[E]}{\mass[E]} \right] \left(1 - \Eqconst^{-1}\right)
                                                & \text{if \ce{E} is a reactant} \\
                        \fwdrate \frac{\epsilon_{\ce{E}}}{\Mm[E]\threeBody}  \left(1 - \Eqconst^{-1}\right)
                                                & \text{if \ce{E} is not a reactant} \\
                      \end{array}\right.
\end{equation}

Back to \mdot:
\begin{equation}
\begin{split}
\mdot[S] &= \Mm[S] \doverdt{\ce{S}} \\
         &= \Mm[S] \scoef[S]\rate
\end{split}
\end{equation}

\begin{equation}
\renewcommand{\arraystretch}{1.5}
\left\{\begin{array}{l}
\doverdT{\mdot[S]} = \Mm[S] \scoef[S]\doverdT{\rate}\\
\doverdm[E]{\mdot[S]} = \Mm[S] \scoef[S]\doverdm[E]{\rate}
\end{array}\right.
\end{equation}



\section{Let's look at some problems}
\subsection{Data}
Mixture: 
\ce{N2}, \ce{O2}, \ce{N}, \ce{O}, \ce{NO}.
(see thermo tab. at~\ref{data-thermo}).

Reactions:\\*
\begin{tabular}{*{3}{m{4cm}}}
\ce{N2 + M <=> 2 N + M} & \ce{O2 + M <=> 2 O + M} & \ce{NO + M <=> N + O + M} \\[5pt]
\parbox{4cm}{%
$A = 7\,10^{18}~\unit{m^3\,kmol^{-1}\,s^{-1}}$,\\
$\beta = -1.6$,\\
$E_a = 224801.3~\unit{cal\,mol^{-1}}$\\
$\epsilon_\ce{N2} = 1.0$,  $\epsilon_\ce{O2} = 1.0$,  
      $\epsilon_\ce{NO} = 1.0$,  $\epsilon_\ce{N} = 4.2857$,  
      $\epsilon_\ce{O} = 4.2857$.}
&
\parbox{4cm}{%
$A = 2\,10^{18}~\unit{m^3\,kmol^{-1}\,s^{-1}}$,\\
$\beta = -1.5$,\\
$E_a = 117881.7~\unit{cal\,mol^{-1}}$\\
$\epsilon_\ce{N2} = 1.0$,  $\epsilon_\ce{O2} = 1.0$,  
      $\epsilon_\ce{NO} = 1.0$,  $\epsilon_\ce{N} = 5.0$,  
      $\epsilon_\ce{O} = 5.0$.}
&
\parbox{4cm}{%
$A = 5\,10^{12}~\unit{m^3\,kmol^{-1}\,s^{-1}}$,\\
$\beta = 0.0$,\\
$E_a = 149943.0~\unit{cal\,mol^{-1}}$
$\epsilon_\ce{N2} = 1.0$,  $\epsilon_\ce{O2} = 1.0$,  
      $\epsilon_\ce{NO} = 22.0$,  $\epsilon_\ce{N} = 22.0$,  
      $\epsilon_\ce{O} = 22.0$.}
\\\\
\ce{N2 + O <=> NO + N} & \ce{NO + O <=> O2 + N}\\[5pt]
\parbox{4cm}{%
$A = 5.7\,10^{9}~\unit{m^3\,kmol^{-1}\,s^{-1}}$,\\
$\beta = 0.42$,\\
$E_a = 85269.6~\unit{cal\,mol^{-1}}$}
&       
\parbox{4cm}{%
$A = 8.4\,10^{9}~\unit{m^3\,kmol^{-1}\,s^{-1}}$,\\
$\beta = 0.0$,\\
$E_a = 38526.0~\unit{cal\,mol^{-1}}$}
\end{tabular}
\medskip

Data used:\\*
\begin{tabular}{l}
$\Mm_\ce{N} = 14.008~\unit{g\,mol^{-1}}$, \\
$\Mm_\ce{O} = 16.000~\unit{g\,mol^{-1}}$, \\
$\Rg = 8314.4621~\unit{J\,kmol^{-1}\,K^{-1}}$.
\end{tabular}

Conditions:\\*
\begin{tabular}{l}
$P = 10^5~\unit{Pa}$, \\
$T = 1500.0~\unit{K}$.
\end{tabular}


\section{First problem: kinetics regression}
Let's face it, \Antioch\ is mainly about calculating
efficiently, with elegance and a sexy attitude, rates,
be them with respect to mass or molar concentration. 

The first thing to clear out is the template choice. 
\Antioch\ is very suple about this, 
The objects we will need are
a \prog{ReactionSet} and a \prog{KineticsEvaluator}. We will
use as \prog{double} a \prog{double} and as vector the standard 
\prog{std::vector< >} 

\begin{verbatim}
// C++
#include <string>
#include <vector>

// Antioch
#include "antioch/vector_utils.h"

#include "antioch/antioch_asserts.h"
#include "antioch/chemical_species.h"
#include "antioch/chemical_mixture.h"
#include "antioch/reaction_set.h"
#include "antioch/read_reaction_set_data_xml.h"
#include "antioch/cea_thermo.h"
#include "antioch/kinetics_evaluator.h"

int compute_rates(const std::string& input_name,
  std::vector<double> & omega_dot,
  std::vector<double> & domega_dot_dT,
  std::vector<std::vector<double> > & domega_dot_drho_s)
{
  std::vector<std::string> species_str_list;
  const unsigned int n_species = 5;
  species_str_list.reserve(n_species);
  species_str_list.push_back( "N2" );
  species_str_list.push_back( "O2" );
  species_str_list.push_back( "N" );
  species_str_list.push_back( "O" );
  species_str_list.push_back( "NO" );

  Antioch::ChemicalMixture<double> chem_mixture( species_str_list );
  Antioch::ReactionSet<double> reaction_set( chem_mixture );
  Antioch::CEAThermodynamics<double> thermo( chem_mixture );

  Antioch::read_reaction_set_data_xml<double>( input_name, true, 
                                               reaction_set );

  const double T = 1500.0; // K
  const double P = 1.0e5; // Pa

  // Mass fractions
  std::vector<double> Y(n_species,0.2);

  const double R_mix = chem_mixture.R(Y); //get R_tot in J.kg-1.K-1

  const double rho = P/(R_mix*T); // kg.m-3

  std::vector<double> molar_densities(n_species,0.0);
  chem_mixture.molar_densities(rho,Y,molar_densities);

  std::vector<double> h_RT_minus_s_R(n_species);
  std::vector<double> dh_RT_minus_s_R_dT(n_species);

  typedef typename Antioch::CEAThermodynamics<double>::template 
                Cache<double> Cache;
  thermo.h_RT_minus_s_R(Cache(T),h_RT_minus_s_R);
  thermo.dh_RT_minus_s_R_dT(Cache(T),dh_RT_minus_s_R_dT);

  Antioch::KineticsEvaluator<double> kinetics( reaction_set, 0 );

  omega_dot.resize(n_species,0.L);
  domega_dot_dT(n_species,0.L);
  domega_dot_drho_s(n_species);
  for( unsigned int s = 0; s < n_species; s++ )
    {
      domega_dot_drho_s[s].resize(n_species);
    }
  
  kinetics.compute_mass_sources( T, molar_densities, 
                                 h_RT_minus_s_R, omega_dot);

  kinetics.compute_mass_sources_and_derivs( T, molar_densities, 
                                            h_RT_minus_s_R, dh_RT_minus_s_R_dT,
                                            omega_dot, domega_dot_dT, 
                                            domega_dot_drho_s );
  return 0;
}
\end{verbatim}


\section{Second problem: kinetics equilibrium}
\subsubsection{With kinetics}
A steady state is defined by
%
\begin{equation}
\forall\:s,\quad \doverdt{\conc[s]} = 0
\label{equilibrium:def}
\end{equation}
%
The system to be solved is of the form:
\begin{equation}
A\times x = b
\end{equation}
with $b$ the vector of \mdot, $A$ the matrixes of \doverdm[E]{\mdot[s]} for
the species (rows are $s$ and columns are $E$) and $x$ the vector of the solution \mass.
To close the system, we use the mass conservation equation and use a
species to ensure it:
\begin{equation}
\sum_s \mass[s] = \mathrm{mass_{tot}}
\label{mass_cons}
\end{equation}
with $\mathrm{mass_{tot}}$ being a constant, here the density of mass of the system.

So, for $N$ chemical species, we have the system:
\begin{equation}
\left[\begin{array}{cccc}
\doverdm[s_1]{\mdot[s_1]}     & \doverdm[s_2]{\mdot[s_1]}     & \cdots & \doverdm[s_N]{\mdot[s_1]} \\
\doverdm[s_1]{\mdot[s_2]}     & \doverdm[s_2]{\mdot[s_2]}     & \cdots & \doverdm[s_N]{\mdot[s_2]} \\
\vdots                        & \vdots                        & \vdots & \vdots                    \\
\doverdm[s_1]{\mdot[s_{N-1}]} & \doverdm[s_2]{\mdot[s_{N-1}]} & \cdots & \doverdm[s_N]{\mdot[s_{N-1}]}\\
1                             & 1                             & \cdots & 1\\
\end{array}\right]
\left[\begin{array}{c}
\Delta\mass[s_1]\\
\Delta\mass[s_2]\\
\vdots\\
\Delta\mass[s_N]\\
\end{array}\right]
=
\left[\begin{array}{c}
\mdot[s_1]\\
\mdot[s_2]\\
\mdot[s_1]\\
\vdots\\
\mdot[s_{N_1}]\\
\sum_{s=1}^N\mass[s] - \mathrm{mass_{tot}}
\end{array}\right]
\label{eq:matrixes}
\end{equation}
The total fixed mass is calculated thanks to the ideal gas state equation
(see section~\ref{relations})
\begin{equation}
\mathrm{mass_{tot}} = \Mm[\mathrm{mix}] \frac{P}{\Rg T}
\label{tot_mass}
\end{equation}
with \Mm[\mathrm{mix}] calculated as seen in section~\ref{relations}.
Thus an initial guess of \massfrac\ is necessary.
If you don't have any idea, let's consider the situation
\begin{chemicalEquation}
\ce{A + B ->[k_1] C + D ->[k_2] E + F}
\label{youpi}
\end{chemicalEquation}
we have
\begin{equation}
\doverdt{\conc[C]} = k_1\conc[A]\conc[B] - k_2\conc[C]\conc[D]
\end{equation}
therefore, a first approximation can be
\begin{equation}
\conc[C^{(\text{approx})}] = \frac{k_1\conc[A]\conc[B]}{k_2\conc[D]} = \frac{\mathrm{prod}}{\mathrm{loss}}
\end{equation}

This will be efficient in somewhat easy situations, meaning you're looking for the
steady state of minor species for instance.

\subsubsection{With thermodynamics}
A thermodynamic phase is at equilibrium for a minimized Gibbs energy (at \Temp, \Press\ constant), with
the relation
\begin{equation}
\dd\Gibbs = \Vol\dd\Press - \Entr\dd\Temp + \sum_s\chempot[s]\dd\Mol[s]
\end{equation}
and we have Euler's identity
\begin{equation}
\Gibbs = \sum_s \Mol[s]\chempot[s]
\label{Euler_id}
\end{equation}
with also,
\begin{equation}
\left(\doverdext[r]{\Gibbs[s]}\right)_{\Temp,\Press} = \scoef[s,r]\chempot[s]
\end{equation}
We note
\begin{equation}
\DGibbs_r = \sum_s \scoef[s,r]\chempot[s] \left[= \sum_s \left(\doverdext[r]{\Gibbs[s]}\right)_{\Temp,\Press}\right]
\end{equation}
Considering
\begin{equation}
\chempot_s = \doverdn[s]{\Gibbs[s]}
\end{equation}
We have, deduced from~\ref{Euler_id}
\begin{equation}
\chempot[s] = \gibbs[s]
\end{equation}
Thus,
\begin{equation}
\chempot[s] = \chempotZ[s] + \Rg\Temp\ln\left(\frac{\press[s]}{\pz}\right)
     \left[ = \chempotZ[s] + \Rg\Temp\ln\left(\frac{\Press}{\pz}\right) + \Rg\Temp\ln\left(\molarfrac[s]\right) \right]
\end{equation}
The story behind chemical extent is:
\begin{equation}
\Mol[s] = \Mol[s](t=0) + \sum_r \scoef[s,r] \ext[r]
\end{equation}
Using the ideal gas equation:
\begin{equation}
\Press = \conc \Rg \Temp
\end{equation}
thus
\begin{equation}
\begin{split}
\chempot[s] & = \chempotZ[s] + \Rg\Temp\ln\left(\frac{\Rg \Temp \molar[s]}{\pz}\right) \\
            & = \chempotZ[s] + \Rg\Temp\ln\left(\frac{\Rg \Temp}{\pz}\left(\molar[s](t=0) + \sum_r \scoef[s,r]\frac{\ext[r]}{\Vol}\right)\right) \\
\end{split}
\end{equation}
and therefore
\begin{equation}
\begin{split}
\doverdext[r]{\chempot[s]}      & = \frac{\pz}{\Vol}\frac{\scoef[s,r]}{\molar[s](t=0) + \sum_{r'} \scoef[s,r']\frac{\ext[r']}{\Vol}}\\
\ddoverddext{\chempot[s]}{r}{l} & = -\frac{\pz}{\Vol^2}\frac{\scoef[s,l]\scoef[s,r]}{\left(\molar[s](t=0) + \sum_{r'} \scoef[s,r']\frac{\ext[r']}{\Vol}\right)^2}\\
\end{split}
\end{equation}
Usually, it's better to consider the system per unit of volume, 
using an extent of reaction per volume, thus the equations
become:
\begin{equation}
\begin{split}
\chempot[s]                     & = \chempotZ[s] + \Rg\Temp\ln\left(\frac{\Rg \Temp}{\pz}\left(\molar[s](t=0) + \sum_r \scoef[s,r]\ext[r]\right)\right) \\
\doverdext[r]{\chempot[s]}      & = \pz\frac{\scoef[s,r]}{\molar[s](t=0) + \sum_{r'} \scoef[s,r']\ext[r']}\\
\ddoverddext{\chempot[s]}{r}{l} & = -\pz\frac{\scoef[s,l]\scoef[s,r]}{\left(\molar[s](t=0) + \sum_{r'} \scoef[s,r']\ext[r']\right)^2}\\
\end{split}
\end{equation}

Equilibrium is defined by 
\begin{equation}
\forall\; r,\; \DGibbs_r = 0
\end{equation} 
or 
\begin{equation}
\min \Gibbs(\{\chempot[s]\}_s)
\end{equation}

\subsubsection{\texorpdfstring{$\forall\;r,\;\DGibbs_r = 0$}{Reaction enthalpy}}

\begin{equation}
\begin{split}
\DGibbs_r & = \sum_s \scoef[s,r] \chempot[s]\\
\doverdext[i]{\DGibbs_r}   & = \sum_s\scoef[s,r]\doverdext[i]{\chempot[s]} \\
\ddoverddext{\DGibbs_r}{i}{j} & = \sum_s\scoef[s,r]\ddoverddext{\chempot[s]}{i}{j} \\
\end{split}
\end{equation}

For $R$ reactions:
\begin{equation}
\left[\begin{array}{cccc}
\sum_s\scoef[s,1]\doverdext[1]{\chempot[s]} & \sum_s\scoef[s,1]\doverdext[2]{\chempot[s]} & \cdots & \sum_s\scoef[s,1]\doverdext[R]{\chempot[s]}\\
\sum_s\scoef[s,2]\doverdext[1]{\chempot[s]} & \sum_s\scoef[s,2]\doverdext[2]{\chempot[s]} & \cdots & \sum_s\scoef[s,2]\doverdext[R]{\chempot[s]}\\
\vdots &\vdots &\vdots &\vdots \\
\sum_s\scoef[s,R]\doverdext[1]{\chempot[s]} & \sum_s\scoef[s,R]\doverdext[2]{\chempot[s]} & \cdots & \sum_s\scoef[s,R]\doverdext[R]{\chempot[s]}\\
\end{array}\right]
\times
\left[\begin{array}{c}
\Delta\ext[1] \\
\Delta\ext[2] \\
\vdots \\ 
\Delta\ext[R]
\end{array}\right]
=
\left[\begin{array}{c}
\sum_s \scoef[s,1] \chempot[s] \\ 
\sum_s \scoef[s,2] \chempot[s] \\ 
\vdots \\ 
\sum_s \scoef[s,R] \chempot[s]
\end{array}\right]
\end{equation}

\subsubsection{\texorpdfstring{$\min\Gibbs(\{\chempot[s]\}_s)$}{Phase enthalpy minimization}}

\begin{equation}
\begin{split}
\Gibbs & = \sum_s\Mol[s]\chempot[s]
         = \sum_s\left(\Mol[s]^0 + \sum_r\scoef[s,r]\ext[r]\right)\chempot[s] \\
\doverdext[r]{\Gibbs} & = \sum_s \left[\left(\Mol[s]^0 + \sum_{r'}\scoef[s,r']\ext[r']\right)\doverdext[r]{\chempot[s]}
                                + \scoef[s,r]\chempot[s]\right] \\
\ddoverddext{\Gibbs}{r}{l} & = \sum_s \left[\left(\Mol[s]^0 + \sum_{r'}\scoef[s,r']\ext[r']\right)\ddoverddext{\chempot[s]}{r}{l}
                                        + \scoef[s,l]\doverdext[r]{\chempot[s]}
                                        + \scoef[s,r]\doverdext[l]{\chempot[s]}\right] \\
\end{split}
\end{equation}
Considering these equations per unit of volume, one obtains:
\begin{equation}
\begin{split}
\Gibbs & = \sum_s\left(\molar[s]^0 + \sum_r\scoef[s,r]\ext[r]\right)\chempot[s] \\
\doverdext[r]{\Gibbs} & = \sum_s \left[\left(\molar[s]^0 + \sum_{r'}\scoef[s,r']\ext[r']\right)\doverdext[r]{\chempot[s]}
                                + \scoef[s,r]\chempot[s]\right] \\
\ddoverddext{\Gibbs}{r}{l} & = \sum_s \left[\left(\molar[s]^0 + \sum_{r'}\scoef[s,r']\ext[r']\right)\ddoverddext{\chempot[s]}{r}{l}
                                        + \scoef[s,l]\doverdext[r]{\chempot[s]}
                                        + \scoef[s,r]\doverdext[l]{\chempot[s]}\right] \\
\end{split}
\end{equation}



\appendix
\section{Demonstrations}
\label{demo}
\subsection{\texorpdfstring{$\frac{1}{\Mm_\text{mix}} = \sum_s \frac{y_s}{\Mm_s}$}{Mixture molar mass}}
\label{demo-Mm}
\[
\begin{split}
\Mm_\text{mix} & = \frac{m_\text{mix}}{c_\text{mix}}
                 = \frac{m_\text{mix}}{\sum_s c_s}
                 = \frac{m_\text{mix}}{\sum_s \frac{m_s}{\Mm_s}}\\
\Rightarrow
\frac{1}{\Mm_\text{mix}}
               & = \frac{\sum_s\frac{m_s}{\Mm_s}}{m_\text{mix}}
                 = \sum_s\frac{\frac{m_s}{\Mm_s}}{m_\text{mix}}
                 = \sum_s\frac{\frac{m_s}{m_\text{mix}}}{\Mm_s}
                 = \sum_s\frac{y_s}{\Mm_s}
\end{split}
\]

\subsection{Equilibrium--inverse rate constant}

for the reaction:
\begin{chemicalEquation}
\ce{\scoefabs[A] A + \scoefabs[B] B <=>[k_1(T,p)][k_{-1}(T,p)] \scoefabs[C] C + \scoefabs[D] D}
\label{eq:equat}
\end{chemicalEquation}

We consider this equilibrium being composed of elemental steps
(%
\ce{\scoefabs[A] A + \scoefabs[B] B -> \scoefabs[C] C + \scoefabs[D] D} and
(\ce{\scoefabs[C] C + \scoefabs[D] D -> \scoefabs[A] A + \scoefabs[B] B}
are elementary processes). 

\subsubsection{Thermo}
\label{demo-eq}

By definition, the equilibrium constant will be
\begin{equation}
K = \prod_s \conc[S]_{\eq}^{\scoef[s]}
\label{eq:Kdef}
\end{equation}
$\conc[S]_{\eq}$ being the concentration of species \ce{S} at equilibrium.
For ideal gas: $p = \conc\Rg T \Rightarrow \conc = \frac{p}{\Rg T}$, with \Rg\ the ideal
gas constant and \conc\ the concentration. Thus
\begin{equation}
\begin{split}
K & = \prod_s \left(\frac{p^{(\eq)}_s}{\Rg T}\right)^{\scoef[s]} \\
  & = \prod_s \left(\frac{p^{(\eq)}_s}{\pz}\frac{\pz}{\Rg T}\right)^{\scoef[s]} \\
  & = \left(\frac{\pz}{\Rg T}\right)^{\sum_s\scoef[s]} \prod_s \left(\frac{p^{(\eq)}_s}{\pz}\right)^{\scoef[s]} \label{eq:K}
\end{split}
\end{equation}
Also, for ideal gas:
\begin{equation}
g(T) = g^0(T) + \Rg T \ln\left(\frac{p}{\pz}\right)
\end{equation}
wich gives
\begin{equation}
\frac{p}{\pz} = \exp\left(\frac{g(T) - g^0(T)}{\Rg T}\right)
\end{equation}
replacing in \ref{eq:K}
\begin{equation}
\begin{split}
K & = \left(\frac{\pz}{\Rg T}\right)^{\sum_s\scoef[s]} \prod_s \exp\left(\scoef[s] \frac{g(T)_s^{(\eq)} - g^0(T)}{\Rg T}\right) \\
  & = \left(\frac{\pz}{\Rg T}\right)^{\sum_s\scoef[s]} \exp\left(\frac{\Delta_rG(T)^{(\eq)}}{\Rg T} - \frac{\Delta_rG^0(T)}{\Rg T}\right) \\
\end{split}
\end{equation}
and, by definition, at equilibrium
\begin{equation}
\Delta_rG(T) = 0
\end{equation}
we obtain, noting $\gamma = \sum_s \scoef[s]$,
\begin{equation}
 K = \left(\frac{\pz}{\Rg T}\right)^\gamma \exp\left(-\frac{\Delta_rG^0(T)}{\Rg T}\right) \\
\end{equation}

\subsubsection{Kinetics}
\label{demo-eqrate}

By definition: $\frac{\dd \conc}{\dd t} = 0$ for any species concerned by the equilibrium.

Thus, for any species \ce{S}: 
$\frac{1}{\scoef[S]}\frac{\dd\conc[S]}{\dd t} =  0$, which can be rewritten as
\begin{equation}
\fwdrate \prodReac = \bkwdrate \prodProd
\label{eq-conccond}
\end{equation}
Developping \ref{eq:Kdef}
\begin{equation}
\Eqconst = \prod_{s}\conc[S]^{\scoef[S]} 
          = \frac{\displaystyle\prodProd}
                 {\displaystyle\prodReac}
\label{eq-defK}
\end{equation}
We deduce, from \ref{eq-conccond} and \ref{eq-defK}
\begin{equation}
\Eqconst = \frac{\fwdrate}{\bkwdrate}
\end{equation}


\section{Program flow}
\label{progflow}
\subsection{Kinetics}
\label{progflow:kinetics}
Figure~\ref{kinpf} describe the data flow from the upper
object \KineticsEvaluator\ to the rate constants objects.
The idea here is to have a light high-level object that will
be fast to build, so easily threadable. All the necessary data
and the bulk of the work is in lower-level objects.
%
\begin{figure}
\centering
\includegraphics{kinetics_relationships}
\caption{\label{kinpf}Global vision of the kinetics in \Antioch.
The gray area frames objects that should not be threaded, whereas
\KineticsEvaluator\ is light and thread-safe. The passed values of
interest are the temperature \prog{const StateType \& T}, the
particle flux \prog{const ParticleFlux<VectorCoeffType> \& pf} and
the composition \prog{const VectorStateType \& dens}.}
\end{figure}
%
\subsection{A classical case}

Let's consider a simple bimolecular reaction, elementary,
with a Kooij kinetics model. The informations we need for
a complete kinetics calculation are the temperature, the
composition and the thermodynamics. The thermodynamics are
calculated in another place of the program, what \Antioch\
will provide will be the Gibbs energy for all the concerned
species (see section~\ref{kinetics_gen}, \textbf{Going backward}
subsection). So at the \KineticsEvaluator\ level, we need
to provide \Antioch's guts with:
\begin{itemize}
\item a \prog{const StateType \& T} for the temperature,
\item a \prog{const VectorStateType \& molar\_densities} for
                the densities,
\item a \prog{const VectorStateType \& h\_RT\_minus\_s\_R} for
                the thermodynamics.
\end{itemize}

The goal is to have the quantity \doverdt{S} for all species \ce{S} concerned by the 
reactive system (see \ref{phys:kinetics_theory_integrated}).

\subsubsection{Data flow}

\paragraph{\KineticsEvaluator}
will gain from \ReactionSet\ the net rates of
the reactions,  i.e. the term \rate[r] in \ref{phys:kinetics_theory_integrated}.
Thus \KineticsEvaluator\ will, for every species, perform the loop
over the concerned reactions and multiply by the stoichiometric coefficient.
Technically, \KineticsEvaluator\ gives \ReactionSet\ a \prog{VectorCoeffType}
to fill with these terms.

\paragraph{\ReactionSet}
will equate the values of the provided \prog{VectorCoeffType} with
the returning value given by the \Reaction. To do that, it passes down to
the \Reaction\ object the temperature, the composition, the thermodynamics and
the value $\frac{\pz}{\Rg \Temp}$, used for the equilibrium constant calculation,
needed for the backward rate constant \eqref{therm:K_therm}.

\paragraph{\Reaction} will asks the \KineticsType\ object (its rate constants) to calculate
the forward rate constant of the reaction (term \fwdratecons[r]) by passing
it the temperature. Then it will be able 
to calculate the backward rate constant (term \bkwdratecons[r]), 
the forward (term \fwdrate[r]) and backward rate (term \bkwdrate[r]), 
and finally the net rate (term \rate[r]).

\paragraph{\KineticsType} will return the forward rate constant \fwdratecons[r] given
the temperature.

\subsubsection{Template flow}

\paragraph{\KineticsEvaluator} asks for a \prog{VectorCoeffType} to be filled. It
provides a \prog{StateType} for the temperature and a \prog{VectorStateType} for
the densities.

\paragraph{\ReactionSet} fills each value of the \prog{VectorCoeffType} with
the returning type of \Reaction.

\paragraph{\Reaction} returns a \prog{StateType}, which is the defined type of the
provided temperature.

\paragraph{\KineticsType} returns a \prog{StateType}, defined by the type of the
provided temperature.

\subsection{A particle flux reaction}

A reaction occuring with a particle flux is another type of reaction than the one
previously considered. Basically, instead of having molecules bumping into each
other and joyfully transforming into other molecules, this kind of reaction is
a molecule shot by particles, and transforming into another molecule(s). A simple
example is photoinduced ionization, a photon will hit a molecule, be absorbed, and
the excedental energy will pop out an electron, thus describing the reaction
\ce{A ->[\mathrm{h}\nu] A+}. In the classical view, this means that the rate constant
is constant, as it does not depend on the temperature. \Antioch\ supports at this
moment only photochemical reaction, but what will be describe here holds for any
particle flux-induced reaction.

The only change so is that instead of passing down a temperature, \KineticsEvaluator,
\ReactionSet\ and \Reaction\ need to pass down a particle flux (\ParticleFlux). The
big difference is the type of the data. A temperature is an atomic type (typically
a \prog{double} or a \prog{float}, while the \ParticleFlux\ is a \ParticleFlux, and
contains a vector type (the full description is \ParticleFlux\prog{<VectorCoeffType>}),
as is needs to store a flux intensity and a grid (wavelength, energy, \dots) corresponding
to those intensity. A photon flux will be characterized by an intensity per wavelength.

Thus the program flow stays the same, just replace the temperature with the particle
flux. Where it gets tricky is at the \Reaction\ level, where it returns the net reaction
rate value. Two things to consider here: a reaction occuring through particle flux
chemistry is irreversible in nature, so no backward rate consideration (nor thermodynamics),
we need to extract a \prog{StateType} out of a \ParticleFlux\prog{<VectorCoeffType>}.
The concerned method is (templates left out)
\begin{center}
\verb!StateType Reaction::compute_rate_of_progress() const!
\end{center}
with \prog{StateType} provided by the passed temperature. In this case however, we but
need the molecular densities (\prog{VectorStateType}) and the \ParticleFlux. So the
question is, how do we get the \prog{StateType} type?

Several possibilities here:
\begin{enumerate}
\item\label{type_pass} we pass down the temperature anyway, just for the type;
\item\label{meta_paradise} heck man, meta computing is so great, let's just refactore those method!
\end{enumerate}

The method \ref{type_pass} is simple, we would need to overload the methods:
\begin{verbatim}
  template<typename CoeffType, typename VectorCoeffType>
  template <typename StateType, typename VectorStateType>
  inline
  StateType Reaction<CoeffType,VectorCoeffType>::compute_rate_of_progress( 
                const VectorStateType& molar_densities,
                const StateType& T,  
                const StateType& P0_RT,  
                const VectorStateType& h_RT_minus_s_R) const;
\end{verbatim}
with
\begin{verbatim}
  template<typename CoeffType, typename VectorCoeffType>
  template <typename StateType, typename VectorStateType>
  inline
  StateType Reaction<CoeffType,VectorCoeffType>::compute_rate_of_progress( 
                const VectorStateType& molar_densities,
                const ParticleFlux<VectorStateType>& pf,
                const StateType & /*T*/) const;
                
\end{verbatim}
and so forth down to \KineticsType.

Method \ref{meta_paradise} would require a little bit more perversion, but no
overloading. Basically the idea would be to have the definition
\begin{verbatim}
  template<typename CoeffType, typename VectorCoeffType>
  template <typename DataType, typename VectorStateType, typename StateType>
  inline
  typedef typename Antioch::returned_type<DataType>::type
        Reaction<CoeffType,VectorCoeffType>::compute_rate_of_progress( 
                const VectorStateType& molar_densities,
                const DataType& T,  
                const StateType& P0_RT,  
                const VectorStateType& h_RT_minus_s_R) const;
\end{verbatim}
with the meta function
\begin{verbatim}
  template <typename T, typename Enable=void>
  struct returned_type;
\end{verbatim}
that would be defined as
\begin{verbatim}
template <typename T>
struct returned_type<T>
{
  typedef T type;
};
\end{verbatim}
for all there is (eigen, metaphysicl, valarray, vector, vexcl, ANTIOCH\_PLAIN\_SCALAR), and
as
\begin{verbatim}
template <typename T>
struct returned_type<ParticleFlux<T> >
{
  typedef typename Antioch::value_type<T>::type type;
};
\end{verbatim}
for the particle flux.


\end{document}
