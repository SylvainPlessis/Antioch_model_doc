\subsection{The code}

Let's face it, \Antioch\ is mainly about calculating
efficiently, with elegance and a sexy attitude, rates,
be them with respect to mass or molar concentration. 

The first thing to clear out is the template choice. 
\Antioch\ is very supple about this, in this example
we will consider a simple choice: a \double\ for a 
scalar and a \stdvector\ for a vector.

\linenumbers
\begin{verbatim}
// C++
#include <string>
#include <vector>

// Antioch
#include "antioch/vector_utils.h"

#include "antioch/antioch_asserts.h"
#include "antioch/chemical_species.h"
#include "antioch/chemical_mixture.h"
#include "antioch/reaction_set.h"
#include "antioch/read_reaction_set_data_xml.h"
#include "antioch/cea_thermo.h"
#include "antioch/kinetics_evaluator.h"

int compute_rates(const std::string& input_name,
  std::vector<double> & omega_dot,
  std::vector<double> & domega_dot_dT,
  std::vector<std::vector<double> > & domega_dot_drho_s)
{
  std::vector<std::string> species_str_list;
  const unsigned int n_species = 5;
  species_str_list.reserve(n_species);
  species_str_list.push_back( "N2" );
  species_str_list.push_back( "O2" );
  species_str_list.push_back( "N" );
  species_str_list.push_back( "O" );
  species_str_list.push_back( "NO" );

  Antioch::ChemicalMixture<double> chem_mixture( species_str_list );
  Antioch::ReactionSet<double> reaction_set( chem_mixture );
  Antioch::CEAThermodynamics<double> thermo( chem_mixture );

  Antioch::read_reaction_set_data_xml<double>( input_name, true, 
                                               reaction_set );

  const double T = 1500.0; // K
  const double P = 1.0e5; // Pa

  // Mass fractions
  std::vector<double> Y(n_species,0.2);

  const double R_mix = chem_mixture.R(Y); //get R_tot in J.kg-1.K-1

  const double rho = P/(R_mix*T); // kg.m-3

  std::vector<double> molar_densities(n_species,0.0);
  chem_mixture.molar_densities(rho,Y,molar_densities);

  std::vector<double> h_RT_minus_s_R(n_species);
  std::vector<double> dh_RT_minus_s_R_dT(n_species);

  typedef typename Antioch::CEAThermodynamics<double>::template 
                Cache<double> Cache;
  thermo.h_RT_minus_s_R(Cache(T),h_RT_minus_s_R);
  thermo.dh_RT_minus_s_R_dT(Cache(T),dh_RT_minus_s_R_dT);

  Antioch::KineticsEvaluator<double> kinetics( reaction_set, 0 );

  omega_dot.resize(n_species,0.L);
  domega_dot_dT(n_species,0.L);
  domega_dot_drho_s(n_species);
  for( unsigned int s = 0; s < n_species; s++ )
    {
      domega_dot_drho_s[s].resize(n_species);
    }
  
  kinetics.compute_mass_sources( T, molar_densities, 
                                 h_RT_minus_s_R, omega_dot);

  kinetics.compute_mass_sources_and_derivs( T, molar_densities, 
                                            h_RT_minus_s_R, dh_RT_minus_s_R_dT,
                                            omega_dot, domega_dot_dT, 
                                            domega_dot_drho_s );
  return 0;
}

int main(int argc, char* argv[])
{
  // Check command line count.
  if( argc < 2 )
    {
      std::cerr << "Error: Must specify reaction set XML input file." 
                << std::endl;
      antioch_error();
    }

  std::vector<double> omega_dot;
  std::vector<double> domega_dot_dT;
  std::vector<std::vector<double> > domega_dot_drho_s;

  compute_rate(std::string(argv[1],omega_dot,
                                   domega_dot_dT,
                                   domega_dot_drho_s))

  //do stuff with the omega_dot s vectors and matrix

  return 0;
}
\end{verbatim}
\nolinenumbers

Chemistry happens in a chemical mixture. Thus,
we need to define our chemical system. The appropriate object is a
\ChemicalMixture, defined in file \file{chemical\_mixture.h} 
(lines~10 and 30). We also
need a place to manage the reactions, this is the \ReactionSet, defined
in the file \file{reaction\_set.h} (lines~11 and 31). To calculate
the backward rate constants, as seen in subsection~\ref{phys:equilibrium},
we need the thermodynamics of the species. \Antioch\ uses the object
\CEAThermodynamics, defined in file \file{cea\_thermo.h}. Finally, the
object that controls the kinetics is the \KineticsEvaluator, defined
in the file \file{kinetics\_evaluator.h} (lines~14 and 58).
%
\begin{table}
\centering
\begin{tabular}{llc}\toprule
\hfil Objects      & \hfil Header                 & Lines \\\midrule
\ChemicalMixture   & \file{chemical\_mixture.h}   & 10, 30\\
\ReactionSet       & \file{reaction\_set.h}       & 11, 31\\
\CEAThermodynamics & \file{cea\_thermo.h}         & 13, 32\\
\KineticsEvaluator & \file{kinetics\_evaluator.h} & 14, 58\\
\bottomrule
\end{tabular}
\caption{\label{tuto:kin_reg_step}Main objects, their locations and the 
        corresponding lines in the given example program.}
\end{table}
%
These informations are summarized in Tab.~\ref{tuto:kin_reg_step}. We will
now look into each of those object in more details.

\subsection{A \ChemicalMixture}
\label{tuto:ChemMix}

For performance reasons, \Antioch\ has all the species she knows and their
characteristics hard-coded\footnote{May evolve in the futur, knowing that
performance is the main issue.}. Thus to define a chemical mixture, all
is needed is to point out to \Antioch\ the species we want. It is done
by giving \emph{at the construction step} a \stdvector\ of \stdstring\
containing the names of the desired species (lines~21 to 30).
It takes care of defining everything that's needed to find efficiently
any data on the provided list of species.

This object will provide the information needed of a mixture, provided
a composition. Basically you'll need to get the molar composition
at the end, if only because this is what the \KineticsEvaluator\ wants
(see below, subsection~\ref{tuto:KineticsEval}). As some of us are
not enough chemist too much physicist, instead of having directly
a molar composition, it can happen that you have a mass composition
instead of a molar composition\footnote{Life is hard, but \Antioch\
will help you get over this one.}. Lines 40--48 illustrate this,
the steps are as follow:
\begin{enumerate}
\item obtain the mixture mass ideal gas constant \Rmix\ 
        in \unit{J\,kg^{-1}\,K^{-1}} (line~43);
\item calculate the mass density of the mixture in \unit{kg\,m^{-3}}
        (line~45);
\item ask for the molar composition (line~48).
\end{enumerate}


\subsection{A \ReactionSet}

A \ReactionSet\ acts on a mixture by means of a chemical reactions
network. The mixture is defined \emph{at construction time} by
providing the \ChemicalMixture\ concerned, the chemical network
is added afterward, using the method 
\verb!Antioch::read_reaction_set_data_xml!, defined in the file
\file{read\_reaction\_set\_data\_xml.h} (lines~12 and 34). This
method takes three arguments, the xml file where the reactions are
to be read, a boolean for verbosity purposes: true if you want the
method to tell you the reactions, false if you don't; and the \ReactionSet\ 
object to store the chemical network.

\subsection{A \CEAThermodynamics}
\label{tuto:CEATherm}

As previously said, everything is hard-coded in \Antioch, thus
the only data a \CEAThermodynamics\ needs is the chemical
mixture, still \emph{at construction time}.

This object will compute the NASA polynoms for the wanted
species at a given temperature. There we have a very interesting
template shims: the NASA polynoms (Eq.~\ref{therm:DH} and \ref{therm:DS}
in subsection~\ref{phys:NASA_therm}) have logarithms in their
definitions, which is slow to calculate. Thus to store $\ln(\Temp)$
would be a gain of time. This is the story of lines~53 and 54. This
\object{Cache} enables a gain of time by caching the value
of $\ln(\Temp)$. The fancy of line~53 is needed to manage
properly the template capacities of \Antioch.
Once the \object{Cache} is created, \CEAThermodynamics\ will
calculate the necessary thermodynamics, as shown in lines~55
and 56.

\subsection{A \KineticsEvaluator}
\label{tuto:KineticsEval}

This is the top-level object, where you asks for
answers once all is set. Basically, it computes
rates, derivatives with respect to temperature,
molar or mass concentrations given a composition in
moles, the thermodynamics value $\frac{\gibbs}{\Rg\Temp}$
for each species, and the temperature.
These are lines~68, 69 and 71--74.

So you need to be clear on the three steps to use its power:
\begin{itemize}
\item a temperature (line~37);
\item a molar composition (see the \ChemicalMixture\ object
        at subsection~\ref{tuto:ChemMix}, lines~40--48);
\item a thermodynamics description (see the \CEAThermodynamics\
        object, lines~50--56).
\end{itemize}

\subsection{A note on the \object{resize}}

\Antioch\ is meant to be a passing in the lives of
the wanted quantities, thus all the calculated values
are calculated and stored in a array given by the
user. Arises there the question of who gives the vector
the required size? \Antioch's answer is the user. Thus
before using \Antioch's capacity, the storage of the
wanted quantities must be created and properly sized,
as seen on the lines~47, 50, 51, 60--66 and 88--90.
