What we want is 
$\mdot[S] = \doverdt{\mass[S]}$, and its derivatives:
$\doverdT{\mdot[S]}$, $\doverdm[j]{\mdot[S]}$ with respect to species $j$.

The kinetics model's value is noted \kinMod, the rate constant is noted \rateCons, the
rate of the reaction is noted \rate.

The kinetics model is Kooij 
\begin{equation}
\kinMod = \KooijEq,
\label{KooijEq}
\end{equation}
the chemical process are elementary 
\begin{equation}
\rateCons = \kinMod
\label{ChemProEl}
\end{equation} 
or three-body 
\begin{equation}
\rateCons = \kinMod \threeBody.
\label{ChemProTB}
\end{equation}

We use the equations
\begin{equation}
\mass[S] = \Mm[S] \conc[S]
\end{equation}
\begin{equation}
\dd\mass[S] = \Mm[S]\,\dd\conc[S]
\end{equation}
as everything in kinetics theory is expressed in concentrations and 
not in terms of mass.

We have thus:
\begin{equation}
\mdot[S] = \doverdt{\mass[S]} = \Mm[S] \doverdt{\conc[S]}
\end{equation}
\begin{equation}
\doverdm[E]{\mdot[S]} = \frac{\Mm[S]}{\Mm[E]}\doverdc[E]{\conc[S]}
\end{equation}
\begin{equation}
\doverdT{\mdot[S]} = \Mm[S]\doverdT{\conc[S]}
\end{equation}
but for Paul's sake, we will do the full derivation with mass.

\subsection{Forward}

The forward rate \fwdrate\ for a reaction is by definition for species \ce{S}:
\begin{equation}
\begin{split}
\fwdrate &= \frac{1}{\scoef[S]}\frac{\dd\conc[S]}{\dd t}\\
         &= \frac{1}{\scoef[S]} \fwdratecons \prodReac
\end{split}
\label{ratefDef}
\end{equation}

\paragraph{To derive with respect to \Temp.}
For any species \ce{S} participating in the reaction\footnote{remember, 
$\scoef = \left\{\begin{array}{l}-\scoefabs\text{ for reactants}\\\scoefabs\text{ for products}\end{array}\right.$}.
thus,
\begin{equation}
\doverdT{\fwdrate} = \frac{1}{\scoef[S]}
                   \doverdT{\fwdratecons}
                   \prodReac
\end{equation}
and, for the elementary process, following~\ref{ChemProEl} and \ref{KooijEq}:
\begin{equation}
\begin{split}
\doverdT{\fwdratecons} & = \doverdT{\kinMod} \\
                       & = \frac{\kinMod}{\Temp} \left(\frac{E_a}{\Rg \Temp} + \beta\right)
\end{split}
\end{equation}
For the three-body process, following~\ref{ChemProTB} and \ref{KooijEq}:
\begin{equation}
\begin{split}
\doverdT{\fwdratecons} & = \doverdT{\kinMod} \threeBody \\
                       & = \frac{\kinMod}{\Temp} \left(\frac{E_a}{\Rg \Temp} + \beta\right) \threeBody
\end{split}
\end{equation}

Finally,
for the elementary processes
\begin{equation}
\doverdT{\fwdrate} = \frac{1}{\scoef[S]} \frac{\kinMod}{\Temp} \left(\frac{E_a}{\Rg \Temp} + \beta\right)
                                                \prodReac
\label{derivTEP}
\end{equation}
and the three-body processes
\begin{equation}
\doverdT{\fwdrate} = \frac{1}{\scoef[S]} \frac{\kinMod}{\Temp} \left(\frac{E_a}{\Rg \Temp} + \beta\right) \threeBody
                                                \prodReac
\label{derivTTB}
\end{equation}

\paragraph{To derive with respect to \mass.}
Knowing that $\conc = \frac{\mass}{\Mm}$, we substitute in~\ref{ratefDef}
\begin{equation}
\fwdrate = \frac{1}{\scoef[S]} \fwdratecons \prod_\text{reactants}  \left(\frac{\mass[\reac]}{\Mm[\reac]}\right)^{\scoefabs[\reac]}
\end{equation}
We obtain for a derivation with respect to \mass[E]\ of species \ce{E}:
\begin{equation}
\doverdm[E]{\fwdrate} = \fwdrate \left[
                                \doverdm[E]{\fwdratecons} \frac{1}{\fwdratecons} +
                                \underbrace{\frac{\scoefabs[E]}{\mass[E]}}_{\text{if \ce{E} is a reactant}}
                              \right]
\end{equation}

For elementary process:
\begin{equation}
\doverdm[E]{\fwdratecons} = 0
\end{equation}
For a three-body process:
\begin{equation}
\doverdm[E]{\fwdratecons} = \kinMod \frac{\epsilon_{\ce{E}}}{\Mm[E]}
\end{equation}

Finally, for an elementary process:
\begin{equation}
\doverdm[E]{\fwdrate} = \left\{\begin{array}{ll}
                        \fwdrate \frac{\scoefabs[E]}{\mass[E]} & \text{if \ce{E} is a reactant} \\
                        0                                    & \text{if \ce{E} is not a reactant} \\
                      \end{array}\right.
\end{equation}
and for a three-body process:
\begin{equation}
\renewcommand{\arraystretch}{1.5}
\doverdm[E]{\fwdrate} = \left\{\begin{array}{ll}
                        \fwdrate \left[\frac{\epsilon_{\ce{E}}}{\Mm[E]\threeBody} + \frac{\scoefabs[E]}{\mass[E]} \right] 
                                                & \text{if \ce{E} is a reactant} \\
                        \fwdrate \frac{\epsilon_{\ce{E}}}{\Mm[E]\threeBody} 
                                                & \text{if \ce{E} is not a reactant} \\
                      \end{array}\right.
\end{equation}

\subsection{Backward}

The backward rate is
\begin{equation}
\bkwdrate = \frac{1}{\scoef[S]}\bkwdratecons\prodProd
\end{equation}

The backward rate constant is given by
\begin{equation}
\bkwdratecons = \frac{\fwdratecons}{\Eqconst}
\end{equation}

\paragraph{Derive with respect to \Temp.}
It makes:
\begin{equation}
\doverdT{\bkwdrate} = \frac{1}{\scoef[S]}\prodProd\doverdT{\bkwdratecons}
\end{equation}
Decomposition gives
\begin{equation}
\doverdT{\bkwdratecons} = \doverdT{\fwdratecons}\Eqconst^{-1} - \doverdT{\Eqconst}\frac{\bkwdratecons}{\Eqconst^{2}}
\label{rateb-decomp}
\end{equation}
For \Eqconst, we have
\begin{equation}
\begin{split}
\doverdT{\Eqconst} & = \doverdT{\left[\left(\frac{\pz}{\Rg \Temp}\right)^{\sum_s \scoef[s]} \exp\left(-\frac{\DGibbsZ(\Temp)}{\Rg \Temp}\right)\right]} \\
                   & = \Eqconst\left[-\frac{\sum_s\scoef[s]}{\Temp} - \doverdT{\left[\frac{\DGibbsZ(\Temp)}{\Rg \Temp}\right]}\right]
\end{split}
\end{equation}
The derived value of \DGibbsZ(\Temp) is easily given, see~\ref{data-thermo}
\begin{equation}
\begin{split}
\doverdT{\left[\frac{\DGibbsZ(\Temp)}{\Rg \Temp}\right]} 
        &= \doverdT{\left[\frac{\DenthZ(T)}{\Rg \Temp}\right]} - \doverdT{\left[\frac{\DenthZ(\Temp)}{\Rg \Temp}\right]} \\
        &= 2\tc{0}\Temp^{-3} + \tc{1}\Temp^{-2}\left(1+\ln(\Temp)\right) + \frac{\tc{3}}{2} + 
           \frac{2}{3}\tc{4}\Temp + \frac{3}{4}\tc{5}\Temp^{2} + \frac{4}{5}\tc{6}\Temp^3 - \tc{8}\Temp^{-2}\\
        &+ \tc{0}\Temp^{-3} + \tc{1}\Temp^{-2} + \tc{2}\Temp^{-1} + \tc{3} + \tc{4}\Temp + \tc{5} \Temp^{2} + \tc{6} \Temp^{3}
\end{split}
\end{equation}
Finally,
\begin{equation}
\doverdT{\left[\frac{\DGibbsZ(\Temp)}{\Rg \Temp}\right]} =
        3\tc{0}\Temp^{-3} + \tc{1} \Temp^{-2} \left(2 + \ln(\Temp)\right) + \frac{3}{2} \tc{3} + \frac{5}{3} \tc{4} \Temp
        + \frac{7}{4}\tc{5} \Temp^2 + \frac{9}{5}\tc{6} \Temp^3 - \tc{8} \Temp^{-2}
\end{equation}
\emph{Note:} $\ln(\Temp)$ should be understood as $\ln\left(\frac{\Temp}{\mathrm{\Temp_{ref}}}\right)$ with 
$\mathrm{\Temp_{ref}} = 1$~K.

Thus, for elementary processes
\begin{equation}
\begin{split}
\doverdT{\bkwdrate} = & \frac{1}{\scoef[S]} \prodProd \\
&\bigg[
        \frac{\kinMod}{\Temp} \left(\frac{E_a}{\Rg \Temp} + \beta\right)\prodReac \Eqconst^{-1}
                  - 3\tc{0}\Temp^{-3} - \tc{1} \Temp^{-2} \left(2 + \ln(\Temp)\right) - \frac{3}{2} \tc{3} \\
&                 - \frac{5}{3} \tc{4} \Temp - \frac{7}{4}\tc{5} \Temp^2 - \frac{9}{5}\tc{6} \Temp^3 + \tc{8} \Temp^{-2} \bigg]
\end{split}
\end{equation}
and for third-body processes
\begin{equation}
\begin{split}
\doverdT{\bkwdrate} = & \frac{1}{\scoef[S]} \prodProd \\
&\bigg[
        \frac{\kinMod}{\Temp} \left(\frac{E_a}{\Rg \Temp} + \beta\right) \threeBody \Eqconst^{-1} 
                  - 3\tc{0}\Temp^{-3} - \tc{1} \Temp^{-2} \left(2 + \ln(\Temp)\right) - \frac{3}{2} \tc{3} \\
&                 - \frac{5}{3} \tc{4} \Temp - \frac{7}{4}\tc{5} \Temp^2 - \frac{9}{5}\tc{6} \Temp^3 + \tc{8} \Temp^{-2} \bigg]
\end{split}
\end{equation}

\paragraph{Derivation with respect to mass.}
Using~\ref{rateb-decomp}
\begin{equation}
\begin{split}
\doverdm[E]{\bkwdratecons} & = \doverdm[E]{\fwdratecons}\Eqconst^{-1} - \doverdm[E]{\Eqconst}\frac{\bkwdratecons}{\Eqconst^{2}} \\
                           & = \doverdm[E]{\fwdratecons}\Eqconst^{-1}
\end{split}
\label{rateb-decomp-m}
\end{equation}
thus, for elementary processes
\begin{equation}
\doverdm[E]{\rate} = \left\{\begin{array}{ll}
                        \fwdrate \frac{\scoefabs[E]}{\mass[E]} \Eqconst^{-1} & \text{if \ce{E} is a reactant} \\
                          0                                                  & \text{if \ce{E} is not a reactant} \\
                      \end{array}\right.
\end{equation}
and for three-body processes
\begin{equation}
\renewcommand{\arraystretch}{1.5}
\doverdm[E]{\fwdrate} = \left\{\begin{array}{ll}
                        \fwdrate \left[\frac{\epsilon_{\ce{E}}}{\Mm[E]\threeBody} + \frac{\scoefabs[E]}{\mass[E]} \right] \Eqconst^{-1}
                                                & \text{if \ce{E} is a reactant} \\
                        \fwdrate \frac{\epsilon_{\ce{E}}}{\Mm[E]\threeBody} \Eqconst^{-1}
                                                & \text{if \ce{E} is not a reactant} \\
                      \end{array}\right.
\end{equation}

\subsection{Net rate}

The net rate is
\begin{equation}
\rate = \fwdrate - \bkwdrate
\end{equation}
thus,
\begin{equation}
\doverdT{\rate} = \doverdT{\fwdrate} - \doverdT{\bkwdrate}
\end{equation}
and
\begin{equation}
\doverdm{\rate} = \doverdm{\fwdrate} - \doverdm{\bkwdrate}
\end{equation}
thus, for elementary processes
\begin{equation}
\begin{split}
\doverdT{\rate[{\text{net}}]} &= \frac{1}{\scoef[S]} \frac{\kinMod}{\Temp} \left(\frac{E_a}{\Rg \Temp} + \beta\right) \prodReac\\
& -
\frac{1}{\scoef[S]} \prodProd 
\bigg[
        \frac{\kinMod}{\Temp} \left(\frac{E_a}{\Rg \Temp} + \beta\right)\prodReac \Eqconst^{-1} \\
                  & \hspace{3cm} - 3\tc{0}\Temp^{-3} - \tc{1} \Temp^{-2} \left(2 + \ln(\Temp)\right) - \frac{3}{2} \tc{3} \\
                  & \hspace{3cm} - \frac{5}{3} \tc{4} \Temp - \frac{7}{4}\tc{5} \Temp^2 - \frac{9}{5}\tc{6} \Temp^3 + \tc{8} \Temp^{-2} \bigg]
\end{split}
\end{equation}
and for three-body processes
\begin{equation}
\begin{split}
\doverdT{\rate[{\text{net}}]} & = \frac{1}{\scoef[S]} \frac{\kinMod}{\Temp} \left(\frac{E_a}{\Rg \Temp} + \beta\right) \threeBody\prodReac\\
& -
\frac{1}{\scoef[S]} \prodProd 
\bigg[
        \frac{\kinMod}{\Temp} \left(\frac{E_a}{\Rg \Temp} + \beta\right)\threeBody\prodReac \Eqconst^{-1} \\
                  & \hspace{3cm} - 3\tc{0}\Temp^{-3} - \tc{1} \Temp^{-2} \left(2 + \ln(\Temp)\right) - \frac{3}{2} \tc{3} \\
                  & \hspace{3cm} - \frac{5}{3} \tc{4} \Temp - \frac{7}{4}\tc{5} \Temp^2 - \frac{9}{5}\tc{6} \Temp^3 + \tc{8} \Temp^{-2} \bigg]
\end{split}
\end{equation}
and for the mass, elementary processes
\begin{equation}
\doverdm[E]{\rate} = \left\{\begin{array}{ll}
                        \fwdrate \frac{\scoefabs[E]}{\mass[E]}\left(1-\Eqconst^{-1}\right) & \text{if \ce{E} is a reactant} \\
                          0                                                                & \text{if \ce{E} is not a reactant} \\
                      \end{array}\right.
\end{equation}
and for the three-body processes
\begin{equation}
\renewcommand{\arraystretch}{1.5}
\doverdm[E]{\rate} = \left\{\begin{array}{ll}
                        \fwdrate \left[\frac{\epsilon_{\ce{E}}}{\Mm[E]\threeBody} + \frac{\scoefabs[E]}{\mass[E]} \right] \left(1 - \Eqconst^{-1}\right)
                                                & \text{if \ce{E} is a reactant} \\
                        \fwdrate \frac{\epsilon_{\ce{E}}}{\Mm[E]\threeBody}  \left(1 - \Eqconst^{-1}\right)
                                                & \text{if \ce{E} is not a reactant} \\
                      \end{array}\right.
\end{equation}

Back to \mdot:
\begin{equation}
\begin{split}
\mdot[S] &= \Mm[S] \doverdt{\ce{S}} \\
         &= \Mm[S] \scoef[S]\rate
\end{split}
\end{equation}

\begin{equation}
\renewcommand{\arraystretch}{1.5}
\left\{\begin{array}{l}
\doverdT{\mdot[S]} = \Mm[S] \scoef[S]\doverdT{\rate}\\
\doverdm[E]{\mdot[S]} = \Mm[S] \scoef[S]\doverdm[E]{\rate}
\end{array}\right.
\end{equation}

