\subsection{With kinetics}
An equilibrium (more rigorously a steady state), is defined
by
\begin{equation}
\forall\:s,\quad \doverdt{\conc[s]} = 0
\label{equilibrium:def}
\end{equation}
Thus, as \antioch\ is all about masses, we will consider
the equivalent:
\begin{equation}
\forall\:s,\quad \doverdt{\mass[s]} = \mdot[s] = 0
\label{equilibrium:def_by_mass}
\end{equation}
To have the detailed equations of the kinetics, see
section~\ref{derivations}.

The system to be solved is of the form:
\begin{equation}
A\times x = b
\end{equation}
with $b$ the vector of \mdot, $A$ the matrixes of \doverdm[E]{\mdot[s]} for
the species (rows are $s$ and columns are $E$) and $x$ the vector of the solution \mass.
To close the system, we use the mass conservation equation and use a
species to ensure it:
\begin{equation}
\sum_s \mass[s] = \mathrm{mass_{tot}}
\label{mass_cons}
\end{equation}
with $\mathrm{mass_{tot}}$ being a constant, here the density of mass of the system.

So, for $N$ chemical species, we have the system:
\begin{equation}
\left[\begin{array}{cccc}
\doverdm[s_1]{\mdot[s_1]}     & \doverdm[s_2]{\mdot[s_1]}     & \cdots & \doverdm[s_N]{\mdot[s_1]} \\
\doverdm[s_1]{\mdot[s_2]}     & \doverdm[s_2]{\mdot[s_2]}     & \cdots & \doverdm[s_N]{\mdot[s_2]} \\
\vdots                        & \vdots                        & \vdots & \vdots                    \\
\doverdm[s_1]{\mdot[s_{N-1}]} & \doverdm[s_2]{\mdot[s_{N-1}]} & \cdots & \doverdm[s_N]{\mdot[s_{N-1}]}\\
1                             & 1                             & \cdots & 1\\
\end{array}\right]
\left[\begin{array}{c}
\Delta\mass[s_1]\\
\Delta\mass[s_2]\\
\vdots\\
\Delta\mass[s_N]\\
\end{array}\right]
=
\left[\begin{array}{c}
\mdot[s_1]\\
\mdot[s_2]\\
\mdot[s_1]\\
\vdots\\
\mdot[s_{N_1}]\\
\sum_{s=1}^N\mass[s] - \mathrm{mass_{tot}}
\end{array}\right]
\label{eq:matrixes}
\end{equation}
The total fixed mass is calculated thanks to the ideal gas state equation
(see section~\ref{relations})
\begin{equation}
\mathrm{mass_{tot}} = \Mm[\mathrm{mix}] \frac{P}{\Rg T}
\label{tot_mass}
\end{equation}
with \Mm[\mathrm{mix}] calculated as seen in section~\ref{relations}.
Thus an initial guess of \massfrac\ is necessary.
If you don't have any idea, let's consider the situation
\begin{chemicalEquation}
\ce{A + B ->[k_1] C + D ->[k_2] E + F}
\label{youpi}
\end{chemicalEquation}
we have
\begin{equation}
\doverdt{\conc[C]} = k_1\conc[A]\conc[B] - k_2\conc[C]\conc[D]
\end{equation}
therefore, a first approximation can be
\begin{equation}
\conc[C^{(\text{approx})}] = \frac{k_1\conc[A]\conc[B]}{k_2\conc[D]} = \frac{\mathrm{prod}}{\mathrm{loss}}
\end{equation}

This will be efficient in somewhat easy situations, meaning you're looking for the
steady state of minor species for instance.

\subsection{With thermodynamics}
A thermodynamic phase is at equilibrium for a minimized Gibbs energy, with
the relation
\begin{equation}
\dd\Gibbs = \Vol\dd\Press - \Entr\dd\Temp + \sum_s\chempot[s]\dd\Mol[s]
\end{equation}
and we have Euler's identity
\begin{equation}
\Gibbs = \sum_s \Mol[s]\chempot[s]
\end{equation}
So, for $R$ chemical reactions, we have the system:
\begin{equation}
\left[\begin{array}{cccc}
\ddoverddext{\Gibbs}{1}{1}   & \ddoverddext{\Gibbs}{1}{2}   & \cdots & \ddoverddext{\Gibbs}{1}{R} \\
\ddoverddext{\Gibbs}{2}{1}   & \ddoverddext{\Gibbs}{2}{2}   & \cdots & \ddoverddext{\Gibbs}{2}{R} \\
\vdots                       & \vdots                       & \vdots & \vdots                    \\
\ddoverddext{\Gibbs}{R-1}{1} & \ddoverddext{\Gibbs}{R-1}{2} & \cdots & \ddoverddext{\Gibbs}{R-1}{R} \\
1                            & 1                            & \cdots & 1\\
\end{array}\right]
\left[\begin{array}{c}
\Delta\ext[1]\\
\Delta\ext[2]\\
\vdots\\
\Delta\ext[R]\\
\end{array}\right]
=
\left[\begin{array}{c}
\doverdext[1]{\Gibbs}\\
\doverdext[2]{\Gibbs}\\
\vdots\\
\doverdext[R-1]{\Gibbs}\\
0
\end{array}\right]
\label{eq:matrixes_xi}
\end{equation}
Basically:
\begin{equation}
\DGibbs = \DGibbsZ + \sum_s\Rg \Temp \ln\left(\frac{p_s}{\pz}\right)
\end{equation}
The story behind chemical extent is:
\begin{equation}
\Mol[s] = \Mol[s](t=0) + \sum_r \scoef[s,r] \ext[r]
\end{equation}
using ideal gas equation:
\begin{equation}
\Press \Vol = \Mol \Rg \Temp
\end{equation}
\begin{equation}
\DGibbs = \DGibbsZ + \sum_s\Rg \Temp \ln\left(\frac{\Rg \Temp}{\pz}
                                                \left[\conc_s(t=0) +
                                                           \sum_r\scoef[s,r]\frac{\ext[r]}{\Vol}
                                                \right]
                                        \right)
\end{equation}
so
\begin{equation}
\doverdext[r]{\DGibbs} = \Rg\Temp \sum_s\frac{\frac{\scoef[s,r]}{\Vol}}
                                             {\conc_s(t=0) + \sum_{r'} \scoef[s,r']\frac{\ext[r']}{\Vol}}
\end{equation}
and
\begin{equation}
\ddoverddext{\DGibbs}{r}{l} = -\Rg\Temp \sum_s\frac{\scoef[s,r]}{\Vol}\frac{\scoef[s,l]}{\Vol}
                                        \left[\conc_s(t=0) + \sum_{r'}\scoef[s,r']\frac{\ext[r']}{\Vol}\right]^{-2}
\end{equation}
now to simplify we will consider the extent by unit of volume ($\frac{\ext}{\Vol}$):
\begin{equation}
\begin{split}
\DGibbs & =  \DGibbsZ + \Rg\Temp \sum_s \ln\left(\frac{\Rg\Temp}{\pz}\left[\conc_s(t=0) + \sum_r\scoef[s,r]\ext[r]\right]\right) \\
\doverdext[r]{\DGibbs} & = \Rg\Temp \sum_s\frac{\scoef[s,r]}{\conc_s(t=0) + \sum_{r'} \scoef[s,r']\ext[r']} \\
\ddoverddext{\DGibbs}{r}{l} & =  -\Rg\Temp \sum_s\scoef[s,r]\scoef[s,l]\left[\conc_s(t=0) + \sum_{r'}\scoef[s,r']\ext[r']\right]^{-2}
\end{split}
\end{equation}
