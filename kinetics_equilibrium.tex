An equilibrium (more rigorously a steady state), is defined
by
\begin{equation}
\forall\:s,\quad \doverdt{\conc[s]} = 0
\label{equilibrium:def}
\end{equation}
Thus, as \antioch\ is all about masses, we will consider
the equivalent:
\begin{equation}
\forall\:s,\quad \doverdt{\mass[s]} = \mdot[s] = 0
\label{equilibrium:def_by_mass}
\end{equation}
To have the detailed equations of the kinetics, see
section~\ref{derivations}.

The system to be solved is of the form:
\begin{equation}
A\times x = b
\end{equation}
with $b$ the vector of \mdot, $A$ the matrixes of \doverdm[E]{\mdot[s]} for
the species (rows are $s$ and columns are $E$) and $x$ the vector of the solution \mass.
To close the system, we use the mass conservation equation and use a
species to ensure it:
\begin{equation}
\sum_s \mass[s] = \mathrm{mass_{tot}}
\label{mass_cons}
\end{equation}
with $\mathrm{mass_{tot}}$ being a constant, here the mass of the system
per unit of volume (thus $\rho$).

So, for $N$ chemical species, we have the system:
\begin{equation}
\left[\begin{array}{cccc}
\doverdm[s_1]{\mdot[s_1]}     & \doverdm[s_2]{\mdot[s_1]}     & \cdots & \doverdm[s_N]{\mdot[s_1]} \\
\doverdm[s_1]{\mdot[s_2]}     & \doverdm[s_2]{\mdot[s_2]}     & \cdots & \doverdm[s_N]{\mdot[s_2]} \\
\vdots                        & \vdots                        & \vdots & \vdots                    \\
\doverdm[s_1]{\mdot[s_{N-1}]} & \doverdm[s_2]{\mdot[s_{N-1}]} & \cdots & \doverdm[s_N]{\mdot[s_{N-1}]}\\
1                             & 1                             & \cdots & 1\\
\end{array}\right]
\left[\begin{array}{c}
\mass[s_1]\\
\mass[s_2]\\
\vdots\\
\mass[s_N]\\
\end{array}\right]
=
\left[\begin{array}{c}
\mdot[s_1]\\
\mdot[s_2]\\
\mdot[s_1]\\
\vdots\\
\mdot[s_{N_1}]\\
\sum_{s=1}^N\mass[s] - \mathrm{mass_{tot}}
\end{array}\right]
\label{eq:matrixes}
\end{equation}
The total fixed mass is calculated thanks to the ideal gas state equation
(see section~\ref{relations})
\begin{equation}
\mathrm{mass_{tot}} = \Mm[\mathrm{mix}] \frac{P}{\Rg T}
\label{tot_mass}
\end{equation}
with \Mm[\mathrm{mix}] calculated as seen in section~\ref{relations}.
Thus an initial guess of \massfrac\ is necessary.
Let's consider the situation
\begin{equation}
\ce{A + B ->[k_1] C + D ->[k_2] E + F}
\end{equation}
we have
\begin{equation}
\doverdt{\conc[C]} = k_1\conc[A]\conc[B] - k_2\conc[C]\conc[D]
\end{equation}
therefore, a first approximation can be
\begin{equation}
\conc[C] = \frac{k_1\conc[A]\conc[B]}{k_2\conc[D]} = \frac{\mathrm{prod}}{\mathrm{loss}}\conc[C]
\end{equation}
